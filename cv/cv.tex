\documentclass[english,10pt,a4paper]{scrartcl}

\usepackage{times}
\usepackage{fullpage}
\usepackage{hyperref} 
\usepackage{url}
\usepackage{tabularx}
\usepackage{multicol}
\usepackage{eurosym}
\usepackage{color}


\title{Alexis Roche}

\author{}
\date{}


\makeatletter
\newcommand{\matphi}{\boldsymbol{\Phi}}
\newcommand{\req}[1]{Equation~(\ref{#1})}
\newcommand*{\rom}[1]{\expandafter\@slowromancap\romannumeral #1@}

\newcommand{\heading}[1]{\vspace*{.75cm}\noindent{\Large{#1}}\vspace*{-.25cm}\newline\noindent\rule{\textwidth}{0.8pt}\vspace*{.25cm}}

\newcommand{\bighead}[3]{\noindent\begin{minipage}{.85\textwidth}{\textbf{#1}}\newline {\textbf{#2}}\end{minipage}\begin{minipage}{.15\textwidth}\begin{flushright}{\textbf{#3}}\end{flushright}\end{minipage}\newline\noindent\rule{.85\textwidth}{0.4pt}}

\newcommand{\head}[2]{{\noindent {\textbf{#1}} $\mid$ {\textbf{#2}}}}

\makeatother

\begin{document}

%\maketitle


\begin{flushright} 
{\huge {\textbf{Alexis Roche}}}\\
\begin{small}
+351 910 327 756 $\mid$ \url{alexis.roche@gmail.com}\\
\url{https://www.linkedin.com/in/alexis-roche-3815592}\\
French National $\mid$ DOB 12/8/1972 $\mid$ Married, 2 Children\\
\end{small}
\end{flushright}


\heading{Professional Profile}

\noindent Senior-level engineer and applied researcher with a specialist expertise in computer vision, medical image analysis and machine learning, operating in the fields of applied research, healthcare, electro-domestic and entertainment industry. Graduated from the French ``Grande \'Ecole" CentraleSupelec (formely, Centrale Paris), and holder of a PhD. Strong experience in developing research and commercial software. {\em Keywords:} artificial intelligence, image processing, computer vision, machine learning, deep learning, statistics, brain imaging, Python.



\heading{Key Skills}

\noindent \textbf{DATA SCIENCE:} In-depth knowledge of statistics, computer vision, medical image analysis, traditional machine learning and deep learning.

\ \\
\noindent \textbf{NEUROIMAGING:} Extensive experience in developing algorithms for structural and functional brain magnetic resonance imaging (MRI) in the form of open-source and commercial software for neuroscience, clinical research \& clinical practice. 

\ \\
\noindent \textbf{SCIENTIFIC COMPUTING:} Over ten years experience in Python programming using the NumPy/SciPy ecosystem and interfacing C~language with Python.

\ \\
\noindent \textbf{INTERACTION:} Proven ability to work in multidisciplinary environments at the interface of industry and academic research, as well as to develop successful collaborations with physicians and fundamental life science researchers.

\ \\
\noindent \textbf{SCIENTIFIC WRITING:} Main author of 25 peer-reviewed scientific articles published in international journals, proceedings and books, co-author of more than 100 scholarly publications and 5 international patents.


\heading{Career Summary}

\bighead{SENIOR COMPUTER VISION ENGINEER}{Didimo, Porto, Portugal}{2019--present}
\nopagebreak
\begin{itemize}
\item Developing computer vision algorithms for high-fidelity 3D avatars.
\end{itemize}
\ \\

\bighead{SENIOR COMPUTER VISION SCIENTIST}{CoVii, Ar\c celik/Beko group, Porto, Portugal}{2017--2019}
\nopagebreak
\begin{itemize}
\item Developed embedded object recognition and tracking algorithms for smart domestic appliances using traditional computer vision and deep learning (VUXHub and Artisan applications demonstrated at IFA Berlin, 2017--18).
\end{itemize}

\ \\
\bighead{LEAD CLINICAL RESEARCH -- ADVANCED CLINICAL IMAGING TECHNOLOGY}{Siemens Healthineers / Lausanne University Hospital (CHUV), Switzerland}{2011--2017}
\nopagebreak
\begin{itemize}
\item Led algorithmic development of the {\em MorphoBox} Siemens prototype for brain morphometry using anatomical MRI to help radiological reading for patients with suspected neurodegeneration.
\item Contributed a number of clinical validation studies of brain morphometry in collaboration with neurologists and radiologists, via statistical analysis and data mining techniques, in Alzheimer's disease and other dementia, mild cognitive impairment, multiple sclerosis, migraine, anorexia, \ldots 
\item Co-supervised three PhD students and two MSc students.
\item Main organizer of annual Siemens/CHUV {\em brain imaging} workshops in Lausanne, Switzerland, 2012--2016.
\end{itemize}


\ \\
\bighead{PERMANENT RESEARCHER -- NEUROIMAGING INSTITUTE}{French Atomic Commission (CEA), Paris, France}{2002--2011}
\nopagebreak
\begin{itemize}
\item Developed advanced algorithms for image processing and statistical analysis of functional, anatomical and diffusion-weighted MRI data (population analysis, spatio-temporal image registration, real-time imaging).
\item French ANR-funded project management: PI (Karametria project, statistical analysis of brain structures, budget: 620 K\euro, 2009--2011), team leader (NIBB project, studying language in infants via functional neuroimaging, 2006--2009).
\item Active contributor to the NiPy software library (Neuroimaging in Python, \url{www.nipy.org}) from 2006.
\item Main supervisor of two PhD students and four MSc students.
\item Worked as Academic Guest at Computer Vision Laboratory, Swiss Federal Institute of Technology Zurich (ETHZ), Switzerland, 2009--2011, on brain image registration. 
\end{itemize}

\ \\
\bighead{RESEARCH ASSISTANT -- WOLFSON MEDICAL VISION LABORATORY}{University of Oxford, UK}{2001--2002}
\nopagebreak
\begin{itemize}
\item Developed advanced medical image registration algorithms.
\item Consulting work for Mirada Solutions Ltd (now Siemens Molecular Imaging) on image-based deformation tracking (brain, liver).
\end{itemize}

\ \\
\bighead{CONTINGENT SCIENTIST -- FRENCH NATIONAL SERVICE}{General Directorate for Armament (DGA), Vernon, France}{1996--1997}
\nopagebreak
\begin{itemize}
\item Developed plane trajectory simulator using stochastic process models.
\end{itemize}

\heading{Education}
\nopagebreak
\head{PhD, Engineering Science (1997--2001)}{University of Nice-Sophia Antipolis, France}
\begin{itemize}
\item Funded by French National Research Institute (INRIA), Sophia Antipolis unit.
\item Developed generic algorithm for multimodal image registration, with applications in radiotherapy, image-guided surgery and neuroscience.
\end{itemize}

\ \\
\head{MSc, Cognitive Science (1995--1996)}{University Pierre \& Marie Curie, Paris \rom{6}, France}
\nopagebreak
\begin{itemize}
\item Internship at Experimental Psychology Laboratory, National Center for Scientific Research (CNRS), Paris, France, on modeling human perception of tempo using artificial neural networks.
\end{itemize}

\ \\
\head{Engineer Degree (equivalent MSc) (1993--1996)}{Ecole Centrale Paris (now CentraleSupelec), France}
\nopagebreak
\begin{itemize}
\item Third year specialization in Applied Mathematics.
\end{itemize}



%%\item Other training: Neuronal dynamics open course, EPFLx, Switzerland (2014).



\heading{Additional information}
\nopagebreak
\noindent 
\begin{minipage}{.15\textwidth}
\textbf{Languages:}
\end{minipage}
\begin{minipage}{.85\textwidth}
Native French, fluent English, basic Italian \& Portuguese. %German.
\end{minipage}

\ \\
\noindent 
\begin{minipage}{.15\textwidth}
\textbf{Programming:}
\ \\ 
\ \\
\ \\ 
\ \\
\ \\ 
\end{minipage}
\begin{minipage}{.85\textwidth}
{\tiny$\bullet$} C/C++ \\
{\tiny$\bullet$} Python (numpy, scipy, pylab, skimage, opencv, sklearn, TensorFlow, keras, cython...)\\ 
{\tiny$\bullet$} Matlab\\
{\tiny$\bullet$} R. \\
{\tiny$\bullet$} Version control software (git, svn).\\
\end{minipage}

\ \\
\noindent 
\begin{minipage}{.15\textwidth}
\textbf{Publications:}
\ \\ 
\ \\
\ \\
\ \\
\end{minipage}
\begin{minipage}{.85\textwidth}
{\tiny$\bullet$} 115 scholarly publications.\\
{\tiny$\bullet$} Full list available at ORCID: \url{http://orcid.org/0000-0002-4821-6893} \\
{\tiny$\bullet$} Google scholar statistics: 5754 citations, h-index: 32, i10-index: 56 (September 2019).\\
{\tiny$\bullet$} 5 published international patents. \\
\end{minipage}

%\ \\
%\noindent 
%\begin{minipage}{.15\textwidth}
%\textbf{ Reviewing:}
%\ \\ 
%\ \\
%\end{minipage}
%\begin{minipage}{.85\textwidth}
%IEEE Transactions on Medical Imaging, Medical Image Analysis, NeuroImage, Frontiers in
% Neuroscience, IEEE Transactions on Image Processing, Signal Image and Video Processing, MICCAI conference, \ldots
%\end{minipage}


\ \\
\noindent 
\begin{minipage}{.15\textwidth}
\textbf{Teaching:}
\ \\ 
\ \\
\  \\
\end{minipage}
\begin{minipage}{.85\textwidth}
{\tiny$\bullet$} Brain morphometry post-graduate lecture (2h), Lausanne: UNIL (2014/16), EPFL (2015/16).\\
{\tiny$\bullet$} Statistical analysis in neuroimaging, INSERM/CNRS continuing education, France (10h/year from 2003 to 2015).\\
\end{minipage}


\ \\
\noindent 
\begin{minipage}{.15\textwidth}
\textbf{References:}
\end{minipage}
\begin{minipage}{.85\textwidth}
Available on request.
\end{minipage}

%\newpage

%\section*{Publication list}
%\vspace*{-1cm}

%\subsection*{Book chapters}
%\subsection*{Journal papers}
%\subsection*{Peer-reviewed conference papers}
%\subsection*{Abstracts}
%\subsection*{Preprints}
%\subsection*{Theses}

\begin{thebibliography}{100}

\subsection*{Book chapters}

\bibitem{inbook:15}
\underline{A.~Roche}.
\newblock {\em {Image, Video \& 3D Data Registration: Medical, Satellite \&
  Video Processing Applications with Quality Metrics}}, chapter {Medical Image
  Registration Measures}.
\newblock Wiley, August 2015.

\bibitem{book:10}
J.-B. Poline, B.~Thirion, \underline{A.~Roche}, and S.~M\'eriaux.
\newblock {\em Foundational Issues in Human Brain Mapping}, chapter
  {Intersubject variability in fMRI data: causes, consequences, and related
  analysis strategies}, pages 173--192.
\newblock MIT press, 2010.

\bibitem{book:05}
X.~Pennec, N.~Ayache, \underline{A.~Roche}, and P.~Cachier.
\newblock {\em Multi-Sensor Image Fusion and Its Applications}, chapter
  Non-rigid {MR/US} registration for tracking brain deformations, pages
  107--143.
\newblock CRC Press, Boca Raton, USA, first edition, 2006.

\bibitem{book:02}
S.~Warfield, A.~Guimond, \underline{A.~Roche}, A.~Bharatha, A.~Tei, F.~Talos, J.~Rexilius,
  J.~Ruiz-Alzola, C.-F. Westin, S.~Haker, S.~Angenent, A.~Tannenbaum,
  F.~Jolesz, and R.~Kikinis.
\newblock {\em Brain Mapping: The Methods}, chapter Advanced Nonrigid
  Registration Algorithms for Image Fusion, pages 661--690.
\newblock Academic Press, San Diego, USA, second edition, 2002.


\subsection*{Journal papers}

\bibitem{flash:16}
\underline{A.~Roche}, B.~Mar\'echal, T.~Kober, G.~Krueger, P.~Hagmann, P.~Maeder, and
  R.~Meuli.
\newblock {Assessing brain volumes using MorphoBox prototype}.
\newblock {\em MAGNETOM Flash, the Siemens magazine of MRI: RSNA Edition},
  2017.

\bibitem{ijasp:13}
\underline{A.~Roche}.
\newblock {Approximate inference via variational sampling}.
\newblock {\em International Journal of Applied Statistics and Probability},
  1(3):110--120, 2013.

\bibitem{tmi:11}
\underline{A.~Roche}.
\newblock {A four-dimensional registration algorithm with application to joint
  correction of motion and slice timing in fMRI}.
\newblock {\em IEEE Transactions on Medical Imaging}, 30(8):1546--1554, 2011.

\bibitem{media:11}
\underline{A.~Roche}, D.~Ribes, M.~Bach~Cuadra, and G.~Krueger.
\newblock {On the Convergence of EM-Like Algorithms for Image Segmentation
  using Markov Random Fields}.
\newblock {\em Medical Image Analysis}, 15(6):830--839, 2011.

\bibitem{ni:07}
\underline{A.~Roche}, S.~M\'eriaux, M.~Keller, and B.~Thirion.
\newblock {Mixed-effects statistics for group analysis in fMRI: A nonparametric
  maximum likelihood approach}.
\newblock {\em Neuroimage}, 38:501--510, 2007.

\bibitem{tmi:01b}
\underline{A.~Roche}, X.~Pennec, G.~Malandain, and N.~Ayache.
\newblock {Rigid Registration of 3D Ultrasound with MR Images: a New Approach
  Combining Intensity and Gradient Information}.
\newblock {\em IEEE Transactions on Medical Imaging}, 20(10):1038--1049, 2001.

\bibitem{ijist:00}
\underline{A.~Roche}, G.~Malandain, and N.~Ayache.
\newblock {Unifying Maximum Likelihood Approaches in Medical Image
  Registration}.
\newblock {\em International Journal of Imaging Systems and Technology: Special
  issue on 3D imaging}, 11:71--80, 2000.

\bibitem{pv:16}
M.J. Fartaria, A.~Sorega, G.~Krueger, T.~Kober, C.~Granziera, M.~Bach~Cuadra,
  and \underline{A.~Roche}.
\newblock {Partial Volume-Aware Assessment of Multiple Sclerosis Lesions}.
\newblock 2018.
\newblock {\em NeuroImage:Clinical}.
\newblock Accepted.

\bibitem{plos:16}
G.~Bonnier, T.~Kober, M.~Schluep, R.~Du~Pasquier, G.~Krueger, R.~Meuli,
  C.~Granziera, and \underline{A.~Roche}.
\newblock {A New Approach for Deep Gray Matter Analysis Using Partial-Volume
  Estimation}.
\newblock {\em PlosOne}, 11(2):e0148631, February 2016.

\bibitem{plos:14}
K.~O'Brien, G.~Krueger, P.~Hagmann, P.~Maeder, T.~Kober, J.~Marques,
  F.~Lazeyras, and \underline{A.~Roche}.
\newblock {Robust T1-weighted structural brain imaging and morphometry at 7T
  using MP2RAGE}.
\newblock {\em PlosOne}, 9(6):e99676, June 2014.

\bibitem{neuroradiology:17}
J.~Boto, G.~Gkinis, \underline{A.~Roche}, T.~Kober, B.~Mar{\'e}chal, N.~Ortiz, K.-O.~L{\"o}vblad, F.~Lazeyras, and M.I.~Vargas.
\newblock {Evaluating anorexia-related brain atrophy using MP2RAGE-based morphometry}.
\newblock {\em European Radiology}, 27(12):5064--5072, 2017.

\bibitem{investigative_radiology:17}
M.J.~Fartaria, K.~O'Brien, A.~Sorega, G.~Bonnier, \underline{A.~Roche}, P.~Falkovskiy, G.~Krueger, T.~Kober, M.~Bach Cuadra, and C.~Granziera.
\newblock {An ultra-high field study of cerebellar pathology in early relapsing-remitting multiple sclerosis using MP2RAGE}.
\newblock {\em Investigative radiology}, 52(5):265--273, 2017.

\bibitem{ni:16}
P.~Falkovskiy, D.~Brenner, T.~Feiweier, S.~Kannengiesser, B.~Mar\'echal,
  T.~Kober, \underline{A.~Roche}, K.~Thostenson, R.~Meuli, D.~Reyes, T.~Stoecker, M.A.
  Bernstein, J.-P. Thiran, and G.~Krueger.
\newblock {Comparison of accelerated T1-weighted whole-brain structural-imaging
  protocols}.
\newblock {\em Neuroimage}, 124:157--167, 2016.

\bibitem{neurorad:16}
S.~Haller, P.~Falkovskiy, R.~Meuli, J.-P. Thiran, G.~Krueger, K.-O. Lovblad,
  \underline{A.~Roche}, T.~Kober, and B.~Mar\'echal.
\newblock {Basic MR sequence parameters systematically bias automated brain
  volume estimation}.
\newblock {\em Neuroradiology}, 2016.

\bibitem{jmri:15}
M.J. Fartaria, G.~Bonnier, \underline{A.~Roche}, T.~Kober, R.~Meuli, D.~Rotzinger,
  R.~Frackowiak, M.~Schluep, R.~Du~Pasquier, J.-P. Thiran, G.~Krueger,
  M.~Bach~Cuadra, and C.~Granziera.
\newblock Automated detection of white matter and cortical lesions in early
  stages of multiple sclerosis.
\newblock {\em Journal of Magnetic Resonance Imaging}, 43(6):1445--1454, 2015.

\bibitem{biomed:15}
G.~Bonnier, \underline{A.~Roche}, D.~Romascano, S.~Simioni, D.E. Meskaldji, D.~Rotzinger,
  Y.-C. Lin, G.~Menegaz, M.~Schluep, T.J. Du~Pasquier, R.~Sumpf, J.~Frahm,
  J.-P. Thiran, G.~Krueger, and C.~Granziera.
\newblock {Multicontrast MRI quantification of focal inflammation and
  degeneration in multiple sclerosis}.
\newblock {\em BioMed research international}, 2015(Article ID 569123), 2015.

\bibitem{nic:15}
C.~Granziera, A.~Daducci, A.~Donati, G.~Bonnier, D.~Romascano, \underline{A.~Roche},
  M.~Bach~Cuadra, D.~Schmitter, S.~Kl{\"o}ppel, R.~Meuli, A.~von Gunten, and
  G.~Krueger.
\newblock {A multi-contrast MRI study of microstructural brain damage in
  patients with mild cognitive impairment}.
\newblock {\em NeuroImage: Clinical}, 8:631--639, 2015.

\bibitem{nic:14}
D.~Schmitter, \underline{A.~Roche}, B.~Mar\'echal, D.~Ribes, A.~Abdulkadir, M.~Bach~Cuadra,
  A.~Daducci, C.~Granziera, S.~Kl\"oppel, P.~Maeder, R.~Meuli, and G.~Krueger.
\newblock {An evaluation of volume-based morphometry for prediction of mild
  cognitive impairment and Alzheimer's disease}.
\newblock {\em NeuroImage: Clinical}, 7:7--17, 2015.

\bibitem{hbm:15}
D.~Romascano, D.E. Meskaldji, G.~Bonnier, S.~Simioni, D.~Rotzinger, Y.-C. Lin,
  G.~Menegaz, \underline{A.~Roche}, M.~Schluep, R.~Du~Pasquier, J.~Richiardi, D.~Van
  De~Ville, A.~Daducci, T.~Sumpf, J.~Fraham, J.-P. Thiran, G.~Krueger, and
  C.~Granziera.
\newblock {Multicontrast connectometry: A new tool to assess cerebellum
  alterations in early relapsing-remitting multiple sclerosis}.
\newblock {\em Human Brain Mapping}, 36(4):1609--1619, 2015.

\bibitem{actn:14}
G.~Bonnier, \underline{A.~Roche}, D.~Romascano, S.~Simioni, D.~Meskaldji, D.~Rotzinger,
  Y.-C. Lin, G.~Menegaz, M.~Schluep, R.~Du~Pasquier, T.~Sumpf, J.~Frahm, J.-P.
  Thiran, G.~Krueger, and C.~Granziera.
\newblock {Advanced MRI unravels the nature of tissue alterations in early
  multiple sclerosis}.
\newblock {\em Annals of clinical and translational neurology}, 1(6):423--432,
  2014.

\bibitem{hbm:14}
C.~Granziera, A.~Daducci, D.~Romascano, \underline{A.~Roche}, G.~Helms, G.~Krueger, and
  N.~Hadjikhani.
\newblock {Structural abnormalities in the thalamus of migraineurs with aura: A
  multiparametric study at 3 T}.
\newblock {\em Human Brain Mapping}, 35(4):1461--1468, 2013.

\bibitem{cerebellum:13}
C.~Granziera, D.~Romascano, A.~Daducci, \underline{A.~Roche}, M.~Vincent, G.~Krueger, and
  N.~Hadjikhani.
\newblock {Migraineurs without aura show microstructural abnormalities in the
  cerebellum and frontal lobe}.
\newblock {\em The Cerebellum}, 12(6):812--818, 2013.

\bibitem{neurology:12}
C.~Granziera, A.~Daducci, D.~Meskaldji, \underline{A.~Roche}, P.~Maeder, P.~Michel,
  N.~Hadjikhani, G.~Sorensen, R.~Frackowiak, J.-P. Thiran, R.~Meuli, and
  G.~Krueger.
\newblock {A new early and automated MRI-based predictor of motor improvement}.
\newblock {\em Neurology}, 79(1):39--46, 2012.

\bibitem{sinica:08}
M.~Keller, \underline{A.~Roche}, A.~Tucholka, and B.~Thirion.
\newblock {Dealing with Spatial Normalization Errors in fMRI Group Inference
  using Hierarchical Modeling}.
\newblock {\em Statistica Sinica}, 18(4):1357--1374, 2008.

\bibitem{media:08}
C.~Poupon, \underline{A.~Roche}, J.~Dubois, J.-F. Mangin, and F.~Poupon.
\newblock {Real-Time MR Diffusion Tensor and Q-ball Imaging using Kalman
  Filtering}.
\newblock {\em Medical Image Analysis}, 12(5):527--534, 2008.
\newblock Elsevier Medical Image Analysis -- MICCAI'07 best paper.

\bibitem{tmi:07}
E.~Duchesnay, A.~Cachia, \underline{A.~Roche}, D.~Rivi\`ere, Y.~Cointepas,
  D.~Papadopoulos-Orfanos, M.~Zilbovicius, J.-L. Martinot, and J.-F. Mangin.
\newblock Classification from cortical folding patterns.
\newblock {\em IEEE Transactions on Medical Imaging}, 26(4):553--565, 2007.

\bibitem{ni:07b}
B.~Thirion, P.~Pinel, S.~M\'eriaux, \underline{A.~Roche}, S.~Dehaene, and J.-B. Poline.
\newblock {Analysis of a Large fMRI cohort: Statistical and Methodological
  Issues for Group Analyses}.
\newblock {\em Neuroimage}, 35(1):105--120, 2007.

\bibitem{tmi:07b}
B.~Thirion, P.~Pinel, A.~Tucholka, \underline{A.~Roche}, P.~Ciuciu, J.-F. Mangin, and J.-B.
  Poline.
\newblock {Structural Analysis of fMRI Data Revisited: Improving the
  Sensitivity and Reliability of fMRI Group Studies}.
\newblock {\em IEEE Transactions on Medical Imaging}, 26(9):1256--1269, 2007.

\bibitem{pnas:06}
G.~Dehaene-Lambertz, L.~Hertz-Pannier, J.~Dubois, S.~M\'eriaux, \underline{A.~Roche}, and
  M.~Sigman.
\newblock Functional organization of perisylvian activation during presentation
  of sentences in preverbal infants.
\newblock {\em Proceedings of the National Academy of Sciences},
  103:14240--14245, 2006.

\bibitem{hbm:06}
S.~M\'eriaux, \underline{A.~Roche}, G.~Dehaene-Lambertz, B.~Thirion, and J.-B. Poline.
\newblock {Combined permutation test and mixed-effect model for group average
  analysis in fMRI}.
\newblock {\em Human Brain Mapping}, 27(5):402--410, 2006.

\bibitem{hbm:06b}
B.~Thirion, G.~Flandin, P.~Pinel, \underline{A.~Roche}, and J.-B. Ciuciu, P.~Poline.
\newblock {Dealing with the shortcomings of spatial normalization:
  Multi-subject parcellation of fMRI datasets}.
\newblock {\em Human Brain Mapping}, 27(8):678--693, August 2006.

\bibitem{pls:05}
J.-B. Poline, A.~Ciuciu, \underline{A.~Roche}, and B.~Thirion.
\newblock Quelle confiance accorder aux images du cerveau en action?
\newblock {\em Pour la Science}, 338:138--142, 2005.

\bibitem{tmi:02}
L.~Freire, \underline{A.~Roche}, and J.-F. Mangin.
\newblock {What is the best similarity measure for motion correction in fMRI
  time series?}
\newblock {\em IEEE Transactions on Medical Imaging}, 21(5):470--484, 2002.

\bibitem{tmi:01}
A.~Guimond, \underline{A.~Roche}, N.~Ayache, and J.~Meunier.
\newblock {Multimodal Brain Warping Using the Demons Algorithm and Adaptative
  Intensity Corrections}.
\newblock {\em IEEE Transactions on Medical Imaging}, 20(1):58--69, 2001.

\bibitem{ivc:00}
S.~Ourselin, \underline{A.~Roche}, G.~Subsol, X.~Pennec, and N.~Ayache.
\newblock {Reconstructing a 3D Structure from Serial Histological Sections}.
\newblock {\em Image and Vision Computing}, 19(1--2):25--31, 2000.


\subsection*{Peer-reviewed conference papers}

\bibitem{miccai:14}
\underline{A.~Roche} and F.~Forbes.
\newblock {Partial volume estimation in brain MRI revisited}.
\newblock In {\em Medical Image Computing and Computer-Assisted Intervention
  (MICCAI'14)}, volume 8673 of {\em Lecture Notes in Computer Science}, pages
  771--778, Boston, USA, September 2014. Springer Verlag.

\bibitem{miccai:12}
\underline{A.~Roche}.
\newblock {Closed-Form Relaxation for MRF-MAP Tissue Classification Using
  Discrete Laplace Equations}.
\newblock In {\em Medical Image Computing and Computer-Assisted Intervention
  (MICCAI'12)}, volume 7511 of {\em Lecture Notes in Computer Science}, pages
  355--362, Nice, France, October 2012. Springer Verlag.

\bibitem{isbi:04}
\underline{A.~Roche}, P.-J. Layahe, and J.-B. Poline.
\newblock {Incremental Activation Detection in fMRI Time Series using Kalman
  Filtering}.
\newblock In {\em International Symposium on Biomedical Imaging (ISBI'04)},
  pages 376--379, Washington, DC, USA, April 2004.

\bibitem{miccai:04}
\underline{A.~Roche}, P.~Pinel, S.~Dehaene, and J.-B. Poline.
\newblock {Solving Incrementally the Fitting and Detection Problems in fMRI
  Time Series}.
\newblock In {\em Medical Image Computing and Computer-Assisted Intervention
  (MICCAI'04)}, volume 3217 of {\em Lecture Notes in Computer Science}, pages
  719--726, Saint-Malo, France, September 2004. Springer Verlag.

\bibitem{eccv:00}
\underline{A.~Roche}, A.~Guimond, N.~Ayache, and J.~Meunier.
\newblock {Multimodal Elastic Matching of Brain Images}.
\newblock In {\em European Conference on Computer Vision (ECCV'00)}, volume
  1843 of {\em Lecture Notes in Computer Science}, pages 511--527, Dublin,
  Ireland, June 2000. Springer Verlag.

\bibitem{miccai:00}
\underline{A.~Roche}, X.~Pennec, M.~Rudolph, D.~P. Auer, G.~Malandain, S.~Ourselin, L.~M.
  Auer, and N.~Ayache.
\newblock {Generalized Correlation Ratio for Rigid Registration of 3D
  Ultrasound with MR Images}.
\newblock In S.~Delp, A.M. DiGioia, and B.~Jaramaz, editors, {\em Medical Image
  Computing and Computer-Assisted Intervention (MICCAI'00)}, volume 1935 of
  {\em Lecture Notes in Computer Science}, pages 567--577, Pittsburgh, USA,
  October 2000. Springer Verlag.

\bibitem{miccai:99}
\underline{A.~Roche}, G.~Malandain, N.~Ayache, and S.~Prima.
\newblock {Towards a Better Comprehension of Similarity Measures used in
  Medical Image Registration}.
\newblock In {\em Medical Image Computing and Computer-Assisted Intervention
  (MICCAI'99)}, volume 1679 of {\em Lecture Notes in Computer Science}, pages
  555--566, Cambridge, UK, September 1999. Springer Verlag.

\bibitem{miccai:98}
\underline{A.~Roche}, G.~Malandain, X.~Pennec, and N.~Ayache.
\newblock {The Correlation Ratio as a New Similarity Measure for Multimodal
  Image Registration}.
\newblock In {\em Medical Image Computing and Computer-Assisted Intervention
  (MICCAI'98)}, volume 1496 of {\em Lecture Notes in Computer Science}, pages
  1115--1124, Cambridge, USA, October 1998. Springer Verlag.

\bibitem{miccai:09}
M.~Keller, M.~Lavielle, M.~Perrot, and \underline{A.~Roche}.
\newblock {Anatomically Informed Bayesian Model Selection for fMRI Group Data
  Analysis}.
\newblock In {\em Medical Image Computing and Computer-Assisted Intervention
  (MICCAI'09)}, volume 5762 of {\em Lecture Notes in Computer Science}, pages
  450--457, London, UK, September 2009. Springer Verlag.

\bibitem{isbi:08}
M.~Keller and \underline{A.~Roche}.
\newblock {Increased Sensitivity in fMRI Group Analyses using Mixed-Effect
  Modeling}.
\newblock In {\em International Symposium on Biomedical Imaging (ISBI'08)},
  pages 548--551, Paris, France, May 2008.

\bibitem{isbi:08b}
P.~Ciuciu, T.~Vincent, A.-L. Foulque, and \underline{A.~Roche}.
\newblock {Improved fMRI group studies based on spatially varying
  non-parametric BOLD signal modeling}.
\newblock In {\em International Symposium on Biomedical Imaging (ISBI'08)},
  pages 1263--1266, Paris, France, May 2008.

\bibitem{isbi:07}
P.~Ciuciu, P.~Abry, C.~Rabrait, H.~Wendt, and \underline{A.~Roche}.
\newblock {Leader-Based Multifractal Analysis for Evi fmri Time Series: Ongoing
  vs Task-Related Brain Activity}.
\newblock In {\em International Symposium on Biomedical Imaging (ISBI'07)},
  pages 404--407, Washington, DC, USA, May 2007.

\bibitem{miccai:01b}
S.~Granger, X.~Pennec, and \underline{A.~Roche}.
\newblock {Rigid Point-Surface Registration using an EM Variant of ICP for
  Computed Guided Oral Implantology}.
\newblock In W.J. Niessen and M.A. Viergever, editors, {\em Medical Image
  Computing and Computer-Assisted Intervention (MICCAI'01)}, volume 2208 of
  {\em Lecture Notes in Computer Science}, pages 752--761, Utrecht,
  Netherlands, October 2001. Springer Verlag.

\bibitem{miccai:08}
M.~Bhattacharjee, A.~Pitiot, \underline{A.~Roche}, D.~Dormont, and E.~Bardinet.
\newblock {Anatomy-Preserving Nonlinear Registration of Deep Brain ROIs using
  Confidence-Based Block-Matching}.
\newblock In {\em Medical Image Computing and Computer-Assisted Intervention
  (MICCAI'08)}, Lecture Notes in Computer Science, New York, USA, September
  2008. Springer Verlag.

\bibitem{miccai:07}
C.~Poupon, F.~Poupon, \underline{A.~Roche}, Y.~Cointepas, J.~Dubois, and J.-F. Mangin.
\newblock {Real-Time MR Diffusion Tensor and Q-Ball Imaging Using Kalman
  Filtering}.
\newblock In {\em Medical Image Computing and Computer-Assisted Intervention
  (MICCAI'07)}, volume 4791 of {\em Lecture Notes in Computer Science}, pages
  27--35, Brisbane, Australia, November 2007. Springer Verlag.

\bibitem{ipmi:07}
B.~Thirion, A.~Tucholka, M.~Keller, P.~Pinel, \underline{A.~Roche}, J.-F. Mangin, and J.-B.
  Poline.
\newblock {High Level Group Analysis of fMRI Data Based on Dirichlet Process
  Mixture Models}.
\newblock In {\em Information Processing in Medical Imaging (IPMI'07)}, Lecture
  Notes in Computer Science, pages 482--494, Kerkrade, Netherlands, 2007.
  Springer Verlag.

\bibitem{isbi:06}
S.~M\'eriaux, \underline{A.~Roche}, B.~Thirion, and G.~Dehaene-Lambertz.
\newblock Robust statistics for nonparametric group analysis in {fMRI}.
\newblock In {\em International Symposium on Biomedical Imaging (ISBI'06)},
  pages 936--939, Washington, DC, USA, April 2006.

\bibitem{mmbia:06}
B.~Thirion, \underline{A.~Roche}, P.~Ciuciu, and J.-B. Poline.
\newblock {Improving sensitivity and reliability of fMRI group studies through
  high level combination of individual subjects results}.
\newblock In {\em Mathematical Methods in Biomedical Image Analysis
  (MMBIA'06)}, page~62, New York, USA, June 2006.

\bibitem{isbi:04b}
P.~Ciuciu, J.~Idier, \underline{A.~Roche}, and C.~Pallier.
\newblock Outlier detection for robust region-based estimation of the
  hemodynamic response function in event-related f{MRI}.
\newblock In {\em International Symposium on Biomedical Imaging (ISBI'04)},
  pages 392--395, Washington, DC, USA, April 2004.

\bibitem{isbi:04c}
E.~Duchesnay, \underline{A.~Roche}, D.~Rivi\`ere, D.~Papadopoulos-Orfanos, Y.~Cointepas,
  and J.-F. Mangin.
\newblock Population classification based on structural morphometry of cortical
  sulci.
\newblock In {\em International Symposium on Biomedical Imaging (ISBI'04)},
  pages 1276--1279, Washington, DC, USA, 2004.

\bibitem{miccai:02}
E.~Bardinet, P.~Cathier, \underline{A.~Roche}, and D.~Dormont.
\newblock {A Posteriori Validation of Pre-Operative Planning in Functional
  Neurosurgery By Quantification of Brain Pneumocephalus}.
\newblock In W.J. Niessen and M.A. Viergever, editors, {\em Medical Image
  Computing and Computer-Assisted Intervention (MICCAI'02)}, volume 2488(1) of
  {\em Lecture Notes in Computer Science}, pages 323--330, Tokyo, Japan,
  November 2002. Springer Verlag.

\bibitem{miccai:01}
E.~Bardinet, A.C.F. Colchester, \underline{A.~Roche}, Y.~Zhu, Y.~He, S.~Ourselin,
  B.~Nailon, S.A. Hojjat, J.~Ironside, S.~Al-Sarraj, N.~Ayache, and J.~Wardlaw.
\newblock {Registration of Reconstructed Post Mortem Optical Data with MR Scans
  of the Same Patient}.
\newblock In W.J. Niessen and M.A. Viergever, editors, {\em Medical Image
  Computing and Computer-Assisted Intervention (MICCAI'01)}, volume 2208 of
  {\em Lecture Notes in Computer Science}, pages 957--965, Utrecht,
  Netherlands, October 2001. Springer Verlag.

\bibitem{miccai:01d}
H.~Delingette, E.~Bardinet, D.~Rey, J.-D. Lemarechal, J.~Montagnat,
  S.~Ourselin, \underline{A.~Roche}, D.~Dormont, J.~Yelnik, and N.~Ayache.
\newblock {YAV++: a Software Platform for Medical Image Processing and
  Visualization}.
\newblock In {\em Workshop on Interactive Medical Image Visualization and
  Analysis, satellite symposia of MICCAI'01}, Utrecht, Netherlands, October
  2001.

\bibitem{miccai:01c}
S.~Ourselin, E.~Bardinet, D.~Dormont, G.~Malandain, \underline{A.~Roche}, and N.~Ayache.
\newblock {Fusion of Histological Sections and MR Images: Towards the
  Construction of an Atlas of the Human Basal Ganglia}.
\newblock In W.J. Niessen and M.A. Viergever, editors, {\em Medical Image
  Computing and Computer-Assisted Intervention (MICCAI'01)}, volume 2208 of
  {\em Lecture Notes in Computer Science}, pages 743--751, Utrecht,
  Netherlands, October 2001. Springer Verlag.

\bibitem{miar:01}
X.~Pennec, N.~Ayache, \underline{A.~Roche}, and P.~Cachier.
\newblock {Non-rigid MR/US registration for tracking brain deformations}.
\newblock In {\em International Workshop on Medical Imaging and Augmented
  Reality (MIAR'01)}, pages 79--86, Shatin, Hong Kong, June 2001. IEEE Computer
  Society Press.

\bibitem{miccai:00c}
A.~Colchester, S.~Ourselin, Y.~Zhu, E.~Bardinet, Y.~He, \underline{A.~Roche}, S.~Al-Sarraj,
  B.~Nailon, J.~Ironside, and N.~Ayache.
\newblock {3-D Reconstruction of Macroscopic Optical Brain Slice Images}.
\newblock In S.~Delp, A.M. DiGioia, and B.~Jaramaz, editors, {\em Medical Image
  Computing and Computer-Assisted Intervention (MICCAI'00)}, volume 1935 of
  {\em Lecture Notes in Computer Science}, pages 95--105, Pittsburgh, USA,
  October 2000. Springer Verlag.

\bibitem{paris:00}
O.~Migneco, G.~Malandain, \underline{A.~Roche}, I.~Dygai, F.~Bussiere, N.~Ayache, and
  J.~Darcourt.
\newblock Absolute calibration of hmpao spect using xenon 133.
\newblock In {\em Congr\`es Europ\'een de M\'edecine Nucl\'eaire}, Paris,
  France, September 2000.

\bibitem{miccai:00b}
S.~Ourselin, \underline{A.~Roche}, S.~Prima, and N.~Ayache.
\newblock {Block Matching: a General Framework to Improve Robustness of Rigid
  Registration of Medical Images}.
\newblock In S.~Delp, A.M. DiGioia, and B.~Jaramaz, editors, {\em Medical Image
  Computing and Computer-Assisted Intervention (MICCAI'00)}, volume 1935 of
  {\em Lecture Notes in Computer Science}, pages 557--566, Pittsburgh, USA,
  October 2000. Springer Verlag.

\bibitem{wbir:99}
S.~Ourselin, \underline{A.~Roche}, G.~Subsol, and X.~Pennec.
\newblock {Automatic Alignment of Histological Sections}.
\newblock In F.~Pernus, S.~Kovacic, H.S. Stiehl, and M.A. Viergever, editors,
  {\em International Workshop on Biomedical Image Registration (WBIR'99)},
  pages 1--13, Bled, Slovenia, August 1999.


\subsection*{Abstracts}

\bibitem{adpd:17}
\underline{A.~Roche}, D.~Damian, F.~Pedron, B.~Mar\'echal, P.~Hagmann, P.~Maeder, R.~Meuli,
  T.~Kober, and J.-F. D\'emonet.
\newblock {Automated prediction of typical and mixed Alzheimer's disease using
  combined routine brain volumetry and cognitive assessment}.
\newblock In {\em Submitted to AD-PD conference}, Vienna, 2017.

\bibitem{ismrm:14}
\underline{A.~Roche}, D.~Schmitter, B.~Mar\'echal, D.~Ribes, A.~Abdulkadir, M.~Bach~Cuadra,
  A.~Daducci, C.~Granziera, S.~Kl\"oppel, P.~Maeder, R.~Meuli, and G.~Krueger.
\newblock {Volume-based vs. voxel-based brain morphometry in Alzheimer's
  disease prediction}.
\newblock In {\em Annual Meeting of the International Society for Magnetic
  Resonance in Medicine (ISMRM'14)}, Milan, Italy, 2014.
\newblock The first two authors contributed equally.

\bibitem{ohbm:05}
\underline{A.~Roche}.
\newblock {When is it a good time to smooth fMRI data?}
\newblock In {\em Annual Meeting of the Organization for Human Brain Mapping
  (OHBM'05) CD-Rom Neuroimage 26(1)}, Toronto, Canada, June 2005.

\bibitem{ohbm:04}
\underline{A.~Roche} and J.-B. Poline.
\newblock {Kalman Filtering for Real-Time fMRI Analysis}.
\newblock In {\em Annual Meeting of the Organization for Human Brain Mapping
  (OHBM'04) CD-Rom Neuroimage 22(1)}, Budapest, Hungary, June 2004.

\bibitem{ohbm:03}
\underline{A.~Roche}, F.~Kherif, G.~Flandin, and J.-B. Poline.
\newblock Should {fMRI} data be analyzed using a single {BOLD} response model
  across regions?
\newblock In {\em Annual Meeting of the Organization for Human Brain Mapping
  (OHBM'03) CD-Rom Neuroimage 19(2)}, New York, USA, June 2003.

\bibitem{ismrm:16}
P.~Falkovskiy, B.~Mar\'echal, T.~Kober, P.~Maeder, R.~Meuli, J.-P. Thiran, and
  \underline{A.~Roche}.
\newblock {Impact of image acquisition systems on Alzheimer's disease-related
  atrophy detection}.
\newblock In {\em Annual Meeting of the International Society for Magnetic
  Resonance in Medicine (ISMRM'16)}, Singapore, 2016.

\bibitem{ismrm:16b}
P.~Falkovskiy, B.~Mar\'echal, S.~Yan, Z.~Jin, T.~Qian, K.~O'Brien, R.~Meuli,
  J.-P. Thiran, G.~Krueger, T.~Kober, and \underline{A.~Roche}.
\newblock {Quantitative comparison of MP2RAGE skull-stripping strategies}.
\newblock In {\em Annual Meeting of the International Society for Magnetic
  Resonance in Medicine (ISMRM'16)}, Singapore, 2016.

\bibitem{ohbm:16}
P.~Falkovskiy, B.~Mar\'echal, P.~Maeder, T.~Kober, J.-P. Thiran, R.~Meuli, and
  \underline{A.~Roche}.
\newblock {Correcting Effects of Magnetic Resonance Field Strength on Brain
  Volumetry}.
\newblock In {\em Annual Meeting of the Organization for Human Brain Mapping
  (OHBM'16)}, Geneva, Switzerland, 2016.

\bibitem{ismrm:15}
G.~Bonnier, J.-P Thiran, G.~Krueger, C.~Granziera, and \underline{A.~Roche}.
\newblock {Concentration maps improve detection of gray matter alteration in
  cerebellum and deep gray matter structures}.
\newblock In {\em Annual Meeting of the International Society for Magnetic
  Resonance in Medicine (ISMRM'15)}, Toronto, Canada, 2015.

\bibitem{ismrm:13}
K.~O'Brien, G.~Krueger, F.~Lazeyras, R.~Grueter, and \underline{A.~Roche}.
\newblock {A simple method to denoise MP2RAGE}.
\newblock In {\em Annual Meeting of the International Society for Magnetic
  Resonance in Medicine (ISMRM'13)}, Salt Lake City, USA, 2013.

\bibitem{ismrm:11}
D.~Ribes, B.~Mortamet, M.~Bach~Cuadra, C.R. Jack, R.~Meuli, G.~Krueger, and
  \underline{A.~Roche}.
\newblock {Comparison of tissue classification models for automatic brain MR
  segmentation}.
\newblock In {\em Annual Meeting of the International Society for Magnetic
  Resonance in Medicine (ISMRM'11)}, Montreal, Canada, 2011.

\bibitem{ohbm:09}
L.~Favre, A.-L. Fouque, T.~Vincent, A.~Tucholka, M.~Keller, G.~Operto,
  B.~Thyreau, C.~Clouchoux, L.~Risser, A.~Moreno, D.~Geffroy, Y.~Cointepas,
  O.~Coulon, P.~Ciuciu, B.~Thirion, and \underline{A.~Roche}.
\newblock {A Comprehensive fMRI Processing Toolbox for BrainVISA}.
\newblock In {\em Annual Meeting of the Organization for Human Brain Mapping
  (OHBM'09)}, San Francisco, USA, 2009.

\bibitem{ismrm:16c}
S.~Haller, P.~Falkovskiy, R.~Meuli, J.-P. Thiran, G.~Krueger, K.-O. Lovblad,
  \underline{A.~Roche}, T.~Kober, and B.~Mar\'echal.
\newblock {Basic MR sequence parameters systematically bias automated brain
  volume estimation}.
\newblock In {\em Annual Meeting of the International Society for Magnetic
  Resonance in Medicine (ISMRM'16)}, Singapore, 2016.

\bibitem{ismrm:16d}
M.J. Fartaria, G.~Bonnier, T.~Kober, \underline{A.~Roche}, D.~Rotzinger, M.~Schluep,
  R.~Du~Pasquier, J.-P. Thiran, G.~Krueger, R.~Meuli, M.~Bach~Cuadra, and
  C.~Granziera.
\newblock {Longitudinal automated detection of white matter and cortical
  lesions in relapsing-remitting multiple sclerosis}.
\newblock In {\em Annual Meeting of the International Society for Magnetic
  Resonance in Medicine (ISMRM'16)}, Singapore, 2016.

\bibitem{ismrm:16e}
M.J. Fartaria, G.~Bonnier, T.~Kober, K.~O'Brien, \underline{A.~Roche}, B.~Mar\'echal,
  D.~Rotzinger, R.~Meuli, J.-P. Thiran, G.~Krueger, M.~Bach~Cuadra, and
  C.~Granziera.
\newblock {Focal cerebellar pathology in early relapsing-remitting multiple
  sclerosis patients: a MP2RAGE study at 3T and 7T MRI }.
\newblock In {\em Annual Meeting of the International Society for Magnetic
  Resonance in Medicine (ISMRM'16)}, Singapore, 2016.

\bibitem{ismrm:15b}
M.J. Fartaria, G.~Bonnier, \underline{A.~Roche}, T.~Kober, R.~Meuli, D.~Rotinger,
  M.~Schluep, R.~Du~Pasquier, J.-P. Thiran, G.~Krueger, C.~Granziera, and
  M.~Bach~Cuadra.
\newblock {Exploration of advanced MR imaging contrasts for automated detection
  of white matter and cortical lesions in early-stages of multiple sclerosis}.
\newblock In {\em Annual Meeting of the International Society for Magnetic
  Resonance in Medicine (ISMRM'15)}, Toronto, Canada, 2015.

\bibitem{ismrm:15c}
Y.~Wang, B.~Mar\'echal, D.~Neumann, \underline{A.~Roche}, J.D. West, B.C. McDonald, M.A.
  Keiski, D.J. Smith, A.J. Saykin, and G.~Krueger.
\newblock {Detecting atrophy in chronic moderate and severe traumatic brain
  injury using an automated volume-based morphometry toolbox}.
\newblock In {\em Annual Meeting of the International Society for Magnetic
  Resonance in Medicine (ISMRM'15)}, Toronto, Canada, 2015.

\bibitem{ismrm:14b}
D.~Romascano, D.E. Meskaldji, G.~Bonnier, S.~Simioni, D.~Rotzinger, Y.-C. Lin,
  G.~Menegaz, \underline{A.~Roche}, M.~Schluep, and R.~Du~Pasquier.
\newblock {Cerebellar Connectomics Provide New Biomarkers in Early Multiple
  Sclerosis}.
\newblock In {\em Annual Meeting of the International Society for Magnetic
  Resonance in Medicine (ISMRM'14)}, Milan, Italy, 2014.

\bibitem{ismrm:14c}
G.~Bonnier, \underline{A.~Roche}, D.~Romascano, S.~Simioni, D.E. Meskaldji, D.~Rotzinger,
  Y.-C. Lin, G.~Menegaz, M.~Schluep, and R.~Du~Pasquier.
\newblock {Multiple Sclerosis Lesion Fingerprint Using Multicontrast MRI}.
\newblock In {\em Annual Meeting of the International Society for Magnetic
  Resonance in Medicine (ISMRM'14)}, Milan, Italy, 2014.

\bibitem{ismrm:13b}
M.~Bach~Cuadra, S.~Gelin, \underline{A.~Roche}, O.~Esteban, J.~Kober, T.~Marques,
  C.~Granziera, and G.~Krueger.
\newblock {Classical segmentation methods on novel MR imaging: a study of brain
  tissue segmentation of MP2RAGE vs MPRAGE}.
\newblock In {\em Annual Meeting of the International Society for Magnetic
  Resonance in Medicine (ISMRM'13)}, Salt Lake City, USA, 2013.

\bibitem{ismrm:13c}
G.~Bonnier, T.~Stumpf, D.~Romascano, \underline{A.~Roche}, M.~Schluep, R.~Dupasquier,
  C.~Granziera, and G.~Krueger.
\newblock {Ultrafast T2 mapping of multiple sclerosis pathology in early
  disease}.
\newblock In {\em Annual Meeting of the International Society for Magnetic
  Resonance in Medicine (ISMRM'13)}, Salt Lake City, USA, 2013.

\bibitem{ismrm:12}
T.~Kober, \underline{A.~Roche}, D.~Ribes, R.~Meuli, and G.~Krueger.
\newblock {Atlas-free brain tissue segmentation using a single T1-weighted MRI
  acquisition}.
\newblock In {\em Annual Meeting of the International Society for Magnetic
  Resonance in Medicine (ISMRM'12)}, Melbourne, Australia, 2012.

\bibitem{ismrm:12b}
B.~Mar\'echal, T.~Kober, T.~Hilbert, D.~Ribes, N.~Chevrey, \underline{A.~Roche}, J.-P.
  Thiran, R.~Meuli, and G.~Krueger.
\newblock {Automated quality control in MR-based brain morphometry}.
\newblock In {\em Annual Meeting of the International Society for Magnetic
  Resonance in Medicine (ISMRM'12)}, Melbourne, Australia, 2012.

\bibitem{ohbm:09b}
M.~Brett, J.~Taylor, C.~Burns, K.J. Millman, F.~Perez, \underline{A.~Roche}, B.~Thirion,
  and M.~D'Esposito.
\newblock {NIPY: an open library and development framework for FMRI data
  analysis}.
\newblock In {\em Annual Meeting of the Organization for Human Brain Mapping
  (OHBM'09) Neuroimage 47:S196}, San Francisco, USA, 2009.

\bibitem{ismrm:08}
F.~Poupon, \underline{A.~Roche}, J.-F. Mangin, and C.~Poupon.
\newblock Real-time {MR} diffusion tensor and {Q-ball} imaging using {Kalman}
  filtering.
\newblock In {\em Annual Meeting of the International Society for Magnetic
  Resonance in Medicine (ISMRM'08)}, Toronto, Canada, 2008.

\bibitem{ohbm:06}
S.~M\'eriaux, \underline{A.~Roche}, G.~Dehaene-Lambertz, and J.-B. Poline.
\newblock When do mixed-effect models fail to improve detection sensitivity in
  f{MRI} group activation maps?
\newblock In {\em Annual Meeting of the Organization for Human Brain Mapping
  (HBM'06) CD-Rom Neuroimage 31(1)}, Florence, Italy, 2006.

\bibitem{ohbm:05b}
S.~M\'eriaux, \underline{A.~Roche}, J.~Catry, O.~Chauvin, and J.-B. Poline.
\newblock Robust statistics for f{MRI} group analyses.
\newblock In {\em Annual Meeting of the Organization for Human Brain Mapping
  (HBM'05) CD-Rom Neuroimage 26(1)}, Toronto, Canada, June 2005.

\bibitem{ohbm:04c}
E.~Duchesnay, \underline{A.~Roche}, D.~Rivi\`ere, D.~Papadopoulos-Orfanos, Y.~Cointepas,
  and J.-F. Mangin.
\newblock Guessing the sex from the shapes of cortical folds.
\newblock In {\em Annual Meeting of the Organization for Human Brain Mapping
  (OHBM'04) CD-Rom Neuroimage 22(1)}, Budapest, Hungary, June 2004.

\bibitem{ohbm:04d}
G.~Flandin, X.~Pennec, \underline{A.~Roche}, W.~Penny, N.~Ayache, and J.-B. Poline.
\newblock Multi-subject anatomo-functional classification for activation
  studies.
\newblock In {\em Annual Meeting of the Organization for Human Brain Mapping
  (HBM'04) CD-Rom Neuroimage 22(1)}, Budapest, Hungary, June 2004.

\bibitem{ohbm:04b}
S.~M\'eriaux, F.~Kherif, \underline{A.~Roche}, M.~Brett, L.~Garnero, and J.-B. Poline.
\newblock How frequently do we sample inhomogeneous group of subjects in f{MRI}
  studies?
\newblock In {\em Annual Meeting of the Organization for Human Brain Mapping
  (HBM'04) CD-Rom Neuroimage 22(1)}, Budapest, Hungary, June 2004.

\bibitem{cancero:99}
P.~Y. Bondiau, G.~Malandain, K.~Benezery, \underline{A.~Roche}, R.~Ferrand, D.~Ponvert,
  J.~L. Habrand, and J.~N. Bruneton.
\newblock {Fusions Scanner / IRM entri\`erement automatique: perspectives
  d'utilisation pour la radioth\'erapie}.
\newblock In {\em {Proc. du XIXe Forum de Canc\'erologie}}, Paris, June 1999.

\bibitem{rsna:99}
P.~Y. Bondiau, G.~Malandain, K.~Benezery, \underline{A.~Roche}, R.~Ferrand, D.~Ponvert,
  J.~L. Habrand, and J.~N. Bruneton.
\newblock {Automatic CT Scan and MRI Matching: Use for Brain Tumor
  Radiotherapy}.
\newblock In {\em {RSNA'99}}, Chicago, USA, November 1999. Published in
  Supplement to Radiology, 213:236.

\bibitem{dijon:99}
S.~Ourselin, C.~Sattonnet, \underline{A.~Roche}, and G.~Subsol.
\newblock {Automatic alignment of histological sections for 3D reconstruction
  and analysis}.
\newblock In {\em 6\`eme congr\`es annuel de l'Association Fran\c caise de
  Cytom\'etrie}, Dijon, France, October 1999. Published in the Analytical
  Cellular Pathology Journal, 19(3):123.


\subsection*{Preprints}

\bibitem{rr:15}
\underline{A.~Roche}.
\newblock {Composite Bayesian inference}.
\newblock Technical report, 2015.
\newblock Available at arxiv.org (arXiv:1512.07678).

\bibitem{rr:12}
\underline{A.~Roche}.
\newblock {EM algorithm and variants: an informal tutorial}.
\newblock Technical report, 2012.
\newblock Available at arxiv.org (arXiv:1105.1476).

\bibitem{rr:00}
\underline{A.~Roche}, X.~Pennec, G.~Malandain, N.~Ayache, and S.~Ourselin.
\newblock {Generalized Correlation Ratio for Rigid Registration of 3D
  Ultrasound with MR Images}.
\newblock Technical Report 3980, INRIA, July 2000.

\bibitem{rr:99}
\underline{A.~Roche}, G.~Malandain, and N.~Ayache.
\newblock {Unifying Maximum Likelihood Approaches in Medical Image
  Registration}.
\newblock Technical Report 3741, INRIA, July 1999.

\bibitem{rr:98}
\underline{A.~Roche}, G.~Malandain, N.~Ayache, and X.~Pennec.
\newblock {Multimodal Image Registration by Maximization of the Correlation
  Ratio}.
\newblock Technical Report 3378, INRIA, August 1998.

\bibitem{rr:01}
S.~Granger, X.~Pennec, and \underline{A.~Roche}.
\newblock {Rigid Point-Surface Registration using Oriented Points and an EM
  Variant of ICP for Computer Guided Oral Implantology}.
\newblock Technical Report 4169, INRIA, April 2001.

\bibitem{rr:99b}
A.~Guimond, \underline{A.~Roche}, N.~Ayache, and J.~Meunier.
\newblock {Multimodal Brain Warping Using the Demons Algorithm and Adaptative
  Intensity Corrections}.
\newblock Technical Report 3796, INRIA, November 1999.

\bibitem{rr:98b}
S.~Ourselin, \underline{A.~Roche}, G.~Subsol, X.~Pennec, and C.~Sattonet.
\newblock {Automatic Alignment of Histological Sections for 3D Reconstruction
  and Analysis}.
\newblock Technical Report 3595, INRIA, December 1998.


\subsection*{Patents}

\bibitem{patentapp:12}
\underline{A.~Roche}, T.~Kober, and G.~Krueger.
\newblock Automated atlas-free brain tissue classification using specific
  magnetic resonance imaging sequences, 2013.
\newblock Owner: Siemens Schweiz AG, Switzerland. US patent application
  13/855972.

\bibitem{patent:05}
\underline{A.~Roche}, M.~Brady, J.~Declerck, and O.~Corolleur.
\newblock Similarity measures.
\newblock United States patent US2005152617, July 2005.
\newblock Owner: Mirada Solutions Ltd, UK. Also published as EP1513104
  (Europe), JP2005169077 (Japan).

\bibitem{patent:03}
\underline{A.~Roche}, N.~Ayache, G.~Malandain, and X.~Pennec.
\newblock Electronic device for automatic registration of images.
\newblock United States patent US6539127, March 2003.
\newblock Owner: INRIA, France. Also published as EP0977148 (Europe), FR2781906
  (France).

\bibitem{patent:15}
K.~O'Brien and \underline{A.~Roche}.
\newblock {A simple method to denoise MP2RAGE}.
\newblock United States patent US2013P01055, 2013.
\newblock Owner: Siemens Schweiz AG, Switzerland. Also published as EP 2765437.

\bibitem{patent:12}
C.~Poupon, F.~Poupon, \underline{A.~Roche}, and J.-F. Mangin.
\newblock Real-time magnetic resonance diffusion imaging.
\newblock United States patent US8330462, 2007.
\newblock Owner: CEA, France. Also published as EP2203758A1 (Europe).


\subsection*{Theses}

\bibitem{these:01}
\underline{A.~Roche}.
\newblock {\em Recalage d'images m\'edicales par inf\'erence statistique}.
\newblock PhD thesis, Universit\'e de Nice-Sophia Antipolis, February 2001.

\bibitem{dea:96}
\underline{A.~Roche}.
\newblock {Mod\'elisation de la perception du tempo par oscillateurs
  adaptatifs}.
\newblock Master's thesis, {Universit\'e Pierre et Marie Curie, Paris VI},
  1996.

\end{thebibliography}


\end{document}
