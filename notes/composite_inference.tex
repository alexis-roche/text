\documentclass{article}
\usepackage{fullpage}
\usepackage{amsmath,amssymb}
\title{Inf\'erence composite bay\'esienne (ou pas)}
\author{Alexis Roche}

\begin{document}

\maketitle

\section{Introduction}

Le concept de mod\`ele composite... Il faut le voir comme une m\'ethode d'inf\'erence qui proc\`ede par aggr\'egation d'inf\'erences binaires. Chaque inf\'erence binaire est r\'ealis\'ee par un "expert" utilisant {\em sa propre repr\'esentation} et {\em ind\'ependant} au sens o\`u ses r\'esultats ne d\'ependent pas des autres experts. 

Chaque expert "vit" dans un monde des possibles r\'eduit \`a deux hypoth\`eses, dont l'une est commune \`a tous les experts. Plus concr\`etement, sa mission est de calculer un rapport de probabilit\'es entre l'hypoth\`ese dont il est charg\'e et l'hypoth\`ese de r\'ef\'erence commune \`a tous les experts. C'est ensuite au "coordinateur" de construire une fonction de probabilit\'e multi-classes en collectant les r\'eponses des experts. 

La base de donn\'ees d'apprentissage, si elle existe, est centralis\'ee. Tous les experts y ont acc\`es, mais formattent les donn\'ees \`a leur fa\c{c}on, c-\`a-d selon leur repr\'esentation. Chaque expert effectue son apprentissage de fa\c{c}on classique, dans son monde propre, avec sa repr\'esentation propre voire son propre sous-ensemble de donn\'ees. 

Dans le cas de l'apprentissage "\`a la vol\'ee" (qui n'a de sens que dans le cas o\`u chaque expert utilise un mod\`ele g\'en\'eratif), c'est la m\^eme id\'ee de base: chaque expert \'elimine les param\`etres de son mod\`ele selon sa propre repr\'esentation et \'evalue son propre rapport de probabilit\'es que le coordinateur utilise exactement comme dans le cas de l'apprentissage s\'epar\'e. 

En d'autres termes, le coordinateur ne sait pas comment les param\`etres du mod\`ele sont \'elimin\'es. Il en d\'el\`egue la responsabilit\'e aux experts. Sa responsabilit\'e \`a lui est de combiner les rapports de probabilit\'es fournis par les experts. 


\section{Lien avec PDF projection}

Expliquons d\'ej\`a en quoi ce paradigme certes intuitif diff\`ere du paradigme classique pour l'inf\'erence multi-classes. Supposons qu'on ait un mod\`ele g\'en\'eratif de la forme: 
$$
p(y|x) = a(y) b(z_x, x),
$$ 
o\`u $zx = z(x, y)$ est un attribut ``contextuel'' ou ``adaptatif'' (notion plus g\'en\'erale que celle de statistique), on prouve facilement que cette forme se ram\`ene \`a celle du PDF projection theorem: 
\begin{equation}
\label{eq:pdf_proj}
p(y|x) = \pi(y) \frac{p(z_x|x)}{\pi(z_x)}
\end{equation}

Une telle structure peut apparaitre naturellement dans un mod\`ele ou r\'esulter de l'application du principe d'entropie relative maximale si on s\'electionne le mod\`ele en supposant connues les distributions de l'attribut adaptatif, $p(z_x|x)$. Notons que dans ce dernier cas, la distribution de r\'ef\'erence $\pi(y)$ est arbitraire.

Cette strat\'egie d'entropie maximale d\'ecoule en fait d'un probl\`eme de th\'eorie des jeux d\'ecrit par Gr\"unwald, 2005: il s'agit de simuler des donn\'ees en l'absence d'un mod\`ele g\'en\'eratif complet. Le PDF projection est donc une solution optimale \`a un probl\`eme de synth\`ese de donn\'ees.

Mais est-ce qu'on peut invoquer le PDF projection pour faire de l'inf\'erence? Dans son article de 2005 (je crois), Baggenstoss observe que le PDF projection est exact si la repr\'esentation est exhaustive. Autrement dit, on peut se passer du principe de maximum d'entropie si on suppose l'exhaustivit\'e de la repr\'esentation. C'est une justification alternative mais plus forte dans le contexte de l'inf\'erence.


\section{Maximum de vraisemblance compl\`ete: exemple}

Supposons qu'on ait la forme (\ref{eq:pdf_proj}). Si on traite $p(y|x)$ de fa\c{c}on classique, sans tenir compte de cette structure, on voit que l'estimation des param\`etres du mod\`ele par maximum de vraisemblance (par exemple) conduit \`a: 
$$
\max_{\alpha, \beta} \left[
\sum_k \log \frac{\pi_\alpha(y_k)}{\pi_\alpha(z_{x_k})} + \sum_k \log p_\beta(z_{x_k}|x_k)
\right]
$$ 

En supposant que les param\`etres $\alpha$ de $\pi$ et $\beta$ des autres lois g\'en\'eratives sont disjoints et que~$\pi$ est la loi de la classe~0, cela revient \`a maximiser chaque terme ind\'ependamment, donc notamment: 
$$
\max_\alpha \left[
\sum_k f_k(\alpha)
\right]
$$ 
avec: $ f_k(\alpha) = \log \pi_\alpha(y_k)$  si $x_k = 0$ et $f_k(\alpha) = \log \pi_\alpha(y_k)/\pi_\alpha(z_{x_k})$ si $x_k > 0$.

Supposons par exemple qu'on ait 3 classes et que $y = (z_1, z_2)$ sont ind\'ependants dans la classe~0, {\em i.e.} $\pi(z_1, z_2) = \pi(z_1)p(z_2)$. On a alors: $f_k(\alpha) = \log \pi_\alpha(z_{1k})\pi_\alpha(z_{2k})$ si $x_k=0$, $f_k(\alpha) = \log \pi_\alpha(z_{2k})$ si $x_k=1$,  et $f_k(\alpha) = \log \pi_\alpha(z_{1k})$ si $x_k=2$.

Si on suppose en outre que $\pi(z_1)$ et $\pi(z_2)$ correspondent \`a des sous-ensembles disjoints de alpha, on voit que la m\'ethode classique consiste \`a ajuster la distribution ``nulle'' de $z_1$ (l'attribut ``exhaustif'' pour la classe~1) sur les donn\'ees issues des classes~0 et~2... 

Que vient faire la classe~2 dans cette histoire??? N'est-il pas absurde d'utiliser des donn\'ees de la classe~2 pour ajuster la loi de la classe~0? 

Non car on peut se rendre compte que le mod\`ele implique que $z_1$ est distribu\'e de la m\^eme fa\c{c}on dans les classes~0 et~2: 
$$
p_2(z_1, z_2) = \pi(z_1) p_2(z_2) 
\Rightarrow 
p_2(z_1) = \pi(z_1)
$$ 

Il est judicieux, dans ce cas, de ne pas se restreindre aux donn\'ees de la classe 0 pour ajuster sa distribution. Sans surprise, la vraisemblance donne lieu \`a une estimation statistiquement efficace \`a d\'efaut d'\^etre calculatoirement simple. Bien s\^ur, si le mod\`ele n'est pas r\'ealiste (en d'autres termes, si l'attribut contextuel n'est pas exhaustif pour chaque classe), alors ce n'est pas une super id\'ee de m\'elanger les donn\'ees, mais ce n'est peut-\^etre pas non plus une super id\'ee d'utiliser l'approche composite... 


\section{Limitations de la vraisemblance compl\`ete}

Est-ce \`a dire que le maximum de vraisemblance compl\`ete est toujours la meilleure approche? On a d\'ej\`a compris que son application emp\^eche les experts d'\^etre ind\'ependants. En effet, l'ajustement de la loi nulle doit \^etre centralis\'e car il fait intervenir des informations que les experts n'ont pas. Dans l'exemple ci-dessus, l'expert pour la classe~1 ne peut ajuster $\pi(z_1)$ lui-m\^eme car il a besoin pour \c{c}a des valeurs de $z_2$. 

Pour avoir un mod\`ele complet, on doit avoir une repr\'esentation {\em originelle} des donn\'ees, c-\`a-d qu'on doit d\'efinir la variable $y$ alors que celle-ci n'a n'a pas lieu d'\^etre dans l'approche composite. L'ajustement des lois $\pi(z_x)$ via la vraisemblance compl\`ete est-il ind\'ependant de la repr\'esentation originelle? La r\'eponse est non. Si on inclut dans y un attribut z3 qui a, par exemple, la m\^eme variance que $z_1$ sous l'hypoth\`ese de r\'ef\'erence alors, clairement, l'estimation par maximum de vraisemblance compl\`ete de $\pi(z_1)$ va d\'ependre de $z_3$... c-\`a-d qu'on devra mesurer $z_3$ pour pouvoir ajuster cette loi. 

Ce probl\`eme n'est pas sp\'ecifique \`a notre sujet d'\'etude. On sait bien que toute inf\'erence statistique d\'epend du choix d'une repr\'esentation. Si les maximums de vraisemblance marginale et jointe donnaient les m\^emes r\'esultats, \c{c}a se saurait. Et il est bien naturel que ce ne soit pas le cas car l'int\'er\^et de choisir une repr\'esentation plus riche est justement d'incorporer plus d'information - encore faut-il acqu\'erir les donn\'ees qui vont avec. 

Imaginons maintenant que la repr\'esentation choisie soit de grande dimension et implique des tas de param\`etres additionnels... Est-ce que \c{c}a peut conduire \`a un mauvais ajustement des lois $\pi(z_x)$ \`a cause du curse of dimensionality? Je ne pense pas. 

Reste une difficult\'e conceptuelle avec la vraisemblance compl\`ete: le mod\`ele stipule que tel attribut est suffisant pour tester telle hypoth\`ese. Mais l'estimation des param\`etres remet en cause ce principe. 


\section{Vraisemblance r\'eduite}

Mes r\'eflexions de la fin de l'\'et\'e 2017 m'avaient conduit \`a r\'ealiser que l'approche composite \'equivaut \`a une m\'ethode d'\'elimination de param\`etres alternative applicable \`a tout mod\`ele g\'en\'eratif complet d\`es lors qu'il admet une repr\'esentation adaptive r\'eduite (ce qui est toujours le cas, en fait!). Appelons-la vraisemblance r\'eduite. Il y a quelques similitudes avec le concept de vraisemblance restreinte mais ce n'est pas tout-\`a-fait la m\^eme chose. 

La vraisemblance r\'eduite garantit l'exhaustivit\'e {\em en un sens fort}: en clair, elle laisse chaque expert \'eliminer les param\`etres par ses propres moyens. Contrairement \`a la vraisemblance compl\`ete, elle ne fournit pas un estimateur unique des param\`etres de la distribution de r\'ef\'erence (chaque expert y va de son estimateur...). Si les param\`etres des distributions de classe sont li\'es, c'est encore plus le bordel. 

Mais quelle importance si le but assum\'e est d'estimer les rapports de vraisemblance: 
$$
R(x) 
= \frac{p(y|x)}{\pi(y)}
= \frac{p(z_x|x)}{\pi(z_x)}
$$ 

La vraisemblance r\'eduite permet de le faire sans exploiter toute l'information contenue dans les donn\'ees mais avec plusieurs avantages: 
\begin{itemize}
\item exhaustivit\'e forte 
\item coh\'erence asymptotique (et oui, c'est maintenu car la vraisemblance   restreinte reste la vraisemblance...) 
\item ind\'ependance vis-\`a-vis de toute repr\'esentation ``universelle'' 
\item flexibilit\'e: ind\'epedendance des experts et donc possibililt\'e de   combiner leurs r\'esultats sans connaitre les donn\'ees utilis\'ees (et   donc notamment s'ils ont utilis\'e des donn\'ees diff\'erentes) 
\item simplicit\'e calculatoire: on n'a m\^eme pas besoin d'un mod\`ele complet,   il suffit de mod\'eliser les attributs sous chaque hypoth\`ese d'int\'er\^et 
\end{itemize}


\section{Le cas du recalage}

En recalage, l'exhaustivit\'e forte signifie que la mesure de similarit\'e ne d\'epend que de ce qui se passe dans la zone de recouvrement entre les images. Propri\'et\'e on ne peut plus naturelle, mais qu'on n'obtient pas avec la vraisemblance compl\`ete. 

Je suis d\'esormais convaincu de la validit\'e du mod\`ele g\'en\'eratif i.i.d. pr\'esent\'e dans ma th\`ese pour le recalage: il est justifi\'e par le th\'eor\`eme d'\'echangeabilit\'e de de Finetti dans la mesure o\`u on suppose que les pixels/voxels sont m\'elang\'es au pr\'ealable, ce qui revient \`a dire que la repr\'esentation ``universelle'' est un ensemble de pixels (donc sans notion d'ordre). 

On peut soit appliquer le th\'eor\`eme dans la r\'egion de recouvrement des images et obtenir ainsi un mod\`ele incomplet, ou l'appliquer globalement en introduisant des distributions pour les points non-appari\'es et obtenir un mod\`ele qui a l'apparence d'une PDF projection sans en \^etre une. Dans le premier cas, on ne peut pas utiliser la vraisemblance compl\`ete car elle n'existe pas mais on peut utiliser la vraisemblance r\'eduite! 

Cela dit, autant avoir un mod\`ele complet. La vraisemblance compl\`ete est alors applicable et conduit \`a estimer les distributions des points non-appari\'es pr\'ecis\'ement sur les points non-appari\'es. Mais des propri\'et\'es bizarres apparaissent: 
\begin{itemize}
\item La mesure de similarit\'e d\'epend de pixels ext\'erieurs \`a la zone de   recouvrement... 
\item Le nombre de points cibles non-appari\'es est potentiellement infini si l'image cible doit \^etre interpol\'ee. La distribution associ\'ee est alors hyper stable vis-\`a-vis de la transformation mais a aussi un poids infini dans la mesure de recalage... 
\end{itemize}

L'intuition dit qu'il est bien plus naturel d'utiliser la vraisemblance r\'eduite. Tellement plus naturel que c'est une \'evidence. Et c'est ce qui me g\^ene encore un peu dans cette histoire. C'est \'evident mais je n'ai pas d'argument th\'eorique pour rejeter la vraisemblance compl\`ete. 


\section{Elimination de param\`etres}

On peut voir l'\'elimination de param\`etres comme un probl\`eme d'approximation ou, si on pr\'ef\`ere, de {\em d\'ecision}, c-\`a-d un probl\`eme de choix d'une valeur unique dans un ensemble. Il s'agit ici de repr\'esenter un mod\`ele param\'etrique par une distribution unique afin d'\'evaluer {\em l'\'evidence du mod\`ele}, c-\`a-d la probabilit\'e qu'il conf\`ere aux donn\'ees. 

Notons la diff\'erence de perspective avec l'estimation ponctuelle, qui cherche directement \`a estimer la variable d'int\'er\^et, par exemple, dans un cadre bay\'esien, par son mode ou sa moyenne a posteriori. Dans ce cas, les param\`etres de nuisance sont \'elimin\'es par marginalisation car ils n'apparaissent pas dans la fonction de perte associ\'ee au probl\`eme de d\'ecision. 

Ici, l'objectif est en soi d'estimer l'\'evidence de mani\`ere \`a obtenir une {\em fonction} DE VRAISEMBLANCE de la variable d'int\'er\^et x. On ne se contente pas d'un estimateur ponctuel de x. Celui-ci sera obtenu comme sous-produit de l'\'elimination de param\`etres, mais d'autres sous-produits seront disponibles, par exemple l'incertidude sur x. 

Pour chaque $x$, on repr\'esente donc la famille de distributions sur $y$, $p_\theta(y|x)$, par une distribution unique index\'ee par $x$ qui d\'efinit donc une loi conditionnelle $q(y|x)$ libre de tout param\`etre. Comme on travaille \`a $x$ fix\'e, on peut omettre $x$ dans les notations suivantes par souci de simplicit\'e. Comment choisir cette loi?


\subsection{Elimination bay\'esienne}

Une id\'ee naturelle est de choisir une sorte de centre de la famille, c'est-\`a-dire un \'el\'ement situ\'e \`a faible ``distance'' de tout autre \'el\'ement. La ``distance'' en question d\'efinit la fonction de perte et peut \^etre prise, par exemple, au hasard, comme la divergence KL: 
$$L(\theta, q) = D(p_\theta\|q)$$

Dans une perspective bay\'esienne, on minimise le risque int\'egr\'e par rapport \`a la distribution jointe des donn\'ees et des param\`etres, ce qui suppose la d\'efinition d'un a priori pi(theta). On note que la d\'ecision q est ici ind\'ependante des donn\'ees par contrainte du probl\`eme. Ceci donne: 
$$R(q) = \int \pi(\theta) p_\theta(y) \log [p_\theta(y) / q(y)] dy d\theta$$ 

$$\Rightarrow \arg\min R(q) = \int \pi(\theta) p_\theta(y) d\theta$$ 

On tombe nez \`a nez avec la marginalisation. Le bon vieux barycentre, quoi... (qui n'appartient pas forc\'ement \`a la famille). Pas besoin d'une grande th\'eorie pour \c{c}a. 

On peut cependant obtenir d'autres strat\'egies tout aussi bay\'esiennes en choisissant d'autres divergences; par exemple, la divergence KL invers\'ee: $L(\theta, q) = D(q\|p_\theta)$,
\begin{eqnarray*}
R(q) 
& = & \int pi(\theta)q(y) \log \frac{q(y)}{p_\theta(y)} dy d\theta \\
& = & \int q(y) \log q(y) dy  - \int \pi(\theta) q(y) \log p_\theta(y) dy d\theta
\end{eqnarray*}
ce qui donne:
$$ 
\frac{\partial R}{\partial q}
= \log q(y) + 1 - \int \pi(\theta) \log p_\theta(y) dy 
\quad \Rightarrow 
\arg\min R(q) = K \exp\left[\int \pi(\theta) \log p_\theta(y) dy \right]
$$ 

La solution est cette fois la moyenne g\'eom\'etrique. Tout \c{c}a pour dire que la marginalisation n'est {\em pas n\'ecessairement} la m\'ethode bay\'esienne de choix pour l'\'elimination des param\`etres de nuisance. Tout d\'epend du probl\`eme de d\'ecision et de la fonction de perte associ\'ee. 


\subsection{NML}

Il existe une alternative non-bay\'esienne \`a l'\'elimination de param\`etres issue de la communaut\'e minimum description length (MDL): la normalized maximum likelihood distribution (NML). Sa justification classique (cf. Gr\"unwald ou Rissanen) repose sur la notion de {\em regret}, c-\`a-d la diff\'erence de longueur de code entre la distribution caract\'eristique candidate et la distribution qui compresse le plus l'observation: 
$$
L(y, q) = - \log q(y) - \min_\theta [-\log p_\theta(y)] = -\log q(y) + \log \ell_\star(y)
$$ 
avec: $\ell_\star(y) = \max_\theta p_\theta(y)$, la {\em vraisemblance profile} vue comme une fonction de~$y$.

Le maximum de vraisemblance apparait naturellement dans cette formulation en tant que ``meilleur mod\`ele'' mais c'est un mod\`ele na\"if car $\ell_\star(y)$ ne d\'efinit pas une distribution pr\'edictive valide \`a cause du couplage entre l'observation et l'estim\'ee MV. 

Une diff\'erence essentielle avec l'approche pr\'ec\'edente est que la fonction de perte d\'epend de l'observation. Tout se passe en fait comme si l'observation rempla\c{c}ait theta comme ``\'etat inconnu du monde''. Cela a du sens dans la mesure o\`u la distribution caract\'eristique doit \^etre choisie ind\'ependamment de toute observation. En fait, on peut consid\'erer que l'\'etat du monde est donn\'e par le couple $(y, \theta)$, ce qui permet de d\'efinir le regret pour tout $\theta$: 
$$
L(y, \theta; q) = \log \frac{p_\theta(y)}{q(y)}
$$ 

D\`es lors, il est naturel de chercher \`a minimiser le {\em pire regret}: 
$$
R(q) = \max_{y, \theta} L(y, \theta; q)
$$ 
en notant au passage qu'on ne fait pas l'hypoth\`ese que $y$ est distribu\'e selon un certain $p_\theta(y)$. Cette approche \'evite de choisir des a priori sur $y$ et $\theta$. Comme c'est classique dans ce type de probl\`emes minimax, on probabilise $y$ en introduisant le risque moyen:
$$
L(p, \theta; q) 
= E_p [L(y, \theta; q)] 
= \int p(y) \log \frac{p_\theta(y)}{q(y)} dy 
= D(p\|q) - D(p\|p_\theta) 
$$ 

On a \'evidemment: 
$$R(q) = \max_p \max_\theta L(p, \theta, q) = \max_p L(p, q)$$ 
o\`u: 
$$L(p, q) = \int p(y) \log \frac{\ell_\star(y)}{q(y)} dy$$ 

Il suffit alors de montrer que les op\'erateurs min et max commutent:  
$$\min_q R(q) = \min_q \max_p L(p, q) = \max_p \min_q L(p, q)$$ 

On voit alors que quelque soit $p$, la distribution de Shtarkov (1987) dite NML minimise {\em toujours} $L(p, q)$: 
$$\arg\min_q L(p, q) = \frac{\ell_\star(y)}{Z_{\rm NML}}$$ 
avec en plus la propri\'et\'e que le minimum atteint est ind\'ependant de p: 
$$\min_q L(p, q) = \log Z_{\rm NML}$$

Ainsi, la distribution NML est solution de notre probl\`eme! 

Au-del\`a de la nature {\em Bayes robuste} de cette formulation, on voit qu'elle consiste plus fondamentalement \`a remplacer $D(p_\theta\|q)$ dans la 1\`ere formulation par $D(p\|q) - D(p\|p_\theta)$ avec l'id\'ee que $(p,\theta)$ repr\'esente l'\'etat du monde. 


\subsection{Crit\`ere d'Akaike}

Une autre approche bien connue est celle d'Akaike, 1974.

L'id\'ee de base est de d\'efinir l'ad\'equation d'un mod\`ele via la divergence KL ou, de fa\c{c}on \'equivalente, la log-vraisemblance moyenne: 
$$S(p, p_\theta) = \int p(y) \log p_\theta(y) dy$$ 
plut\^ot que via la notion empirique d'\'evidence. Dans cette perspective, la distribution r\'ealisant le maximum de~$S$ sur la famille apparait comme distribution caract\'eristique naturelle et le maximum en question donne une mesure {\em optimiste} qu'il est naturel de consid\'erer comme log-vraisemblance du mod\`ele: 
$$S_\star = \max_\theta S(p, p_\theta)$$ 

Tout semble parfait jusqu'au moment o\`u on r\'ealise qu'on ne ne connait pas $p$, la ``vraie'' distribution des donn\'ees, et que $S_\star$ est donc inconnu. Mais Akaike justifie l'approximation: 
$$S_\star \approx \log \ell_\star(y) - k$$ 
o\`u $k$ est la dimension de $\theta$. Tr\`es franchement, j'ai du mal \`a suivre son raisonnement... 
J'ai l'impression qu'il suffit d'appliquer le fameux th\'eor\`eme de Wilks: 
$$2 \log \frac{\ell_\star(y)}{p(y)} \sim \chi_2(k)$$ 
valable asymptotiquement {\em si $p(y)$ appartient \`a la famille} et sous les conditions usuelles d'\'echantillonnage simple. Wilks nous dit alors que la log-vraisemblance profile sur-estime le score du mod\`ele par la moiti\'e d'un $\chi_2$ \`a $k$ degr\'es de libert\'e. En prenant l'esp\'erance de l'expression ci-dessus, les choses deviennent tr\`es claires: 
$$S_\star = E[\log \ell_\star(y)] - \frac{k}{2}$$ 

Ainsi, la quantit\'e: 
$$\log \ell_\star(y) - k/2$$ 
fournit un estimateur sans biais de $S_\star$. C'est presque la correction d'Akaike... \`a un facteur 2 pr\`es. Akaike se serait-il plant\'e? 

Il semble que cette incoh\'erence vienne du fait qu'Akaike cherche en fait \`a estimer: 
$$S_{\star\star} = \int p(y) \log \ell_\star(y) dy$$ 
et non $S_\star = \max_\theta S(p, p_\theta)$, ce qui me semble plus naturel mais c'est moi. 

Quoiqu'il en soit, tout se passe comme si on d\'efinissait l'\'evidence du mod\`ele comme: $q(y) = \ell_\star(y)/Z_{\rm AIC}$ avec $Z_{\rm AIC} = \exp(k/2)$ ou $\exp(k)$ mais c'est justifi\'e sous l'hypoth\`ese {\em absurde dans le contexte de la s\'election de mod\`ele} que $p(y)$ est dans la famille. Si tous les mod\`eles test\'es sont corrects, il suffit de prendre le plus parcimonieux! 

Peut-on g\'en\'eraliser Wilks au cas d'un mod\`ele mal sp\'ecifi\'e? Pour cela, on peut se reporter au papier de Harts, 1982 et, en recollant les morceaux, voir que: 
$$
E \big[2 \log \frac{\ell_\star(y)}{p_\star(y)} \big] = {\rm trace}(B A^{-1})
$$ 
o\`u $p_\star$ est la projection KL de p sur la famille, avec:
$$
A = -E\left[ \frac{\partial^2 \log p_\theta}{\partial\theta^2} \right],
\quad 
B = E\left[ \frac{\partial \log p_\theta}{\partial\theta} \frac{\partial \log p_\theta}{\partial\theta}^\top \right]
$$ 

Dans le cas o\`u $p=p_\star$ appartient \`a la famille, ces deux matrices (calcul\'ees en $\theta_\star$) sont \'egales (\`a l'information de Fisher), d'o\`u: 
$$
{\rm trace}(B A^{-1}) = {\rm trace}({\bf I}_k) = k
$$ 

On retrouve bien le r\'esultat de Wilks. Mais on voit aussi qu'il ne se g\'en\'eralise pas trivialement. Le biais d\'epend de la ``vraie'' distribution des donn\'ees; et s'il est grand, on pourrait se faire berner par un mod\`ele incorrect. On peut l'estimer empiriquement, ce biais, si \c{c}a rentre dans le budget calculatoire, ou essayer de le majorer... 

On peut voir que $B = A + G$ avec:
$$
G = E\left[\frac{\partial^2 p_\theta}{\partial\theta^2}\big / p_\theta\right]
$$ 

Donc:
$$B A^{-1} = {\bf I}_k + G A^{-1}
\Rightarrow
{\rm trace}(A B^{-1}) = k + {\rm trace}(G A^{-1})
$$ 

Facile de montrer que $G=0$ si $p=p_\star$ mais sinon... ????? 

Cela \'etant, le r\'esultat montre que le biais sur la log-vraisemblance {\em moyenne} est en $O(1/n)$, o\`u $n$ est le nombre de points. Ainsi, lorsqu'on compare des mod\`eles sur un ``grand'' \'echantillon, on peut raisonnablement supposer que le biais est n\'egligeable. 

En conclusion, l'int\'er\^et majeur de l'approche d'Akaike est, \`a mon sens, de justifier la vraisemblance profile comme mesure d'\'evidence \`a la limite des grands \'echantillons. Sous r\'eserve que $n$ ne varie pas en fonction des mod\`eles (ce qui se produit en recalage).



\section{Le paradoxe recalage/appariements}

J'ai but\'e pendant des ann\'ees sur une difficult\'e conceptuelle de ma formulation statistique du recalage: par simple raffinement de l'\'echantillonnage, on peut rendre une transformation infiniment plus vraisemblable qu'une autre et donc rendre l'incertitude bay\'esienne sur le recalage aussi petite qu'on veut. 

Ce ph\'enom\`ene vient du fait que la log-vraisemblance d'une transformation est proportionnelle au nombre de points appari\'es. Or ce nombre est d\'etermin\'e par l'\'echantillonnage et est donc arbitraire. Certes, on s'attend \`a ce que la vraisemblance d\'epende de la repr\'esentation des donn\'ees. Mais \`a ce point? Cette observation m'a fait longtemps douter de l'hypoth\`ese d'ind\'ependance des pixels. 

Elle semble en tous cas contradictoire avec un principe \'el\'ementaire de th\'eorie de l'information. Souviens-toi de l'in\'egalit\'e fondamentale du traitement des donn\'ees. Notre recalage statistique repose sur un pr\'e-traitement des donn\'ees pour le moins radical qui consiste \`a r\'eduire chaque image \`a un histogramme (ou, de fa\c{c}on \'equivalente, une liste des valeurs d'intensit\'e dans un ordre al\'eatoire et ignorant la position des points).

En notant $X$ les param\`etres de transformation, $Y=(I, J)$ le couple d'images et $Z=(f_I, f_J)$ sa repr\'esentation du pauvre, on a une belle cha\^ine de Markov: $X \to Y \to Z$.

Le fait que $X$ soit {\em aussi pr\'ecis que l'on veut} connaissant $Z$ signifie que $I(X,Z)$ est {\em aussi grand que l'on veut}. Or la th\'eorie de l'information affirme que $I(X,Z) \leq I(X,Y)$ donc $I(X,Y)$ est infinie. Toute proc\'edure bay\'esienne de recalage aurait donc n\'ecessairement une incertitude nulle (sauf \'eventuellement \`a reposer sur une repr\'esentation encore plus pauvre que la n\^otre). N'est-ce pas la preuve irr\'efutable qu'il y a une faille dans notre raisonnement? Comment l'algorithme pourrait-il \^etre certain de ne pas se tromper alors que chacun sait qu'il y a toujours des erreurs de recalage?

J'avais not\'e en 2007 que le paradoxe ``disparait'' si on s'int\'eresse \`a des transformations libres. Les param\`etres recherch\'es sont alors des appariements et sont d\'ependants de l'\'echantillonnage. Le fait que la vraisemblance augmente par raffinement de l'\'echantillonnage est alors un simple effet de nombre car les distributions marginales des appariements sont inchang\'ees. On ne peut \^etre certain des appariements \`a quelque r\'esolution que ce soit. 

Contrairement \`a l'intuition qui a parfois fauss\'e mon jugement, l'ind\'ependance a posteriori des appariements n'est pas le signe d'un ``exc\`es de confiance'' de l'algorithme. Bien au contraire! L'ind\'ependance a posteriori, en plus du fait que les lois marginales a posteriori sont ind\'ependantes du contexte (c-\`a-d, n'utilisent pas les pixels voisins du point cible orrespondant) illustre bien l'information limit\'ee fournie par la repr\'esentation. La distribution a posteriori pourrait dfficilement \^etre plus {\em entropique}!

Ouf... l'in\'egalit\'e fondamentale du traitement des donn\'ees n'est pas bafou\'ee, au contraire elle est confirm\'ee dans toute sa gloire.  Mais, alors, si l'incertitude sur des transformations libres est ``grande'', comment peut-elle \^etre nulle pour des transformations param\'etriques? 

Le probl\`eme n'a rien \`a voir avec notre mod\`ele particulier de repr\'esentation des images. Le simple fait d'imaginer qu'on puisse d\'efinir une vraisemblance sur des appariements denses conduit au ``paradoxe''. 

Si la vraisemblance en question est not\'ee L(Q)=p(Y|Q), que vaut p(Y|X), la vraisemblance sur les param\`etres de transformation? On suppose une chaine de Markov: $X \to Q \to D$, autrement dit, que les donn\'ees sont ind\'ependantes de la transformation CONNAISSANT les appariements, ce qui semble \'evident. On a donc: 
$$
p(Y|X) = \int p(Y|Q)p(Q|X) dQ
$$

En supposant (pour fixer les id\'ees) une d\'ependance d\'eterministe $X \mapsto Q$, on a donc, tout simplement: 
$$
L(X) = L(Q(X)),
$$
ce qui g\'en\'eralise notre {\em curse of dimensionality}: plus la grille sous-jacente \`a $Q$ est grande, plus l'ordre de grandeur de la vraisemblance sur $X$ est grande, et donc plus grands sont les rapports de vraisemblances entre transformations test\'ees.  

D'une certaine mani\`ere, on dit juste ici que si on a une infinit\'e de points de contr\^ole, on peut estimer exactement (au sens d'une variance a posteriori nulle) les param\`etres d'une transformation param\'etrique. On ne dit pas que les appariements sont exacts. S'ils l'\'etaient, un nombre fini suffirait. 

Ceci semble paradoxal mais c'est un effet ``m\'ecanique'' de la dimension. Consid\'erons le cas similaire de la r\'egression aux moindres carr\'es, pour laquelle il est bien connu que la variance d'estimation est donn\'ee par: 
$$V = \sigma^2 (X^\top X)^{-1}$$ 

La matrice $X^\top X$ \'etant proportionnelle \`a la fr\'equence d'\'echantillonnage, V est une fonction d\'ecroissante de celle-ci, sauf \`a consid\'erer que la variance du bruit augmente avec la fr\'equence d'\'echantillonnage, ce qui a du sens physique si le bruit est moyenn\'e dans la p\'eriode d'\'echantillonnage. Ca n'en a pas pour nous car le ``bruit'' est ici l'incertitude marginale sur chaque appariement et n'a donc aucune raison ``physique'' de varier en fonction de l'\'echantillonnage. 

Force est de conclure que la variance a posteriori sur nos param\`etres de recalage est virtuellement nulle. Est-ce \`a dire que le recalage est exact? Bien s\^ur que non! 

D'abord, il y a toujours du biais car le mod\`ele de transformation ne peut \^etre parfait. Un cerveau n'est pas rigide. La technique d'imagerie introduit des distorsions g\'eom\'etriques. Il y aura toujours des d\'eformations r\'esiduelles. 

Ensuite, notre mod\`ele suppose connues les position des points dans les images, que ce soit les pixels ``natifs'' ou les points interpol\'es. Quand j'attribue \`a un pixel la position de son centre, n'est-ce pas arbitraire? La vraie position n'est-elle pas ``quelque part dans le pixel''? 

Imaginons l'exp\'erience suivante. On prend une fonction constante par morceaux, on l'\'echantillonne (en ajoutant du bruit et en moyennant le signal par pixel), puis on interpole le signal \'echantillonn\'e. Est-ce qu'on r\'ecup\`ere exactement la fonction originale? Est-ce que les {\em points saillants} (points de contour, coins...) peuvent \^etre localis\'es avec certitude? Evidemment pas

D'une part, cela signifie qu'il est impossible \`a un op\'erateur humain d'\'etablir une v\'erit\'e terrain exacte \`a partir d'amers rep\'er\'es \`a l'oeil. D'autre part, l'algorithme de recalage op\`ere sur des images qui pr\'esentent la r\'ealit\'e d'une mani\`ere ``floue'' non prise en compte par notre mod\`ele statistique.

Comment pourrait-il l'\^etre? La mani\`ere la plus intuitive, \`a mon sens, est de faire d\'ependre les appariements de la transformation de fa\c con non-d\'eterministe en introduisant une incertitude de localisation r\'esiduelle. C'est d\'ej\`a ce que Maes faisait implicitement en 1997 avec l'id\'ee de l'interpolation PV. Cela revient ici \`a modifier $P(Q|X)$ de fa\c{c}on simple. Pour calculer la vraisemblance $L(X)$, on peut utiliser un algorithme~EM dans lequel $Q$ est consid\'er\'e comme donn\'ee manquante. On voit alors que le calcul de la vraisemblance fait apparaitre des poids analogues \`a l'interpolation PV mais {\em modul\'es it\'erativement par la distribution conjointe d'intensitit\'e}.


\section{Synth\`ese}

Le terme "inférence composite" n'est pas bien choisi car la notion de vraisemblance composite est d\'efinie par ailleurs et est compl\`etement orthogonale \`a ce que nous faisons ici: elle correspond \`a l'utilisation de repr\'esentations multiples en parall\`ele pour l'inf\'erence. La "vraisemblance restreinte", elle aussi, est une notion orthogonale correspondant \`a l'utilisation de repr\'esentations emboit\'ees.

Le terme {\em vraisemblance r\'eduite} semble plus ad\'equat.  

La notion centrale est celle de {\em quantit\'e exhaustive}, \`a savoir une fonction des observations et des variables cach\'ees telle que la distribution g\'en\'erative des observations admette la forme: $p(y|x)=f(z(x,y),x)g(y)$. Notons qu'il existe toujours une telle quantit\'e puisque $z(x,y)=y$ marche toujours.

On montre alors que, \`a $x$ fixe, $z_x$ s'interpr\`ete comme une statistique exhaustive pour un certain probl\`eme d'inf\'erence binaire. D'o\`u l'interpr\'etation de la vraisemblance en termes d'aggr\'egation d'opinions. Ce n'est qu'une interpr\'etation possible d'un type de fonction de vraisemblances, concept par ailleurs on ne peut plus standard.

Je trouve l'expression "repr\'esentation adaptative" plus jolie que "quantit\'e exhaustive" mais elle sous-entend que chaque expert serait libre du choix de sa repr\'esentation. Or une repr\'esentation adaptative arbitraire est susceptible de fournir des classifieurs idiots comme celui que j'avais imagin\'e en 2015. La construction inverse par combinaison des experts (c-\`a-d la m\'ethode de Baggenstoss) n'a de sens que si on suppose que la repr\'esentation adaptive est une quantit\'e exhaustive (vis-\`a-vis de la repr\'esentation globale choisie).

Les choses deviennent non-standard quand on envisage l'\'elimination des param\`etres (de nuisance). L'intepr\'etation par aggr\'egation d'opinions justifie une approche distribu\'ee dans laquelle l'\'elimination des param\`etres est d\'el\'egu\'ee aux ``experts''. 

Notons que l'\'elimination distribu\'ee reste cependant compatible avec le paradigme classique car, lorsque la quantit\'e exhaustive n'est autre que l'observation elle-m\^eme, chaque expert travaille avec la m\^eme fonction de vraisemblance (en diff\'erents points). Ce que je propose est donc une g\'en\'eralisation de l'\'elimination traditionnelle de param\`etres.

On peut encore g\'en\'eraliser cette approche \`a la vraisemblance composite en traitant ind\'ependamment chaque composante comme une vraisemblance r\'eductible. C'est une piste de recherche. Hors sujet ici.


\section{Inf\'erence composite revisit\'ee}

J'ai fait fausse route avec la vraisemblance super-composite en recalage. La vraisemblance du recalage n'est ni super-composite, ni m\^eme composite. C'est une vraisemblance classique. Faut-il pour autant jeter ma m\'ethode d'inf\'erence composite \`a la poubelle?

Je doute fort aujourd'hui de l'int\'er\^et de la vraisemblance super-composite. J'avais montr\'e dans mon topo que l'apprentissage de poids adaptatifs induit un effet winner-take-all qui consiste \`a prendre un seul attribut par hypoth\`ese. On retombe alors sur une simple vraisemblance r\'eduite... 

C'est d\^u \`a la contrainte de poids de somme unitaire, directement h\'erit\'ee de notre interpr\'etation de la vraisemblance composite comme une aggr\'egation d'opinions. Faut-il remettre en cause cette vision des choses? 

Il y a quelque chose qui ne colle pas dans cette affaire. L'intuition du Bayes na\"if, aka vraisemblance composite classique (avec poids non-adaptatifs), est que chaque attribut apporte une contribution utile \`a l'inf\'erence, m\^eme s'il existe une certaine redondance entre attributs. 

Or, si on essaie d'apprendre les poids plut\^ot que de les choisir {\em a priori}, il faudrait alors prendre des poids adaptatifs selon ma th\'eorie... pour le r\'esultat qu'on conna\^it. C'est bien s\^ur absurde. Si, par exemple, les attributs sont vraiment ind\'ependants, cette m\'ethode fournit une pond\'eration totalement sous-optimale. Car elle suppose implicitement une sorte de "redondance maximale" entre les attributs qui n'est pas la r\'ealit\'e.

Revenons \`a l'aggr\'egation de vraisemblances. Si j'ai plusieurs $p(z_i|x)$, il est naturel de former une {\em somme} g\'eom\'etrique (log-linear pooling):
$$
p_\lambda(x|y) = \frac{\pi(x)}{Z(\lambda, y)}\prod_i p(z_i|x)^{\lambda_i},
\qquad 
Z(\lambda, y) = \int \pi(x) \prod_i p_i(z_i|x)^{\lambda_i} dx
$$

Pourquoi g\'eom\'etrique plut\^ot qu'arithm\'etique ou autre r\`egle d'aggr\'egation d'opinions?

On peut justifier cette forme par diff\'erents principes: Bay\'esien externe (Genest, 1986), axiomes de bon aloi (Tarentola et Valette, 1982), principes info-variationnels (Garg et al, 2004; Allard et al, 2012)... Toutes ces approches stipulent des poids unitaires ou une somme de poids unitaire qui, alli\'ee \`a un principe de raison insuffisante, justifie des poids uniformes, $\lambda_i\equiv 1/n$. En gros, elles justifient la m\'ethode Bayes na\"ive.

Mais la meilleure justification, \`a mon sens, c'est le principe du maximum d'entropie conditionnelle (Berger et al, 1996) sous des contraintes du type:
$$
E[\log p(z_i|x)] = \bar{\ell}_i,
$$
o\`u l'esp\'erance est prise sur $x$ et $y$. 

Pourquoi est-ce la meilleure approche? Tout simplement parce que le principe du maximum d'entropie est la solution d'un probl\`eme de pr\'ediction optimale (cf Gr\"unwlad, 2005) qui correspond exactement \`a ce qu'on veut faire ici: pr\'edire $x$ au mieux en fonction de $y$ en utilisant les vraisemblances restreintes $p(z_i|x)$. Au fond, nous avons ici un probl\`eme d'apprentissage discriminant et il n'y a pas de magie: on peut initialiser les param\`etres de fa\c con raisonnable, mais on fera forc\'ement mieux en laissant parler les donn\'ees.

Maxent nous dit que si on est capable d'estimer la log-vraisemblance moyenne pour chaque attribut, ce qui suppose une premi\`ere phase d'apprentissage "g\'en\'eratif", alors la forme ci-dessus est optimale en un certain sens pour la pr\'ediction de~$x$. Un deuxi\`eme apprentissage permet de d\'eterminer les poids des "experts" en tenant compte des d\'ependances entre attributs, en maximisant la vraisemblance (conditionnelle):  
$$
L(\lambda) 
= \sum_k \log \frac{p_\lambda(x_k|y_k)}{\pi(x_k)}
= \sum_k \left[
-\log Z(\lambda, y_k) + \sum_i \lambda_i \log p_i(z_{ik}|x_k)
\right],
$$
ce qui revient \`a assimiler les moments $\bar{\ell}_i$ \`a leurs moyennes empiriques dans la fonction duale du MaxEnt. C'est ni plus ni moins qu'un apprentissage supervis\'e. On peut voir ce 2e apprentissage comme une forme de {\em boosting} puisque il s'agit de combiner des inf\'erences simples.

Plusieurs remarques s'imposent:
\begin{enumerate}
    \item L'aggr\'egation d'opinions n'est autre qu'un classifieur MaxEnt d'un genre un peu particulier o\`u les contraintes de moments incorporent des mod\`eles g\'en\'eratifs appris au pr\'ealable plut\^ot que des quantit\'es arbitraires.
    \item  Une fois r\'ealis\'e le premier apprentissage, le deuxi\`eme apprentissage est d\'etermin\'e par les distributions g\'en\'eratives apprises et par la distribution marginale empirique $h(y)$, cf.~ma note {\tt oct14.pdf} qui contient par ailleurs des erreurs.
    \item On peut voir le MaxEnt conditionnel comme un MaxEnt relatif standard sur une distribution jointe de marginale connue $h(y)$, donc de la forme $p(x,y)=h(y)p(x|y)$, et par rapport \`a une loi de r\'ef\'erence de la forme $\pi(x,y)=\pi(x)h(y)$. Il est donc capital d'avoir une estim\'ee de la marginale $h(y)$.
    \item Le choix d'un {\em a priori} sur $x$ est n\'ecessaire pour d\'efinir les contraintes. Il est naturel de choisir $\pi(x)$ (qui est uniforme dans la pr\'esentation de Berger), ce qui revient \`a poser que la distribution empirique des donn\'ees ajust\'ee pour avoir $h(x)=\pi(x)$ est dans l'espace de recherche.
    \item L'espace de recherche est d\'efini par des contraintes portant sur les log-vraisemblances moyennes  qui sont {\em moins fortes} que supposer connues les distributions des attributs. Elles expriment l'id\'ee tr\`es MDL-esque que ce qui est important avec un mod\`ele, ce n'est pas qu'il soit exact, mais le taux de compression qu'il permet d'atteindre. 
    \item Doit-on utiliser des contraintes d'\'egalit\'e ou d'in\'egalit\'e? Les premi\`eres sont plus naturelles mais permettent des poids n\'egatifs... ce qui est malvenu pour des ``opinions''. 
    \item Toutefois, la m\'ethode n'impose pas que les distributions admissibles aient $\pi(x)$ comme loi marginale. En d'autres termes, tout mod\`ele ayant un fort pouvoir pr\'edictif est susceptible de nous int\'eresser qu'il utilise un {\em a priori} farfelu ou pas. C'est pragmatique \`a d\'efaut d'\^etre bay\'esien.
    \item Cela \'etant, on peut voir la r\`egle d'aggr\'egation comme une g\'en\'eralisation de la loi de Bayes dans laquelle $\pi(x)$ joue le r\^ole d'{\em a priori}. On pourrait dire un {\em a priori} composite.
    \item Une g\'en\'eralisation du MaxEnt conditionnel, le MIL (Yuan et Clarke, 1999) permet de rel\^acher la contrainte que $p(y)$ est connue et (\'eventuellement) d'imposer la contrainte $\pi(x)=p(x)$. Elle conduit \`a estimer en alternance les distributions marginale et conditionnelle $p(y)$ et $p(y|x)$ via l'algorithme de Blahut-Arimoto. \c Ca semble indiqu\'e si on n'a pas confiance dans la distribution empirique $h(y)$ mais, dans ce cas, pourquoi avoir confiance dans les moments? 
\end{enumerate}

Au final, la diff\'erence entre l'inf\'erence composite et un bon vieux classifieur MaxEnt (qui comprend notamment la r\'egression logistique comme cas particulier) tient au type de contraintes. Quel avantage potentiel a l'inf\'erence composite? C'est LA question.

Une bonne propri\'et\'e est d'\^etre asymptotiquement exacte dans le cas d'attributs mutuellement ind\'ependants (on a alors $\lambda\equiv 1$). Cela reste vrai en pr\'esence d'attributs r\'epliqu\'es:  les m\^emes attributs se partagent alors des poids de somme unitaire, c'est l'hypoth\`ese implicite de l'aggr\'egation bay\'esienne externe. 

Donc, intuitivement, l'inf\'erence composite est un bon mod\`ele pour des attributs peu ou tr\`es corr\'el\'es, le cas d\'efavorable \'etant a priori celui d'attributs mod\'er\'ement corr\'el\'es.

Ce raisonnement porte sur la forme du mod\`ele pr\'edictif mais se pose aussi la question de l'estimation des param\`etres. Le pr\'e-apprentissage "g\'en\'eratif" (qui fait penser \`a Hinton 2006) permet de r\'eduire le nombre de param\`etres \`a apprendre dans la phase d'apprentissage discriminant. C'est int\'eressant si l'apprentissage "g\'en\'eratif" est statistiquement efficace.

Pour comparer ce qui est comparable, l'alternative serait d'ajuster le mod\`ele pr\'edictif complet $p_{\lambda,\theta}(x|y)$, o\`u $\theta$ repr\'esente les param\`etres des lois g\'en\'eratives, par apprentissage discriminant. Malgr\'e son caract\`ere contre-intuitif compte tenu de la nature de $\theta$, cette approche est en principe indiqu\'ee dans la limite des grands \'echantillons (Ng et Jordan, 2001). 

Nous ne somme pas dans ce contexte. Le fait de travailleur avec une repr\'esentation donn\'ee $y=z_1,z_2,\ldots$, par exemple dans le cadre d'un apprentissage par transfert, revient \`a supposer que nous ne disposons pas d'un ensemble d'apprentissage gigantesque. 

Mais, m\^eme si c'\'etait le cas, est-ce que l'apprentissage purement discriminant ferait forc\'ement mieux? Dans le cas o\`u le mod\`ele est exact, la r\'eponse est non. Dans le cas d'attributs "mod\'er\'ement corr\'el\'es", c'est oui en principe. Beaucoup mieux? Je ne sais pas.

Quoiqu'il en soit, l'inf\'erence composite d\'ecoule d'une approche {\em divide and conquer} qui a le m\'erite de simplifier l'apprentissage sur le plan calculatoire, donc d'apprendre plus vite, de r\'eduire les risques de convergence locale, etc.

En d\'efinitive, cela fait pas mal d'arguments en faveur de l'inf\'erence composite. C'est du boosting de classifieurs probabilistes, une voie interm\'ediaire entre le Bayes na\"if et la r\'egression logistique, de l'apprentissage semi-g\'en\'eratif, c'est original, intuitif, relativement l\'eger sur le plan calculatoire...

Je finis par retomber sur mes pattes. La morale de l'histoire, c'est qu'il peut y avoir un b\'en\'efice \`a  d\'ecomposer l'apprentissage en deux \'etapes: une 1ere phase d'apprentissage g\'en\'eratif pour obtenir les mod\`eles $p(z_i|x)$, une 2e phase d'apprentissage discriminant pour obtenir la pond\'eration. Ca semble id\'eal pour des probl\`emes d'apprentissage {\em small data}. 

On peut imaginer des variantes avec des a priori (conjugu\'es de pr\'ef\'erence) sur les coefficients pour am\'eliorer l'efficacit\'e statistique de l'apprentissage. Imposer, par exemple, qu'ils ne s'\'ecartent pas trop de $1$ ou $1/n$. 

Dans le r\'egime {\em big data}, \`a repr\'esentation fixe, on a l'option MIL quitte \`a balancer des attributs \`a la pelle dans l'entonnoir. Question ouverte. 


\section{Compression}

Let $X\sim p_\star(x)$ and consider a model $q(x)$. Assume we know the compression rate of $q$, but don't know $p_\star$. Does it make sense to consider admissible any distribution $p(x)$ for which the compression rate is at least as large, hence satisfying the following inequality:
$$
L(p,q) \geq L(p_\star,q),
$$
with:
$$
L(p,q) = \int p(x)\log q(x) dx 
$$

This is weird because if we somehow know $L(p_\star,q)$, we know for sure that it can't be larger than what it is, right? So why not using an equality constraint? Unless what we think we know happens to be a lower bound. 


$$
\max_q\min_{p\in P} L(p,q) 
=
\min_{p\in P} \max_q L(p,q)
= \min_{p\in P} L(p,p)
= -\max_{p\in P} H(p)
$$



\section{Pivotalit\'e et remise en cause du principe de maximum de vraisemblance}

C'est ici que je brille de mille feux. Le bouquet final.


\section{Maximum entropie conditionnelle}

$$
D(p) = \int h(y)p(x|y) \log \frac{p(x|y)}{\pi(x)} dxdy
$$

Lagrangian:
$$
{\cal L}(p,\lambda) = D(p) -\lambda^\top (\int h(y)p(x|y)f(x,y)dxdy - F)
$$

Optimal $p$ at fixed $\lambda$:
$$
\frac{\partial{\cal L}}{\partial p}
=
h(y) \log \frac{p(x|y)}{\pi(x)} - h(y) \lambda^\top f(x,y)
= h(y) c(y)
\quad
\Rightarrow
\quad
p_{\lambda}(x|y) = \frac{\pi(x)}{z(\lambda,y)} e^{\lambda^\top f(x,y)}
$$
with:
$$
z(\lambda,y) = \int \pi(x) e^{\lambda^\top f(x,y)} dx
$$

Dual:
$$
\psi(\lambda) = \lambda^\top F - \int h(y) \log z(\lambda, y) dy 
$$

$$
\nabla z
=
\int \pi(x) f(x,y)  e^{\lambda^\top f(x,y)} dx
$$

$$
\nabla \nabla^\top z 
=
\int \pi(x) f(x,y) f(x,y)^\top e^{\lambda^\top f(x,y)} dx
$$

$$
\nabla\psi
= F - \int h(y) \frac{\nabla z}{z} dy
$$

$$
\nabla\nabla^\top\psi
= - \int h(y) \left(
\frac{\nabla \nabla^\top z}{z} 
-
\frac{\nabla z \nabla z^\top}{z^2}
\right)
dy
$$

Note the equivalent expression of the dual function as:
$$
\psi(\lambda)
=
\int h(x,y) \log \frac{p_\lambda(x|y)}{\pi(x)} dx dy
= D(h\|\pi) - D(h\|p_\lambda)
$$


\section{Maxent via the generalized KL}

It's a bit different as the unit integral constraint is made implicit.

Assume $\int\pi(x)dx=1$ and $\int h(y) dy = 1$. 
$$
D(p) 
= \int h(y)p(x|y) \log \frac{p(x|y)}{\pi(x)} dxdy
- \int h(y)p(x|y)dx dy
+ 1
$$

Goal is to minimize $D(p)$ s.t. mean-value constraints on $p$. 

Lagrangian:
$$
{\cal L}(p,\lambda) = D(p) -\lambda^\top (\int h(y)p(x|y)f(x,y)dxdy - F)
$$

Optimal $p$ at fixed $\lambda$:
$$
\frac{\partial{\cal L}}{\partial p}
=
h(y) \log \frac{p(x|y)}{\pi(x)} - h(y) \lambda^\top f(x,y)
= 0
\quad
\Rightarrow
\quad
p_{\lambda}(x|y) = \pi(x) e^{\lambda^\top f(x,y)}
$$

Dual:
$$
\psi(\lambda) = \lambda^\top F - \int h(y)z(\lambda, y) dy + 1
$$
with:
$$
z(\lambda,y) = \int \pi(x) e^{\lambda^\top f(x,y)} dx
$$

$$
\nabla\psi(\lambda)
= F - \int \pi(x) h(y) f(x,y) e^{\lambda^\top f(x,y)} dx dy
$$

$$
\nabla\nabla^\top\psi(\lambda)
= - \int \pi(x) h(y) f(x,y)f(x,y)^\top e^{\lambda^\top f(x,y)} dx dy
$$


\section{Are discriminative models Bayesian?}

My introduction on composite Bayesian inference seems to imply that discriminative models are not Bayesian. Well, they can be represented by a Bayesian network, so they are Bayesian, right? 

So what is the matter with them? Well, they seem to contradict the likelihood principle because we don't use a likelihood function of the variable of interest. Instead, the discriminative model gives us directly what we are intested in, the posterior distribution, as it is a joint distribution of the form:
$$
p(x,y) = \pi(y) p(x|y)
$$

Least we forget, the goal is to make an inference about $x$ from $y$. But both the ``likelihood'' and the ``prior'' are things we won't even try to compute: 
$$
p(y|x) = \frac{p(x,y)}{p(x)}, 
\qquad
p(x) = \int \pi(y) p(x|y) dy
$$

It would be like doing a double inversion... pretty stupid, isn't it? Nonetheless, everything happens {\em as if} we were doing just that -- it's mathematically equivalent. So can we say that discriminative models violate the likelihood principle? No, we can't.

Can we at least claim that $p(x,y)$ is not a realistic generative model? That would sound very picky if $\pi(y)$ is some large sample-based empirical distribution (we don't compute it in practice, but let's imagine...) and $p(x|y)$ is a reasonable predictive distribution. 

Again, what is the matter? The {\em real} difference between generative and discriminative models is that the former have all their parameters describing $p(y|x)$ while the latter have all their parameters describing $p(x|y)$. Is it a question of parameters at the end of the day?

No, it's a question of what the models describe. A generative model describes the process of acquiring data, it has to reflect some physical understanding. A discriminative model describes a convenient approximating family for the recovery of a state, there is no physics involved. As a by-product, the generative model might well produce a good predictive distribution for the data, but it does that without understanding anything about the data.

Just like an histogram is a pretty dumb but good predictor if we have a shit load of data to compute it. Instead, you can have a parametric model that reflects some understanding of what the distribution is. It takes much less data.

La vraie question, c'est de savoir quel est le lien de cause \`a effet entre $x$ et $y$. Si $x$ est la ``cause'' de $y$, alors le mod\`ele g\'en\'eratif est naturel. Si c'est l'inverse, alors le mod\`ele discriminant est une \'evidence.

Par exemple, supposons que $y$ soit une vitesse moyenne et $x$ la distance parcourue en un temps donn\'e. On a $x=t y$ et c'est m\^eme d\'eterministe, donc aucune question \`a se poser: ce mod\`ele purement discriminant dans ce contexte donne la r\'eponse.

On peut se dire que la vitesse est la cause de la distance parcourue, mais la vitesse se d\'eduit tout aussi bien de la distance: $y=x/t$. Donc on pourrait aussi s'en sortir avec un mod\`ele g\'en\'eratif.

Si maintenant notre mesure $y$ est bruit\'ee, il n'y a plus d'ambigu\"it\'e: on \'ecrit $y=x/t+\epsilon$, c'est-\`a-dire qu'on travaille avec un mod\`ele g\'en\'eratif. La vitesse mesur\'ee devient la ``cons\'equence'' de la distance \`a cause du bruit de mesure.

On voit bien que la notion de causalit\'e est assez foireuse.

On voit aussi que la d\'efinition officielle (Wikipedia et cie) de l'inf\'erence statistique est tr\`es restrictive: tirer des conclusions sur une population \`a partir d'un \'echantillon. C'est minable. Mais la d\'efinition permet de se mettre dans le cadre rassurant de l'\'echantillonnage. L'\'echantillon $y$ est bien ``g\'en\'er\'e'' \`a partir de la population.

Je pr\'ef\`ere la d\'efinition de Floyd Bullard: ``Statistical inference means drawing conclusions based on data''.

De mani\`ere g\'en\'erale, il y a l'id\'ee que les donn\'ees, ce qu'on mesure, peut \^etre pr\'edit par l\'etat du monde. Oui, il y a bien un lien de causali\'e implicite et la mesure arrive toujours en bout de cha\^ine.

Dans l'exemple pr\'ec\'edent, l'\'etat du monde c'est $(v,x)$ la vraie vitesse et la vraie distance. On a $y=v+\epsilon$ mais par bonheur il existe une relation entre $x$ et $v$, 
$p(x,v)=\delta(x-tv)p(v)$. Autrement dit le param\`etre de nuisance $v$ est facilement \'elimin\'e. Plus fondamentalement, les donn\'ees apportent une information sur l'\'etat du monde qui permet de mettre \`a jour sa connaissance.


\section{Composite EM algorithm for unsupervised learning}
\label{sec:gem}

In a nutshell: alternate unsupervised learning of partial model parameters ($M$-step) and re-estimation of label posteriors via maxent ($E$-step). 

\begin{itemize}
\item $CM$-step: given a posterior estimate $q(x|y)$, do just as if the features were independent and solve: 
$$
\max_{\theta,\pi}
\int \sum_i h(y)q(x|y) \log p_\theta(z_i|x)\pi(x) dx dy
$$
Assuming that the feature conditional distributions have non-overlapping parameters, this boils down to independent ML-type parameter estimations. In addition, the empirical proportions $\pi(x)$ of the unobserved labels are estimated via:
$$
\pi(x) = \int h(y)q(x|y) dy
$$
\item $CE$-step: given the generative parameters $\theta$ and the reference $\pi$, evaluate the average log-likelihoods:
$$
\bar{\ell}_i
= \int \pi(x)p_\theta(z_i|x)\log p_\theta(z_i|x) dx dy
$$
and update $q(x|y)$ according to the conditional maxent principle:
$$
\min_q
\int h(y)q(x|y) \log \frac{q(x|y)}{\pi(x)} dx dy
$$
subject to:
$$
\int h(y)q(x|y)\log p_\theta(z_i|x) dx dy = \bar{\ell}_i
$$
\end{itemize}

A question arises: how is $h(y)$ defined in both the $M$- and $E$-steps? One obvious possibility is to take it as the empirical distribution of the data, meaning that the empirical distribution of the labels plays the role of the ``prior'', as implicitely assumed above, but is modified on each iteration. 

Another option is to impose the prior $\pi(x)$ and hold it fixed throughout iterations by re-defining the joint empirical distribution before each $M$-step depending on the current posterior estimate via:
$$
h(x,y)=\pi(x)\frac{h_0(y)q(x|y)}{\int h_0(y)q(x|y) dy} ,
$$
yielding the following correction to the empirical data distribution:
$$
h(y) = \sum_i w_i\delta(y-y_i),
\qquad
w_i = \int \pi(x)\frac{q(x|y_i)}{\sum_j q(x|y_j)} dx
.
$$

Open question: are both methods equivalent in the case of the standard EM? I speculate that the answer is yes. But they are not in the more general case of multiple features, so we have to make a choice. 

The maxent problem involved in the $CE$-step is {\em exactly} the same as in the supervised case whether or not the data points are reweighted. We may see the $CE$-step as a surrogate for the actactual $E$-step, which can't be implemented because we lack a full generative model.

Are we minimizing a unique functional as in the variational EM algorithm? Possibly not. We may see this generalization as a two-player game. One player tries to predict several data features independently. The other player tries to predict labels. Together, they find patterns (clusters) in unlabeled data. Does the more general version converge? Does the game have a value? And so on.

In the case of a single feature, is this the well-known EM? The $CM$-step is clearly the same as the $M$-step, what about the $E$-step? It is the same if $h(y)p_\theta(x|y)=p_\theta(y|x)\pi(x)$. However, this condition won't hold if $h(y)$ is an empirical distribution... In this special case, the ``E player'' could use the data distribution model produced by the ``M player'' instead of the empirical one but, more generally, there is no data distribution model so he has to stick to the empirical one.

The key question to be answered is: how is this composite EM trick different from a plain mixture model EM assuming independent features? Basically, what does the $CE$-step bring that the $E$-step doesn't? 

% Accounting here for Xi'An comment on xianblog.wordpress.com
% The sum of the powers is constrained to be equal to one, even though
% I do not understand why the dimensions of the projections play nog
% role in this constraint. Simplicity is advanced as an argument,
% which sounds rather weak…
%This simple constraint is implied by the external Bayesianity requirement: as it turns out, the only aggregation operator which is both externally Bayesian and independent from $x$ boils down to plugging a composite likelihood with unit sum weights into Bayes rule, hence extending the classical notion of likelihood in Bayesian analysis.



\section{Blablabla...}


L'intuition qui est derri\`ere \c{c}a est que le max de vraisemblance donne une bonne estim\'ee de theta mais celle-ci \'etant d\'ependante des donn\'ees, elle ne d\'efinit pas un "code" valide. Le regret exprime la diff\'erence entre un code universel valide et le code "na\"if" fourni par MV. L'autre id\'ee sous-jacente \`a NML est qu'on ne suppose pas que la "vraie" distribution appartient \`a la famille. D\`es lors, il est naturel de consid\'erer le regret maximal comme crit\`ere, d'o\`u un probl\`eme minimax: 

$$Lnml(q) = max_p int p(y) log ML(y)/q(y) dy$$ 

$$==> argmin Lnml(q) = ML(y) / int ML(y) dy$$ 

Cette approche fournit une version corrig\'ee de la "vraisemblance profile" comme estimation de l'\'evidence du mod\`ele. Le terme correctif, qui exprime la complexit\'e du mod\`ele, ne peut \^etre ignor\'e que s'il s'av\`ere ind\'ependant de x. 

Dans le cas de la vraisemblance r\'eduite, ces deux codes sont disponibles en version r\'eduite, c-\`a-d en rempla\c{c}ant $y$ par $z_x$. Ils permettent alors de calculer les rapports d'\'evidence pertinents pour estimer $p(y|x)/\pi(y)$, sans d'ailleurs connaitre $\pi(y)$. C'est toute la substantifique mo\"elle de cette approche qu'est la vraisemblance r\'eduite. 


\end{document}
