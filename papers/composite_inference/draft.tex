\documentclass[english]{scrartcl}

\usepackage{fullpage}
\usepackage{amstext,amssymb,amsmath,amsthm}
\usepackage{graphicx,subfigure}
\usepackage{multirow}
\usepackage{url}
\usepackage{hyperref}
\usepackage{color}

\graphicspath{{.}{pics/}}

\title{Composite Bayesian inference}
%\date{}
\author{Alexis Roche\thanks{\url{alexis.roche@centraliens.net}}}

\def\x{{\mathbf{x}}}
\def\y{{\mathbf{y}}}
\newcommand{\blambda}{{\boldsymbol{\lambda}}}
\newcommand{\Blambda}{{\boldsymbol{\Lambda}}}
\newcommand{\bell}{{\boldsymbol{\ell}}}
\newcommand{\E}{\mathbb{E}}



\begin{document}

\maketitle

\begin{abstract}
We revisit and generalize the concept of composite likelihood as a method to make a probabilistic inference by aggregation of multiple Bayesian agents. The resulting class of predictive models, which we call composite Bayesian models, is a middle way between purely generative and purely discriminative models. This perspective gives insight to choose the weights associated with composite likelihood, either {\em a priori} or via learning; in the latter case, they may be tuned so as to minimize training cross-entropy, yielding an easy-to-solve convex problem. We argue that composite Bayesian inference trades off between interpretability and prediction performance, both of which are crucial to many artificial intelligence tasks.
\end{abstract}


\section{Introduction}
\label{sec:intro}

Textbook statistical inference (frequentist or Bayesian) rests upon the existence of a probabilistic data-generating model that is both empirically valid and computationally tractable. Because this double requirement may be very challenging for multidimensional data, other inference models have been developed in applied science: deliberately misspecified generative models, as in quasi-likelihood \cite{White-82,Walker-13} or na\"ive Bayes \cite{Ng-01} methods; data compression models as in minimum description length \cite{Grunwald-07}; and discriminative models\footnote{A {\em discriminative} model is a parametric family that describes the conditional distribution of the target variable given the data, while a {\em generative} model describes the joint distribution of the target and the data.}, which currently dominate the field of artificial intelligence (AI) and typically require supervised learning on large datasets -- these include many classical (shallow) learning \cite{Ho-95,BergerA-96,Vapnik-00,Rasmussen-06} and deep learning \cite{Lecun-15,Goodfellow-16} techniques (with the exception of deep belief networks \cite{Hinton-06}).

In a closed universe of possibilities, discriminative models can map data to predictions as well as intelligent beings or even better, however they lack introspection in the sense that they cannot {\em test} their predictions. Consider as a straightforward example deciding whether an object is a sauce pan or a frying pan based on its depth. A two-class discriminative model can learn a threshold that correctly classifies most pans in practice, but will confidently classify a blender as a sauce pan, while linear discriminant analysis\footnote{Despite the name, linear discriminant analysis is based on a generative model.}, for instance, could hint that a blender is a poor match to both pan categories. Inability to ``confirm'' or ``justify'' predictions makes discriminative models difficult to interpret.

%Discriminative models, however, lack the notion of {\em likelihood}: they cannot assess whether the data is consistent with their own prediction, however accurate, and in this sense do not have the ability to ``justify'' themselves. Consider as a straightforward example the problem of deciding whether an object is a sauce pan or a frying pan based on depth. A discriminative model can learn a threshold that correctly classifies most pans in practice, but is unable to see that, say, a blender does not match either category. The impossibility of such confirmatory analysis strongly limits interpretation if outlier inputs are to be expected.

There is growing awareness that AI should be interpretable in life-impacting applications \cite{Molnar-18}, where it is (and perhaps should remain) confined to a supportive role in human-driven decision-making processes, if only because decisions need to be explained and justified to other humans, while determining the set of possible outcomes may also be part of the process. For instance, in medical diagnosis, automated disease predictions need to be corroborated by individualized findings to be taken into account. In other words, there is little clinical value in ``black boxes'', but certainly more in automated methods that can justify their outputs just like human experts. One such method that has long been used by clinicians is the familiar reference range analysis, which stems from simple generative models.

%Imagine a doctor telling a patient: ``Sorry, sir, you have Alzheimer's disease because this fancy AI software said so''. Unless the said software is 100\% reliable, which is practically impossible, this would obviously be unacceptable. Acceptable would be to corroborate the diagnosis with some key quantitative empirical observations: low memory test scores, hippocampal atrophy, etc. In other words, inference has to come with some insight. An AI system is but one expert relying on hypotheses which may be invalid, as any expert, and is therefore prone to error. A typical error source is to train a classifier on a subset of classes from the real world. We cannot avoid inference errors but we can safeguard against them by being able to interrogate the system: why do you think what you think? To achieve that, the system should be capable of {\em introspection.

%Vapnik: ``one should solve the classification problem directly and never solve a more general problem as an intermediate step''. True if the goal is classification only. Wrong if introspection is needed because the problem is then more general in nature.

%Human inference does not work in a pre-determined universe of ``causes''. We confront theories with reality to evaluate how likely they are, while knowing that none of the theories considered so far may be ``right''. In fact, we are not looking for the truth, but for a conving enough theory. 

%Composite likelihood seen as a pool of experts comes with a potential introspection mechanism: first sort experts by decreasing influence on the predictive distribution, see how likely the different classes are to the most influential experts.

A related limitation of discriminative models is that they are not suitable for unsupervised learning or on-the-fly parameter estimation because they treat the data and the model parameters as {\em marginally independent} variables, meaning that the data conveys no information about the parameters unless the target variable is observed. This is illustrated in Figure~\ref{fig:graph_comparison} by the respective directed graphs representing generative and discriminative models. For the same basic reason, supervised learning in a discriminative model is statistically less efficient than in a generative model spanning the same family of posteriors, hence it requires more training data to achieve optimal performance \cite{Ng-01}. 

%Overall, pure discriminative models are of little use outside the context of big labeled data.

\begin{figure}[!ht]
\begin{center}
\subfigure[Generative model]{\includegraphics[width=.25\textwidth]{generative.pdf}\label{fig:generative}}
\hspace*{.2\textwidth}
\subfigure[Discriminative model]{\includegraphics[width=.25\textwidth]{discriminative.pdf}\label{fig:discriminative}}
\caption{Directed graphs representing a generative and discriminative models, where $X$, $Y$ and $\theta$ respectively denote the target variable, the data, and the model parameters. Note the marginal independence of data and parameters in the discriminative model.}
\label{fig:graph_comparison}
\end{center}
\end{figure}

This note advocates probabilistic opinion pooling \cite{Genest-86} as a natural way to merge discriminative and generative modeling approaches in order to make predictions that are both accurate and interpretable. The key idea is to combine multiple low-dimensional generative models corresponding to different pieces of information extracted from the input data. Each such model acts as an isolated ``agent'' that uses a single feature to express a testable opinion in the form of a likelihood function of the target variable. The agent opinions are then aggregated into a unique predictive probability distribution analogous to a Bayesian posterior. This strategy may be understood as a probabilistic version of boosting, or as a ``divide and conquer'' approximation to the intractable Bayesian posterior. 

%Importantly, each agent opinion is justifiable as it is based on a generative model.

When choosing the aggregation method as a log-linear pool, the predictive distribution turns out to be proportional to a quantity known in computational statistics as composite likelihood (CL), see \cite{Varin-11} and the references therein. CL was developed as a surrogate of traditional likelihood for parameter estimation. While maximum CL does not inherit the general property of maximum likelihood to achieve asymptotical minimum variance, it is asymptotically consistent under mild conditions \cite{Xu-11} (see Appendix~\ref{app:frequentist} for a basic weak consistency proof), and may offer an excellent trade-off between computational and statistical efficiency in practice.

Based on the opinion pooling interpretation, the weights assigned to the different features in~CL may be optimized for prediction performance in a typical supervised learning scenario. As we will see, this strategy amounts to training a maximum entropy classifier using the feature-based log-likelihoods as basis functions. The likelihood functions themselves may need to be pre-trained, therefore the composite Bayesian approach typically involves a two-stage training scheme: a generative training (to learn the feature-based likelihood parameters) followed by a discriminative training (to learn the feature weights in aggregation).

The opinion pooling framework also suggests a generalization of CL, which we call {\em super composite likelihood}, in which the features may be weighted depending on the target values, leading to a more flexible predictive model that can be trained similarly to CL.

%Technical details are given in the remainder. 

The reader is warned that slightly abusive mathematical notation is sometimes used in the remainder. This is done deliberately for the sake of clarity in places where we think that notation can be lightened without raising ambiguity. 


\section{Composite likelihood as opinion pooling}
\label{sec:log_pool}

Let $\mathbf{Y}$ an observable multivariate random variable with sampling distribution $p(\y|x)$ conditional on some unobserved variable of interest, $X\in{\cal X}$, where ${\cal X}$ is assumed to be a finite set for simplicity. Given an experimental outcome $\y$, the likelihood is the sampling distribution evaluated at $\y$, seen as a function of $x$:
$$
L(x) = p(\y|x)
.
$$

This requires a plausible generative model which, for complex data, may be out of reach or involve too many nuisance parameters. A natural workaround known as data reduction is to extract some lower-dimensional representation $z(\y)\sim f(z|x)$ using a many-to-one mapping, and consider the potentially more convenient likelihood function:
$$
\ell(x) = f(z|x)
.
$$

Substituting $L(x)$ with $\ell(x)$ boils down to restricting the feature space, thereby  ``delegating'' statistical inference to an ``agent'' provided with partial information. While it is valid for such an agent observing~$z$ only to consider $\ell(x)$ as the likelihood function of the problem, the drawback is that $\ell(x)$ might yield too vague a prediction of~$X$ due to the information loss incurred by data reduction. To make the trick statistically more efficient, we may extract several features, $z_i(\y)$ for $i=1,2,\ldots,n$, and try to combine the likelihood functions $\ell_i(x) = f(z_i|x)$ that they elicit.

If we see the likelihoods as Bayesian posterior distributions corresponding to {\em uniform} priors, this is a problem of probabilistic opinion aggregation from possibly redundant sources, for which several methods exist in the literature \cite{Tarantola-82,Genest-86,Garg-04,Allard-12}. A common choice, owing to its simplicity and tendency to produce single peaked distributions, is log-linear pooling\footnote{When $\pi(x)$ is not uniform, it is also called {\em generalized} log-linear pooling, however this terminology could be confusing here since we will present another generalization.}:
\begin{equation}
\label{eq:log_pool}
p_\blambda(x|\y) \propto \pi(x) \prod_{i=1}^n f(z_i|x)^{\lambda_i},
\end{equation} 
where $\pi(x)$ is a reference distribution, and $\blambda=(\lambda_1,\ldots,\lambda_n)$ is a vector of weights. We shall asume for feasibility that the weights are intrinsic to the agents and are therefore independent from the data~$\y$. Also, negative weights should be ruled out to warrant a natural monotonicity property: if all agents agree that an outcome $x_1$ is less probable than an outcome $x_2$, this should be reflected by the consensus probabilities, {\em i.e.}, we should have $p_\blambda(x_1|\y)\leq p_\blambda(x_2|\y)$. 


\section{Composite Bayes rule}
\label{sec:bayes_rule}

Log-linear pooling~(\ref{eq:log_pool}) bears a striking similarity to Bayes rule as it yields the form: 
$$
p_\blambda(x|\y)\propto \pi(x) L^c_\blambda(x),
$$
where the distribution $\pi(x)$ plays the role of a prior, and the function:
\begin{equation}
\label{eq:comp_lik}
L^c_\blambda(x) \equiv \prod_{i=1}^n \ell_i (x)^{\lambda_i}
\end{equation} 
plays that of a likelihood function, and happens to be known in computational statistics as a {\em marginal composite likelihood} \cite{Varin-11}. The slightly more general {\em conditional composite likelihood} form can be derived in the same way as above by conditioning all probabilities on confounding variables, see Appendix~\ref{app:conditional}. 

The clear computational advantage of CL over genuine likelihood is that it is more efficient to evaluate the marginal distributions of each feature than the joint distribution of all features. CL shares a convenient factorized form with the likelihood derived under the assumption of mutual feature independence, often referred to as {\em na\"ive Bayes} method in the machine learning literature, which corresponds to the special case of unitary weights, $\blambda\equiv 1$. Since log-linear opinion pooling does not assume feature independence, it yields both an alternative interpretation and a generalization of na\"ive Bayes, whereby unitary feature weights may be used by default but do not guarantee optimal prediction performance.


\section{Composite likelihood calibration}
\label{sec:calibration}

CL involves two types of parameters:
\begin{enumerate}
    \item ``generative'' parameters, {\em i.e.}, the parameters of the feature generation models $f(z_i|x)$; 
    \item ``discriminative'' parameters, {\em i.e.},  the weights~$\blambda=(\lambda_1,\ldots,\lambda_n)$ assigned to the feature-based likelihood functions in~(\ref{eq:log_pool}).
\end{enumerate}

If training data is available, the generative parameters may be set in a first stage using standard estimation techniques. Otherwise, they may be considered as nuisance parameters and eliminated at prediction time by applying to CL one of the same techniques as for traditional likelihood \cite{Berger-99}. We focus in the sequel on tuning the composite weights.


\subsection{Agnostic weights}

In the absence of knowledge, uniform weights may be chosen under a rule of indifference. CL is then a scaled version of na\"ive Bayes likelihood, the unique weight value being irrelevant to the maximum CL estimator (MCLE). To get meaningful predictive probabilities, it may be tuned empirically so as to best adjust the pseudo posterior variance matrix to the asymptotic MCLE variance matrix \cite{Pauli-11}, or via a close-in-spirit curvature adjustment \cite{Ribatet-12}, as proposed in previous attempts at Bayesian inference from composite likelihood. 

Alternatively, a simple agnostic recommendation we believe to be useful for composite weights is to sum up to one. This is motivated by a characterization theorem for opinion pooling \cite{Genest-86,Genest-86b}: the only {\em externally Bayesian} pooling operator for which the consensus probability of an outcome only depends on the agent probabilities of the same outcome, is the log-linear opinion pool with unit sum weights. Externally Bayesian essentially means that the consensus should not change if an agent opinion is taken into account by the other agents as opposed to being stated independently. Log-linear pooling with unit sum weights also turns out to be an optimal {\em compromise} in the sense that it minimizes the average inclusive Kullback-Leibler (KL) divergence to the agent opinions \cite{Garg-04}. 

%While other divergence measures lead to other optimal pooling operators (for instance, the exclusive KL divergence leads to the linear pool), they all enforce unit sum weights, as is natural in the search for a best compromise. 

Unit sum weights, however, implicitly assume maximum redundancy between features, hence tend to produce conservative predictive probabilities (see Appendix~\ref{app:frequentist}). This means low prediction confidence (over-estimated credibility sets) when features are weakly correlated, as is desirable in general. Conversely, unitary weights ($\blambda\equiv 1$) as in Na\"ive Bayes assume independent features and may thus yield over-confident predictions. An ideal method to weight features is one that can capture feature redundancy and balance it with feature relevance \cite{Peng-05}. This suggests a learning approach whenever feasible.

%, which, as we shall see, is unlikely to pick uniform or unit sum weights.


\subsection{Learning the weights}
\label{sec:learning}

Assuming a training dataset ${\cal D}=\{(x_k,\y_k), k=1,\ldots,N\}$ (possibly the same as for generative pre-training), the most direct way to learn the composite weights is to maximize their likelihood under the composite predictive distribution (or, equivalently, minimize their ``cross-entropy''):
\begin{equation}
\label{eq:train_likelihood}
\max_{\blambda\succeq 0} U(\blambda),
\qquad \text{with} \quad
U(\blambda) \equiv\sum_{k=1}^N \log p_\blambda(x_k|\y_k).
\end{equation}

From~(\ref{eq:log_pool}), we see that the set of conditional distributions $p_\blambda(x|\y)$ spanned by the weights~$\blambda$ is the exponential family with natural parameter~$\blambda$ and basis functions given by the feature-based log-likelihoods:
$$
p_\blambda(x|\y) = \pi(x) \exp[\blambda^\top \bell(x,\y) - a(\blambda,y)],
$$
with:
$$
\ell_i(x,\y) \equiv \log f(z_i|x),
\qquad
a(\blambda, \y) \equiv \log \sum_{x\in{\cal X}} \pi(x) e^{\blambda^\top \bell(x,\y)}.
$$

It follows that the utility function $U(\blambda)$ in~(\ref{eq:train_likelihood}) is concave as a general property of likelihood in exponential families, hence this learning can be implemented using a standard convex optimization algorithm such as limited-memory BFGS \cite{Byrd-95}. See Appendix~\ref{app:training} for some implementation details. 

Because some positive weight constraints may turn out inactive, optimization has a natural tendency to produce sparse weights. As illustrated in Figure~\ref{fig:disc_weight_plot} on the breast cancer diagnostic UCI dataset \cite{Wolberg-94}, there is no clear relation between optimal weights and feature-level discrimination power, showing that maximum likelihood learning differs radically from univariate feature selection as it implicitly takes feature redundancy into account via joint weight optimization.

%\footnote{\url{https://archive.ics.uci.edu/ml/datasets/Breast+Cancer+Wisconsin+(Diagnostic)}}

\begin{figure}[!ht]
  \begin{center}
    \includegraphics[width=.6\textwidth]{disc_weight_plot.pdf}
  \end{center}
\caption{Distribution of optimal composite weights vs.~individual feature discrimination power in a binary classification task (breast cancer UCI dataset). Discrimination is measured by the maximum KL divergence between class-conditional generative distributions.}
\label{fig:disc_weight_plot}
\end{figure}

As is customary and generally good practice, the training examples may be weighted in~(\ref{eq:train_likelihood}) so that the empirical distribution of classes matches the prior~$\pi(x)$, thus enforcing balanced classes if the prior is uniform. Also, some regularization may be needed for stability: we shall in practice maximize $U(\lambda)-\alpha \|\blambda\|^2$, where $\alpha>0$ is a small damping factor.

Other training variants that we mention here but do not particularly recommend for interpretability arise from adding basis functions to the exponential family. For instance, class-dependent offsets may be added so as to learn the ``prior'' together with the composite weights. It is also possible to perform unconstrained optimization to achieve higher training likelihood, but this means that some composite weights are then negative, in violation of the monotonicity property mentionned in Section~\ref{sec:log_pool}.


\paragraph{$I$-projection interpretation.}

Exponential family properties also imply that learning via likelihood maximization~(\ref{eq:train_likelihood}) is dual to an $I$-projection \cite{Csiszar-84}: 
\begin{equation}
\label{eq:i_proj}
\min_{p\in{\cal P}} D(p\|p_0),
\qquad \text{with} \quad
p_0(x,\y) \equiv \pi(x)h(\y),
\end{equation}
subject to the constraints $p(\y)=h(\y)$ and $\E_p[\bell] \succeq \E_h[\bell]$, where $\E_h$ denotes the expectation with respect to the joint {\em empirical} distribution of targets and inputs, $h(\y)$ is the corresponding empirical marginal distribution of inputs, and ${\cal P}$ is the set of joint probability distributions on ${\cal X}\times {\cal Y}$ with ${\cal Y}\equiv\{\y_1,\ldots,\y_N\}$. Note that the problem statement implicitly approximates the true (unknown) joint distribution of $(x,\y)$ by its empirical estimate, hence the need for regularization in practice.

The marginal constraint $p(\y)=h(\y)$ implies that only the conditional distribution $p(x|\y)$ is optimized and that the $I$-projection is equivalent to maximum conditional entropy \cite{BergerA-96}. The mean log-likelihood constraints $\E_p[\bell] \succeq \E_h[\bell]$ encapsulate dependences between the target and the data, and consider a model admissible if it guarantees the same average log-likelihood levels as observed separately for each feature. This is weaker a constraint than assuming that the feature generative models $f(z_i|x)$ are true: it only assumes that the {\em description length} \cite{Grunwald-07} achieved by each feature model is a truthworthy upper bound.

%In the $I$-projection framework, the composite weights are the Lagrange multipliers associated with the mean log-likelihood inequality constraints.

An insightful analysis given in \cite{Grunwald-04} shows that $I$-projection is, under broad conditions, a minimax strategy for prediction according to the log loss. Hence, any $I$-projection is in some sense a best possible predictive model given some arbitrary knowledge exctracted from the training data, which is represented by mean-value constraints and is inexact due to the limited training set size. Optimized composite likelihood only differs from other maximum entropy classifiers such as multinomial logistic regression by its underlying inexact knowledge, which in this case may be interpreted in terms of  feature description length.


\section{Super composite likelihood}
\label{sec:super}

So far, we have assumed that each agent is given a single weight in the opinion aggregation~(\ref{eq:log_pool}). Let us now consider a more general pooling operator where agents can be weighted depending on the target variable:
\begin{equation}
\label{eq:super_pool}
p_\blambda(x|\y) \propto \pi(x) \prod_{i=1}^n \left[\frac{\ell_i(x)}{\ell_i(x_0)}\right]^{\lambda_i(x)},    
\end{equation}
where~$x_0$ is an arbitrary fixed target value, which we call the reference, and $\blambda: {\cal X}\to \mathbb{R}_+^n$ is a weighting function here. Note that the reference weight values $\lambda_i(x_0)$ play no role and can therefore be conventionally assumed to be zero. As discussed in Appendix~\ref{app:hetero}, the main reason for introducing the normalization by a reference is to make the pooling externally Bayesian if the weighting function sums up to one uniformly,
$$
\forall x\in{\cal X}\setminus \{x_0\},\quad
\sum_{i=1}^n \lambda_i(x) = 1,
$$
hence providing a sensible default rule to tune the weights. Note that (\ref{eq:super_pool}) amounts to a log-linear pool if the weights are target-independent as the reference is then effectless. Using the same analogy with Bayes rule as in Section~\ref{sec:bayes_rule}, we see that the quantity:
\begin{equation}
\label{eq:super_comp_lik}
L^{sc}_\blambda(x) \equiv 
\prod_{i=1}^n \left[\frac{\ell_i(x)}{\ell_i(x_0)}\right]^{\lambda_i(x)}
\end{equation} 
acts as a standard likelihood at prediction time. We call this form super composite likelihood (SCL) owing to the ``doubly composite'' aspect of combining distinctive feature-target pairs. SCL further generalizes CL since both forms are seen to be proportional for target-independent weights. 

Conversely, if the weighting function maps each target to a unit vector ({\em i.e.}, $\forall x\in{\cal X}\setminus \{x_0\}$, $\lambda_i(x)=1$ for a single feature~$i$, and zero otherwise), then SCL turns out identical to the {\em class-specific} likelihood surrogate derived in \cite{Baggenstoss-03} from a generative model selection argument; namely, the {\em PDF projection theorem}, which in fact boils down to an $I$-projection \cite{Minka-04}. While our derivation follows a {\em discriminative} model selection approach, it is interesting to see that it includes the class-specific method as a special case, and thus sheds new light on this method.

A compelling advantage of SCL over CL is that it can deal with missing likelihood values, which happen if the feature generative distributions $f(z_i|x)$ are unknown for some targets~$x$. SCL makes it possible to assign zero weights $\lambda_i(x)$ to every feature-target pair for which $\ell_i(x)$ is unknown, yet using nonzero weights for other pairs. Likewise, the features for which the reference likelihood $\ell_i(x_0)$ is unknown receive zero weight for all target values, implying that they have no influence whatsoever on the predictive distribution.

%Interest? If some feature likelihood values are missing, i.e. agents are not able to model the whole set of targets. It's a bit like censoring. Then it's ok if they can model at least two targets including the reference: $\ell_i(x)$ is unknown for the other values but we can give them zero weight so it doesn't matter.

If training data is available, the SCL weights may be learned by maximum likelihood as described for CL in Section~\ref{sec:learning}, yielding a similar convex problem. Since SCL enables to explore a wider class of predictive models than CL, it has the potential to achieve better prediction performance as long as overfitting does not happen (although SCL may not be asymptotically consistent unlike CL, see Appendix~\ref{app:frequentist}). The $I$-projection interpretation still holds, yet with class-specific mean-value constraints:
$$
\forall a\in{\cal X}, \forall i=1,\ldots,n,
\quad
\E_p\left[\delta_{xa}\log \frac{f(z_i|x)}{f(z_i|x_0)}\right]
\geq \E_h\left[\delta_{xa}\log \frac{f(z_i|x)}{f(z_i|x_0)}\right],
$$
where $\delta$ denotes the Kronecker delta. 

%Note that 

%In summary, SCL is a natural generalization of CL with greater flexibility in weight assignments, hence potentially superior prediction performance if trained on a large enough dataset. It, however, depends on an arbitrary reference class. 


%Beside greater flexibility than CL, a disadvantage of SCL it that interpretation becomes relative to an arbitrary reference~$x_0$. In some applications, there may exist a natural reference class to cling to in common practice (for instance, normality in medical diagnosis), but this is not always the case. 



%Another shortcoming is that SCL is not guaranteed to be asymptotically consistent, see Appendix~\ref{app:frequentist}, so it may yield biased MCLE despite 

%Disadvantage: interpetation is relative to an arbitrary class $x_0$. Advantages: richer model (so better predictions), probabilty ratios stable to class addition. Drawbacks: require a reference, no more consistency. Still interpretable, that's why it is worthwile mentioning.


%Argue that SCL more robust than CL to mispecification of feature models. Intuitively, if a feature model is poor, the empirical description length is large, hence can be achieved with a `small' Lagrange multiplier $\lambda_i(x)$. Therefore, unreliable likelihoods $\ell_i(x)$ tend to get low weights.

% which does not contradict the axiomatic derivation in \cite{Genest-86b} as it restricted itself to operators that only depend on the probas evaluated at~$x$ unlike~(\ref{eq:super_pool}).

%The real motivation: when agents operate on restricted target sets. Without loss of generality, assume binary sets. Trick: for each such agent, say we have $p(z_i|x_1)$ and $p(z_i|x_0)$. We define $p(z_i|x)=p(z_i|x_0)$ if $x\not= x_1$. It does not make any sense but the likelihood ratio is then one for any $x\not= x_1$ so won't hamper the system. 


%Due to the distribution of weights between clues, a drawback of composite likelihood is that it is prone to {\em information overload} in the sense that it tends to ``flatten out'' when too many clues are included as relevant clues then get downweighted. If one decides not to merge clues for computational reasons (this would require handling their joint distribution), one could hope to mitigate information overload by assigning strong weights only to those clues that are believed to be ``most informative''. 
% When chosen for computational simplicity, features may not only convey limited information at individual level, their informativeness may also be very much class-dependent. Consider, for instance, diagnosing a disease from a routine medical checkup. Body temperature may point to a bacterial infection by comparison with normality, but would not help detecting a non-infectious cardiovasc ular disease -- and conversely for, say, blood pressure. 

%Reference is implicit if weights are class-independent as it is then effectless. Make sense to have a reference as it gives a customizable baseline for detection. Can be the feature distribution averaged over all classes (not recommended I would say) or the distribution of a particular class if that makes sense (e.g. healthy population). In this case, the super composite likelihood is equivalent to the standard one in the two-class case.

%If some classes are merged or split, then the other classes keep the same weights (assuming $h(\y)$ is kept constant, so we need to make sure the prior is adapted accordingly)! So if a patient was found to have 10 times more probability to be MCI than healthy (and healthy is the reference class), and we later split AD population into, say, pure AD and mixed AD, the MCI probably ratio fact will remain.

%Super composite likelihood is essentially about combining binary versions of composite likelihood. It is analogous to the ``one-versus-rest'' strategy in multiple logistic regression. 

%Still externally Bayesian if weights sum up to one in each class. Alternatively, we can train this model by maximum likelihood, yielding a similar optimization problem as in Section~\ref{sec:learning}. 

%Maxent interpretation. It is an alternative to the strategy considered above, the mean log-likelihood constraints may be expressed for each class rather than being averaged with respect to the prior, leading to a more general predictive model that we call ``super composite likelihood'', in which the weights become class-dependent (see Appendix~\ref{sec:super}). This variant has the potential for higher prediction performance, but poses some conceptual difficulties, in particular that of having to define an arbitrary reference class. 

%used to predict whether breast lumps are benign or malignant based on a set of features extracted from fine-needle aspirate images.

%The most discriminative features tend to get large weight, however some quite discriminative features may get low weight if strongly correlated with a more discriminative feature

%Non-linear means that features with average discrimination power may get low weight as they are superseded by the most discriminative features. But some features with low discrimination power may also get large weight, which does not mean that they are very influential in practice (see above). This is illustrated in Figure~\ref{fig:disc_weight_plot} with the UCI breast cancer dataset. 

%{\color{red} Equality constraints (negative weights) are less interpretable.}

%{\color{red} Offsets.}

%{\color{red} Data weights.}

%Need to weight the data to adjust to prior as a general measure when minimizing cross-entropy, which should yield a conditional model that approximates the ``true'' conditional distribution, $q(x|y)\approx p(x|y)$. So the model will absorb the ``true'' marginal of $x$, which should thus be replaced by the prior if we want the model to mimic ideal Bayesian inference. In our case, it means that the composite likelihood approximates the true likelihood.

%We can further enforce exact prior matching by including offsets ($K-1$ free parameters), in which case the training phase learns $\pi(x)$ so that it matches the prior $\pi(x)$ in the sense $\int h(\y)p_\lambda(x|y)dy = \pi(x)$. This ``deforms'' the prior to improve the training score. But why is it important to enforce this constraint? Again, it all makes things just less interpretable. 

%When including offsets, using homoscedastic generative variance estimation and equality constraints, composite likelihood is equivalent to binomial logistic regression. However, equality constraints are not easily interpretable.

%In some sense, composite weights replace the multiple testing correction required in the frequentist inference paradigm for proper control of false positives. Here, we don't control false positives, we control prediction accuracy.

%Why not using the feature-based posteriors or Bayes factors rather than the likelihoods? It's a possibility (perhaps a connection here with Gr\"unwald's luckiness). A justification for not doing it is that the coordinator should discard the priors used by the agents. The constraints are not purely evidence-based since the moments depend on the coordinator's prior but it is arguable a mess of priors wouldn't be good.

%Single feature case ($y=z$): the maxent solution has the form $p(x|y)\propto\pi(x)p(\y|x)^\lambda$. We expect to have $\lambda=1$ but it's not necessarily the case. It is if $h(\y)=\int\pi(x)p(\y|x)dx$ because then $h(\y)p(x|y)=\pi(x)p(\y|x)$ so the constraint is verified (and active). This also tells us that the data marginal should be consistent with the prior on labels for this much expected consistency property to hold. So, either weight datapoints so as to match a desired prior (e.g., uniform) or take the empirical distribution of labels as the ``prior''. 

%Which brings another question: in this case, the (unique) weight is insensitive to the prior -- is it true in general? The answer is no. The weights generally depend on the prior. Hence the maxent composite likelihood is prior-dependent. This is an important conceptual difference with the classical notion of likelihood.



\section{Discussion}
\label{sec:discussion}

The composite Bayesian framework presented here unifies several seemingly unrelated concepts from computational statistics and machine learning: na\"ive Bayes, composite likelihood \cite{Varin-11}, class-specific method \cite{Baggenstoss-03}. It ultimately provides a rich class of trainable predictive models that work on pre-determined data features, and is therefore suitable for shallow learning or transfer learning tasks, where it may compete with popular discriminative models such as logistic regression, support vector machines or random forests.

Unlike previous attempts at Bayesian inference from incomplete statistical models relying on generative model selection \cite{Yuan-99b,Wang-14} or asymptotic theory \cite{Pauli-11,Ribatet-12}, our construction is based on a discriminative model selection approach. Composite Bayesian models are thus explicitly taylored to prediction and cannot simulate complete data but, contrary to conventional discriminative models, they encapsulate feature generation models.

Training a composite Bayesian model involves the sequential estimation of feature generative parameters and feature relevance weights. This two-stage training is reminiscent of restricted Boltzmann machines (RBMs) \cite{Hinton-06,Fischer-14}, which are fully generative models, with the important difference that it operates on {\em disjoint} sets of parameters in the case of composite Bayesian models, while RBM training modifies the same parameters twice. Also, the generative pre-training of RBMs is typically unsupervised, and it is an open research issue whether composite Bayesian models are suitable for unsupervised learning.

The semi-generative nature of composite Bayesian models enables to test, at prediction time, how consistent the data at hand is with each feature distribution, which in turn enables to validate the overall model prediction. Moreover, since feature log-likelihoods are combined additively, it is easy to determine which features contribute most to a particular comparison of hypotheses about the target, and summarize the rational for a prediction by highlighting a few decisive features. 

{\color{red}Example.}

We believe that there are many applications of statistical inference in which interpretation is the real goal while prediction performance is only an intermediate requisite. A prediction method needs to prove reliable to be relevant; but, once proven reliable, it may need to be interpreted. Composite Bayesian inference may come into play when the {\em why} question is more important than the {\em what} question.


\appendix


\section{Basic frequentist properties of composite likelihood}
\label{app:frequentist}

Let a multivariate observable variable~$\y \sim p(\y|x_\star)$ distributed according to some target value~$x_\star$. For any hypothetical target value~$x$ and any extracted feature $z_i\sim f(z_i|x)$, the following double inequality holds:
$$
0 \leq
E\left[
\log \frac{f(z_i|x_\star)}{f(z_i|x)}
\right]
\leq
E\left[
\log \frac{p(\y|x_\star)}{p(\y|x)}
\right],
$$
where the expectation is taken with respect to the true distribution $p(\y|x_\star)$ at fixed~$x_\star$. This fact follows from two basic properties of KL~divergence: positivity and partition inequality.

Using a weighted sum of such inequalities, we can bracket the expected variations of the logarithm of any composite likelihood function~(\ref{eq:comp_lik}) with positive weights~$\blambda\succeq 0$:
\begin{equation}
\label{eq:variation_bound}
0 \leq
E\left[ \log \frac{L_c(x_\star, \blambda)}{L_c(x,\blambda)} \right]
\leq 
s_\blambda E\left[ \log \frac{L(x_\star)}{L(x)} \right]
,
\end{equation}
with $s_\blambda\equiv \|\blambda\|_1 =\sum_i \lambda_i$, and $L(x)\equiv \log p(\y|x)$ is the true likelihood function.

This implies two general asymptotic properties of composite likelihood:
\begin{itemize}
\item {\em Weak consistency.} The expected composite log-likelihood is maximized by~$x_\star$.
\item {\em Conservativeness.} If the weights sum up to one or less ($s_\blambda\leq 1$), composite likelihood ratios of the true target~$x_\star$ vs.~other target values~$x$ tend to  underestimate true likelihood ratios.
\end{itemize}

Note that the derivation assumes that the weights~$\blambda$ are independent from the target~$x$, therefore these properties do not necessarily extend to SCL, for which weights are target-dependent.


\section{Conditional composite likelihood}
\label{app:conditional}

As a straightforward  extension of marginal CL, each feature-based likelihood may be conditioned by an additional ``independent'' feature $\nu_i(\y)$ considered as a predictor of the ``dependent'' feature, $z_i(\y)$, yielding the more general form:
\begin{equation}
\label{eq:cond_feat_lik}
\ell_i(x) = f(z_i|x,\nu_i).
\end{equation}

Conditioning may be useful if it is believed that $\nu_i$ alone provides little or no information about~$x$, but is informative when considered jointly with~$z_i$, as in the case of regression covariates, for instance. Equation~(\ref{eq:comp_lik}) then amounts to conditional CL \cite{Varin-11}, a more general form of CL which also includes Besag's historical {\em pseudo-likelihood} \cite{Besag-74} developed for image segmentation.


\section{Composite likelihood training}
\label{app:training}

Using the same notation as in Section~\ref{sec:learning}, the likelihood function in~(\ref{eq:train_likelihood}) can be expanded as follows:
$$
U(\blambda) 
= \sum_{k=1}^N \left[
\log \pi(x_k) + \blambda^\top \bell(x_k, \y_k) - a(\blambda,\y_k)
\right]
$$

Maximizing $U(\blambda)$ is therefore equivalent to maximizing: 
$$
\psi(\blambda) \equiv \blambda^\top \bar{\bell} - \bar{a}(\blambda), 
$$
with:
$$
\bar{\bell} \equiv \frac{1}{N} \sum_k \bell(x_k,\y_k),
\qquad
\bar{a}(\blambda) \equiv \frac{1}{N} \sum_k a(\blambda,\y_k).
$$

Note that $\psi(\blambda)$ is nothing but the dual function associated with the $I$-projection problem~(\ref{eq:i_proj}). The derivatives of~$\psi(\blambda)$ are found to be:
\begin{eqnarray*}
\nabla\psi(\blambda)
 & = & \bar{\bell} - \frac{1}{N} \sum_k \nabla a(\blambda,\y_k), \\
\nabla\nabla^\top\psi(\blambda)
 & = & - \frac{1}{N} \sum_k \nabla \nabla^\top a(\blambda,\y_k),
\end{eqnarray*}
with:
$$
\nabla a(\blambda,\y) = \E_{\blambda}(\bell),
\qquad
\nabla \nabla^\top a(\blambda,\y) = {\rm Var}_{\blambda}(\bell),
$$
where $\E_{\blambda}$ and ${\rm Var}_{\blambda}$ respectively denote the expectation and variance with respect to $p_\blambda(x|\y)\propto \pi(x)\exp[\blambda^\top \bell(x,\y)]$ at fixed~$\y$. This shows that $a(\blambda,\y)$ is convex in $\blambda$ since the variance matrix is positive, which, in turn, proves the concavity of~$\psi$.



\section{Heterogeneous log-linear pooling}
\label{app:hetero}

Consider a sequence of probability distributions $p_i(x)$, $i=1,\ldots,n$, representing multiple opinions on a variable $x\in{\cal X}$. The general goal of opinion aggregation is to define an operator $F(p_1,\ldots,p_n)$ that takes opinions as inputs and outputs a single distribution $p(x)$ representing a consensus.

Log-linear pooling \cite{Genest-86} is one particular such operator:
$$
p(x)\propto \pi(x) \prod_{i=1}^n p_i(x)^{\lambda_i},
$$
where $\pi(x)$ is an arbitrary distribution and $\lambda_i\in\mathbb{R}$ are arbitrary weights. It is easily seen that, if the weights sum up to one, log-linear pooling is externally Bayesian in the sense that, for any nonvanishing positive valued function $q:{\cal X}\to \mathbb{R}^\star_+$, opinion pooling and opinion updating commute:
$$
F[N(q p_1), \ldots, N(q p_n)]
=
N[q F(p_1,\ldots, p_n)],
$$
where~$N$ denotes the normalization endomorphism of $\mathbb{R}_+^{\cal X}$:
$$
N(f)(x) = \frac{f(x)}{\sum_{x'\in{\cal X}} f(x')}
.
$$

Conversely, \cite{Genest-86b} proved that any externally Bayesian operator for which $F(p)(x)$ is a function of $x,p_1(x),\ldots,p_n(x)$, is of the above form with unit sum weights and is therefore a log-linear opinion pool. 

A question that arises naturally is whether it is possible to weight opinions depending on targets, so that $\lambda_i(x)$ becomes a function of~$x$. This would make it possible, for instance, to consider an agent reliable when expressing an opinion about a certain hypothesis~$x_1$, but unreliable when it comes to another hypothesis~$x_2$. As a first try, let us consider a straightforward extension of log-linear pooling: 
$$
p(x)\propto \prod_i p_i(x)^{\lambda_i(x)},
$$
where $\lambda_i : {\cal X}\to\mathbb{R}$ is some weighting function. Unfortunately, this operator is generally not externally Bayesian because $F[N(q p_1), \ldots, N(q p_n)](x)
\propto r(x) F(p_1,\ldots, p_n)(x)$ with:
$$
r(x) = q(x)^{\sum_i \lambda_i(x)} \prod_i z_i^{-\lambda_i(x)},
\qquad 
z_i = \sum_{x\in{\cal X}} q(x) p_i(x),
$$
and $r(x)$ cannot be made proportional to $q(x)$, except with constant weights, because the normalizing factors~$z_i$ may differ from one agent to the other.

Consider, however, the following alternative:
\begin{equation}
\label{eq:hetero_log_pool}
p(x) \propto \prod_i \left[\frac{p_i(x)}{p_i(x_0)}\right]^{\lambda_i(x)},    
\end{equation}
where $x_0$ is some arbitrary reference value. This does the trick because, for any positive function $q(x)$, the ratio $q(x)p_i(x)/q(x_0)p_i(x_0)$ is left unchanged by normalization of $q p_i$ since the normalizing factors are the same at both the numerator and the numerator, hence they cancel out, yielding:
$$
F[N(q p_1), \ldots, N(q p_n)](x)
\propto
\left[\frac{q(x)}{q(x_0)}\right]^{\sum_i \lambda_i(x)}
F(p_1, \ldots, p_n)(x),
$$
which shows that~(\ref{eq:hetero_log_pool}) is externally Bayesian whenever the weighting function verifies:
$$
\forall x \in {\cal X}\setminus \{x_0\},
\qquad
\sum_i \lambda_i(x) = 1.
$$

The operator~(\ref{eq:hetero_log_pool}) clearly generalizes standard log-linear pooling and has the property that the consensus evaluated at~$x$, $F(p)(x)$, is a function of $x,p_1(x),\ldots,p_n(x)$ but also $p_1(x_0),\ldots,p_n(x_0)$. The ``trick'' to enable heterogeneous (target-dependent) weights is essentially to combine the odds relative to a fixed reference.

%Let us stress that this result does not contradict the above mentionned characterization theorem \cite{Genest-86b}, which restricts itself to operators for which the consensus evaluated at~$x$, $F(p)(x)$, is a function of $x,p_1(x),\ldots,p_n(x)$ while, in the heterogeneous log-linear pool, $F(p)(x)$ is a function of $x,p_1(x),\ldots,p_n(x)$ but also $p_1(x_0),\ldots,p_n(x_0)$. Note that \cite{Genest-86b} also provides the general characterization of externally Bayesian operators, of which the heterogeneous log-linear pool is just a special case slightly more general than the log-linear pool. 
 
%It may seem at first sight that this result contradicts the characterization theorem \cite{Genest-86b} since~(\ref{eq:hetero_log_pool}) is clearly more general than log-linear pooling. There is no contradiction because the said theorem restricts itself to operators for which the consensus evaluated at~$x$, $F(p)(x)$, is a function of $x,p_1(x),\ldots,p_n(x)$ while, in the heterogeneous log-linear pool, $F(p)(x)$ is a function of $x,p_1(x),\ldots,p_n(x)$ but also $p_1(x_0),\ldots,p_n(x_0)$.



\section{Log-likelihood ratio}

$$
\log r = \log \frac{N(x;\mu_1,\sigma_1)}{N(x;\mu_0,\sigma_0)}
$$

$$
\log N(x;\mu,\sigma) = -\frac{1}{2}\left[
\log(2\pi\sigma^2) + \frac{(x-\mu)^2}{\sigma^2}
\right]
$$

$$
\log r = -\frac{1}{2}
\left[
\log \frac{\sigma_1^2}{\sigma_0^2}
+ \frac{(x-\mu_1)^2}{\sigma_1^2} - \frac{(x-\mu_0)^2}{\sigma_0^2}
\right]
$$

It follows:
\begin{eqnarray*}
\log r 
 & = & 
-\frac{1}{2}\left[
\log \frac{\sigma_1^2}{\sigma_0^2}
+ \frac{\sigma_0^2(x-\mu_1)^2 - \sigma_1^2(x-\mu_0)^2}{\sigma_0^2 \sigma_1^2}
\right] \\
 & = & 
\frac{1}{2}\left[
\log \rho^2
+ \frac{(\rho^2-1)x^2 - 2(\rho^2\mu_0-\mu_1)x+ (\rho^2\mu_0 ^2-\mu_1^2)}{\sigma_0^2}
\right]
\end{eqnarray*}
with $\rho=\sigma_1/\sigma_0$. In the case of same variance ($\rho=1$), this simplies to an affine function of $x$ with offset given by the center of $(\mu_0,\mu_1)$:
$$
\log r = \frac{\mu_1-\mu_0}{\sigma^2} \left(
x - \frac{\mu_0 + \mu_1}{2}
\right) 
$$

In any case,
$$
E[\log r | H_1] = 
\frac{1}{2}
\left[
- \log \frac{\sigma_1^2}{\sigma_0^2}
+ \frac{\sigma_1^2 + (\mu_1-\mu_0)^2}{\sigma_0^2}
- 1
\right]
$$

Let $\delta = (\mu_1-\mu_0)/\sigma_0$. We have:
$$
E[\log r | H_1] = 
\frac{1}{2}
\left[
\delta^2
- \log \rho^2
+ \rho^2  - 1
\right]
$$

We note that $\log r$ is a second-order polynomial of~$x$:
$$
f(x) = a x^2 + b x + c,
$$
with $a=(\rho^2-1)/2\sigma_0^2$ and $b=(\mu_1-\rho^2\mu_0)/\sigma_0^2$.

Assuming $X\sim N(\mu_1,\sigma_1)$, and given an observation $x$, we would like to compute the probability:
\begin{eqnarray*}
p 
& = & P[f(X)>f(x)] \\
& = & P\{[f(X)-f(x)]>0\}\\
& = & P\{[a(X^2-x^2)+b(X-x)]>0\}\\
& = & P\{(X-x)[a(X+x)+b]>0)\}\\
& = & P[(X-x)(X-y)>0]\\
& = & 1 - P[X\in(x,y)]\\
& = & 1 - |G(x)-G(y)|
\end{eqnarray*}
where $y=-x-b/a$, and $G=G_{\mu_1,\sigma_1}$ is the Gaussian partition function. In this case, we have:
$$
\frac{b}{a} = 2 \frac{\mu_1-\rho^2\mu_0}{\rho^2-1}
$$

Special case if $\rho=0$, then $a=0$ and $b=\mu_1-\mu_0$, we can't introduce $y$. Going backward, we see that:
$$
p = P[(X-x)b>0]
$$

If $\mu_1>\mu_0$, the answer is $p=1-G(x)$. If $\mu_1<\mu_0$, it is $G(x)$. If $\mu_1=\mu_0$, it is just zero.

So, the story is quite easy to tell. Say that we want to compare two hypotheses $x_0$ and $x_1$, one of which is assumed to hold. It is easy using CL to isolate the features that produce the most significant contribution to the log-CL difference of $x_1$ and $x_0$. Assume that $x_1$ is the one that yields the largest predictive probability, so we are interested in positive contributions (trying to ask: why does the system think $x_1$ is more probable?). 

For any such feature, we may ask if such a large contribution is expected under $x_1$ in order to confirm the prediction. We then compute the above probability $p$. If $p\approx 0$ or $p\approx 1$, the contribution is abnormal, respectively too large or too small. This raises a flag, indicating that $x$ may NOT be distributed according to $x_1$. And, since $x_1$ is more probable than $x_0$, neither hypothesis holds in fact. It could be a distribution intermediate between $x_0$ and $x_1$ (if $p\approx 1$), or farther away from $x_0$ than $x_1$ (if $p\approx 0$).

Note that the $p$ is insensitive to the feature weight in CL... The weights do however play a role in identifying the features relevant to a particular comparison. If a feature is not relevant, it does not matter if 


%\bibliographystyle{ieeetr}
\bibliographystyle{abbrv}
\bibliography{cvis,stat,alexis}

%%\documentclass[english]{scrartcl}

\usepackage{fullpage}
\usepackage{amstext,amssymb,amsmath,amsthm}
\usepackage{graphicx,subfigure}
\usepackage{multirow}
\usepackage{url}
\usepackage{hyperref}
\usepackage{color}

\graphicspath{{.}{pics/}}

%\newtheorem{theorem}{Theorem}[section]
%\newtheorem{proposition}[theorem]{Proposition}


\title{Composite Bayesian inference}
\date{}
\author{Alexis Roche\thanks{\url{alexis.roche@centraliens.net}}}

\def\x{{\mathbf{x}}}
\def\y{{\mathbf{y}}}
\newcommand{\blambda}{{\boldsymbol{\lambda}}}
\newcommand{\bell}{{\boldsymbol{\ell}}}
\newcommand{\E}{\mathbb{E}}
%\newcommand{\matphi}{\boldsymbol{\Phi}} 
%\def\p{{\bar{\mathbf{p}}}}
%\def\q{{\bar{\mathbf{q}}}}



\begin{document}

\maketitle

\begin{abstract}
We revisit the concept of composite likelihood as a method to make a probabilistic inference by aggregation of multiple Bayesian agents. This view makes it natural to use a machine learning approach to calibrate the weights associated with composite likelihood. For instance, they can be tuned in a typical supervised learning stage so as to minimize cross-entropy on a labeled training dataset, yielding a convex problem. We argue that composite Bayesian inference is a middle way between generative and discriminative modeling approaches that trades off between interpretability and prediction performance, both of which are crucial to many artificial intelligence tasks.
\end{abstract}


\section{Introduction}
\label{sec:intro}

Textbook statistical inference (frequentist or Bayesian) rests upon the existence of a probabilistic data-generating model that is both empirically valid and computationally tractable. Because this double requirement may be very challenging for multi-dimensional data, other inference models have been developed in applied science: deliberately misspecified generative models, as in quasi-likelihood \cite{White-82,Walker-13} or na\"ive Bayes \cite{Ng-01} methods; data compression models as in the minimum description length paradigm \cite{Grunwald-07}; and discriminative models\footnote{A {\em discriminative} model is a parametric family that describes the distribution of a target variable conditional on the data, in contrast with a {\em generative} model, which describes the conditional distribution of the data given the target variable.}, which currently dominate the field of artificial intelligence (AI) and typically require supervised learning on large datasets -- these include ``shallow'' learning \cite{Ho-95,BergerA-96,Vapnik-00,Rasmussen-06} and deep learning \cite{Lecun-15,Goodfellow-16} techniques (with the exception of deep belief networks \cite{Hinton-06}).

Discriminative models, however, lack the ability to establish causality links between data instances and predictions, and therefore cannnot ``support'' their own predictions in a built-in manner. Consider as a straightforward example the problem of deciding whether a pan is a sauce pan or a frying pan based on its depth. A discriminative model can learn an optimal threshold that correctly classifies most pans in practice, but it cannot gain an internal representation that sauce pans tend to be deeper than frying pans, which is {\em why} the thresholding works. To interpret a discriminative model means to ``reverse engineer'' it using external knowledge, which quickly becomes an impossible task as the data dimension increases.

There is growing awareness that AI should be interpretable in life-impacting applications~\cite{Molnar-18}, where it is (and perhaps should remain) confined to a supportive role in human-driven decision making processes. For instance, in medical diagnosis, automated disease predictions should always be corroborated by individualized findings to be communicable and thus be taken into consideration by clinicians -- in other words, there is no interest for ``black boxes'' but for AI systems that can justify their opinions just like human experts. Clinicians have long used reference range analysis for diagnosis, which stems from simple univariate generative models.

%, {\em e.g.}, by pinpointing biomarkers that deviate from healthy reference ranges.

% simple methods based on range analysis of univariate biomarkers can provide meaningful clues, and are thus more useful in practice although less powerful in terms of prediction performance.

%With generative models you can compare data with expected outcomes to get insight. 

%the information they provide to support them is limited to the boundaries between classes, which are difficult to interpret for high-dimensional data representations. 

%Decision making is not about pooling opinions, it is about pooling empirical observations and find which hypothesis they point to.

%Need to emphasize why we need generative models. Below are some valid reasons: unsupervised learning, potentially better performance on small training datasets... But the \#1 reason is that discriminative models are black boxes: they provide an answer but no explanation to support it. Truth is, they perform better in supervised big data context in the sense of accuracy, cross-entropy, and so on, no question about that -- but they are mere inference recipes. Is that what AI should be? Sometimes, yes. But there are application fields where this clearly cannot work, for instance medical diagnosis. 

%Imagine a doctor telling a patient: ``Sorry, sir, you have Alzheimer's disease because this fancy AI software said so''. Unless the said software is 100\% reliable, which is practically impossible, this would obviously be unacceptable. Acceptable would be to corroborate the diagnosis with some key quantitative empirical observations: low memory test scores, hippocampal atrophy, etc. In other words, inference has to come with some insight. An AI system is but one expert relying on hypotheses which may be invalid, as any expert, and is therefore prone to error. A typical error source is to train a classifier on a subset of classes from the real world. We cannot avoid inference errors but we can safeguard against them by being able to interrogate the system: why do you think what you think? To achieve that, the system should be capable of {\em introspection.

%Vapnik: ``one should solve the classification problem directly and never solve a more general problem as an intermediate step''. True if the goal is classification only. Wrong if introspection is needed because the problem is then more general in nature.

%Human inference does not work in a pre-determined universe of ``causes''. We confront theories with reality to evaluate how likely they are, while knowing that none of the theories considered so far may be ``right''. In fact, we are not looking for the truth, but for a conving enough theory. 

%Composite likelihood seen as a pool of experts comes with a potential introspection mechanism: first sort experts by decreasing influence on the predictive distribution, see how likely the different classes are to the most influential experts.

Another limitation of discriminative models is that they are not suitable for unsupervised learning or on-the-fly parameter estimation because they treat the data and the model parameters as {\em marginally independent} variables, meaning that the data conveys no information about the parameters unless the target variable is observed. This is illustrated in Figure~\ref{fig:graph_comparison} by the respective directed graphs representing generative and discriminative models. For the same basic reason, supervised learning in a discriminative model is statistically less efficient than in a generative model spanning the same family of posteriors, hence requires a larger training set to achieve optimal performance \cite{Ng-01}. 

%Overall, pure discriminative models are of little use outside the context of big labeled data.

\begin{figure}[!ht]
\begin{center}
\subfigure[Generative model]{\includegraphics[width=.25\textwidth]{generative.pdf}\label{fig:generative}}
\hspace*{.2\textwidth}
\subfigure[Discriminative model]{\includegraphics[width=.25\textwidth]{discriminative.pdf}\label{fig:discriminative}}
\caption{Directed graphs representing generative and discriminative models, where $X$, $Y$ and $\theta$ respectively denote the target variable, the data, and the model parameters. Note the marginal independence of data and parameters in the discriminative model.}
\label{fig:graph_comparison}
\end{center}
\end{figure}

This note advocates the (old) idea of probabilistic opinion pooling \cite{Genest-86} as a means to incorporate generative information into a discriminative model. Rather than attempting at a full generative model, one may combine multiple low-dimensional generative models for different pieces of information extracted from the input data. Each ``small'' model acts as an isolated ``agent'' that uses a single feature to express an opinion in the form of a likelihood function of the target variable. The agent opinions may then be aggregated into a unique predictive probability distribution analogous to a Bayesian posterior. This idea may be understood as a probabilistic version of boosting, or as a ``divide and conquer'' approximation to the intractable Bayesian posterior. Importantly, predictions made in such a way are interpretable owing to their underlying generative nature.

When choosing the aggregation method as a log-linear pool, the predictive distribution turns out to be proportional to a quantity known as composite likelihood (CL) in computational statistics. As detailed in \cite{Varin-11} and the references therein, CL was developed as a surrogate of traditional likelihood for parameter estimation. While maximum CL does not inherit the general property of maximum likelihood to achieve asymptotical minimum variance, it is asymptotically consistent under mild conditions \cite{Xu-11} and generally offers an excellent trade-off between computational and statistical efficiency.

With the opinion pooling interpretation in mind, the feature weights in CL may be optimized for prediction performance in a typical discriminative learning scenario. As we will see, this strategy amounts to training a maximum entropy classifier using the feature-based log-likelihoods as basis functions. These likelihood functions themselves may need to be pre-trained, therefore the composite inference approach typically invovles a double training scheme: a generative training stage (to learn feature-based likelihood parameters) followed by a discriminative training stage (to learn feature weights in aggregation).

Technical details are given in the remainder.



\section{Composite likelihood as opinion pooling}
\label{sec:log_pool}

Let $\mathbf{Y}$ an observable multivariate random variable with sampling distribution $p(\y|x)$ conditional on some unobserved variable of interest, $X\in{\cal X}$, where ${\cal X}$ is assumed to be a finite set for simplicity. Given an experimental outcome $\y$, the likelihood is the sampling distribution evaluated at $\y$, seen as a function of $x$:
$$
L(x) = p(\y|x)
.
$$

This requires a plausible generative model which, for complex data, may be out of reach or involve too many parameters to be estimated. A natural workaround known as data reduction is to extract some lower-dimensional representation $z=f(\y)$, where $f$ is a many-to-one mapping, and consider the potentially more convenient likelihood function:
$$
\ell(x) = p(z|x)
.
$$

Substituting $L(x)$ with $\ell(x)$ boils down to restricting the feature space, thereby  ``delegating'' statistical inference to an ``agent'' provided with partial information. While it is valid for such an agent observing~$z$ only to consider $\ell(x)$ as the likelihood function of the problem, the drawback is that $\ell(x)$ might yield too vague a prediction of~$X$ due to the information loss incurred by data reduction. To make the trick statistically more efficient, we may extract several features, $z_i=f_i(\y)$ for $i=1,2,\ldots,n$, and try to combine the likelihood functions $\ell_i(x) = p(z_i|x)$ that they elicit.

If we see the likelihoods as posterior distributions corresponding to uniform priors, this is a classical problem of probabilistic opinion aggregation from possibly redundant sources, for which several methods exist in the literature \cite{Tarantola-82,Genest-86,Garg-04,Allard-12}. One that is appealing as it tends to produce single peaked distributions is the {\em generalized logarithmic opinion pool}:
\begin{equation}
\label{eq:log_pool}
p_\blambda(x|\y) \propto \pi(x) \prod_{i=1}^n p(z_i|x)^{\lambda_i},
\end{equation} 
where $\pi(x)$ is a prior distribution, and $\blambda=(\lambda_1,\ldots,\lambda_n)$ is a vector of weights. Negative weights should be ruled out to warrant a natural monotonicity property: if all agents agree that an outcome $x_1$ is less likely than an outcome $x_2$, then the consensus probabilities should always satisfy $p_\blambda(x_1|\y)\leq p_\blambda(x_2|\y)$. Also note that it is possible to have the weights depend on the data~$y$, but this would make the method considerably more complex and less interpretable. 


%This condition holds whenever all weights are positive ($\lambda\succeq 0$), assuming that the agent opinions are upper bounded. Negative weights should be ruled out to warrant a natural monotonicity property: if all agents agree that an outcome $x_1$ is less likely than an outcome $x_2$, then the consensus probabilities should always satisfy $p_\lambda(x_1|\y)\leq p_\lambda(x_2|\y)$. Note that it is possible to have the weights depend on the data~$y$, but this would make the method considerably more complex and less interpretable. 


%Strictly positive weights guarantee the so-called 0/1 forcing property, that is, if an hypothesis~$x$ has zero likelihood according to at least one agent, then its consensus probability vanishes too.

% Not clear yet as to what a negative weight could mean!


\section{Composite Bayes rule}
\label{sec:bayes_rule}

The log-linear pool~(\ref{eq:log_pool}) bears a striking similarity to Bayes rule, yielding  the form: 
$$
p_\blambda(x|\y)\propto \pi(x) L^c_\blambda(x),
$$
where the distribution $\pi(x)$ plays the role of a prior, and the function:
\begin{equation}
\label{eq:comp_lik}
L^c_\blambda(x) \equiv \prod_{i=1}^n \ell_i (x)^{\lambda_i}
\end{equation} 
plays that of a likelihood function. The expression~(\ref{eq:comp_lik}) happens to be known in computational statistics as a {\em marginal composite likelihood} \cite{Varin-11}. The slightly more general {\em conditional composite likelihood} form can be derived in the same way as above by conditioning all probabilities on confounding variables, see Appendix~\ref{appendix:conditional}. The computational advantage of CL over a genuine likelihood function stems from the fact that it is more efficient to evaluate the marginal distributions of each feature than the joint distribution of all features.

CL shares a convenient factorized form with the likelihood derived under the assumption of mutual feature independence, usually referred to as {\em na\"ive Bayes} or {\em simple Bayes} in the machine learning literature, which corresponds to the special case of unitary weights, $\blambda\equiv 1$. While the above derivation does not assume feature independence, it yields an alternative interpretation of na\"ive Bayes as a general feature aggregation method. However, unitary CL weights do not guarantee optimal prediction performance.


\section{Prediction interpretation}

A compelling advantage of the composite approach is to make it easy to interpret a prediction for any instance~$\y$. Given two putative target values $x_0$ and $x_1$, the log-probability ratio of $x_1$ vs~$x_0$ is computed by simply adding up single-feature contributions: 
$$
\log \frac{p_\blambda(x_1|\y)}{p_\blambda(x_0|\y)}
= 
\underbrace{\log \frac{\pi(x_1)}{\pi(x_0)}}_{\text{offset}}
+ \sum_i \underbrace{\lambda_i \log \frac{p(z_i|x_1)}{p(z_i|x_0)}}_{\text{feature contribution}},
$$
therefore it is straightforward to assess which features contribute most to a particular hypothesis comparison (either positively or negatively), see possible disagreements between features, and corroborate predictive probabilities with feature-level testable information. 

Each feature contributes to an hypothesis comparison proportionally to its associated log-likelihood ratio test statistic. Contrary to statistical hypothesis testing, likelihood ratios are not converted into $p$-values but are directly combined into a predictive probability ratio. Non-unitary composite weights reinforce or inhibit features compared to na\"ive Bayes prediction ($\blambda\equiv 1$), and may therefore account for mutual feature dependence if tuned appropriately. Keeping the analogy with hypothesis testing, composite weights may be thought of as the counterpart of multiple testing correction \cite{Benjamini-10}.

To interpret the predictive distribution globally, the user(s) may go through successive pairwise hypothesis comparisons in a dynamic way, starting by comparing the automated prediction, {\em i.e.} the most probable target, to some educated guess or reference class (for instance, normality in medical diagnosis). Next, depending on the feature contributions highlighted by the comparison, they may submit another hypothesis for comparison to the automated prediction. This process may be repeated until the composite prediction is subjectively understood by the user(s), yielding a form of human machine dialog.



\section{Composite weight tuning}
\label{sec:tuning}

%So far, we have assumed that the composite weights~$\blambda=(\lambda_1,\ldots,\lambda_n)$ in~(\ref{eq:log_pool}) are known. 

Let us now turn to the problem of tuning the composite weights~$\blambda=(\lambda_1,\ldots,\lambda_n)$ in~(\ref{eq:log_pool}) as part of the method calibration. It is assumed at this stage that the feature generative models are known, which may require a pre-training stage using standard parameter estimation techniques.


\subsection{Previous work}

In the absence of knowledge, a default rule is to choose constant weights. The CL function is then a scaled version of na\"ive Bayes likelihood, the common weight value being irrelevant to the maximum CL estimator (MCLE). To get meaningful predictive probabilities, it may be tuned empirically so as to best adjust the pseudo posterior variance matrix to the asymptotic MCLE variance matrix \cite{Pauli-11}, or via a close-in-spirit curvature adjustment \cite{Ribatet-12}, in attempts to match frequentist and composite Bayesian notions of uncertainty.

Another justifiable recommendation for log-linear pooling weights is to sum up to one. This may be motivated by the fact that the only pooling operator that preserves {\em external Bayesianity} is the log-linear opinion pool with unit sum weights \cite{Genest-86b}. External Bayesianity essentially means that it should not matter whether the prior is incorporated before or after pooling agent opinions, provided that all agents agree on the same prior. A further justification given in \cite{Garg-04} is that the log-linear pool with unit sum weights minimizes the sum of Kullback-Leibler (KL) divergences to the agent opinions (see also \cite{Wang-14}, Theorem~3, for a related optimality argument).

Unit sum weights, however, implicitly assume strong redundancy between features and thus tend to produce over-conservative predictions when features are weakly correlated, as is generally the case in practice since features are supposed to be chosen for complementarity. An ideal method to tune weights is one that can capture and adjust for the pattern of statistical dependences between features. This calls for a training approach, which, as we will see, is very unlikely to pick constant or unit sum weights.


\subsection{Maximum likelihood training}
\label{sec:training}

Given a labeled training dataset ${\cal D}=\{(x_k,\y_k), k=1,\ldots,N\}$, the most direct approach to tune the composite weights is to maximize their likelihood (or, equivalently, minimize their ``cross-entropy'') under the composite predictive distribution:
\begin{equation}
\label{eq:train_likelihood}
\max_{\blambda\succeq 0} U(\blambda),
\qquad \text{with} \quad
U(\blambda) \equiv\sum_{k=1}^N \log p_\blambda(x_k|\y_k).
\end{equation}

From~(\ref{eq:log_pool}), we see that the set of conditional distributions $p_\blambda(x|\y)$ spanned by the weights~$\blambda$ is the exponential family with natural parameter~$\blambda$ and basis functions given by the feature-based log-likelihoods:
$$
p_\blambda(x|\y) = \pi(x) \exp[\blambda^\top \bell(x,\y) - a(\blambda,y)],
$$
with:
$$
\ell_i(x,\y) \equiv \log p(z_i|x),
\qquad
a(\blambda, y) \equiv \log \sum_{x\in{\cal X}} \pi(x) e^{\blambda^\top \bell(x,\y)}.
$$

It follows that the utility function $U(\blambda)$ is concave as a general property of likelihood in exponential families, hence the training can be implemented using a standard convex optimization scheme such as the limited-memory BFGS algorithm \cite{Byrd-95}, see Appendix~\ref{appendix:training} for details. 

Because some positive weight constraints may turn out inactive, the training has a natural tendency to produce sparse weights. As illustrated in Figure~\ref{fig:disc_weight_plot} on the breast cancer UCI dataset\footnote{\url{https://archive.ics.uci.edu/ml/datasets/Breast+Cancer+Wisconsin+(Diagnostic)}}, there is no clear relation between optimal weights and feature-level discrimination power, clearly showing that maximum likelihood training differs from univariate feature selection as it takes feature redundancy into account.

\begin{figure}[!ht]
  \begin{center}
    \includegraphics[width=.6\textwidth]{disc_weight_plot.pdf}
  \end{center}
\caption{Distribution of optimal composite weights vs individual feature discrimination power in a binary classification task (breast cancer UCI dataset). Discrimination is measured by the maximum KL divergence between class-conditional generative distributions.}
\label{fig:disc_weight_plot}
\end{figure}

As is customary and is generally good practice, the training examples may be weighted in~(\ref{eq:train_likelihood}) so that the empirical distribution of classes matches the prior~$\pi(x)$ (enforcing balanced classes if the prior is uniform). Also, some regularization may be useful for stability: we shall in practice maximize $U(\lambda)-\alpha \|\blambda\|^2$, where $\alpha>0$ is a small damping factor. 

Other training variants that we mention here but do not particularly recommend for interpretability include adding basis functions to the exponential family. For instance, class-dependent offsets may be added, hence learning the ``prior'' together with the composite weights. It is also possible to perform unconstrained optimization to achieve higher training likelihood, meaning that some composite weights are then negative.


\subsection{Maximum entropy interpretation}

Exponential family properties also imply that maximum likelihood training~(\ref{eq:train_likelihood}) is dual to a maximum entropy problem or, more precisely, an $I$-projection \cite{Csiszar-84}:
$$
\min_{p\in{\cal P}} D(p\|\pi \cdot h) = \sum_{x\in{\cal X}}\sum_{y\in{\cal Y}} p(x,\y) \log \frac{p(x,\y)}{\pi(x)h(\y)},
$$
subject to the constraints $p(\y)=h(\y)$ and $\E_p[\bell] \succeq \E_h[\bell]$, where 
${\cal P}$ is the set of joint probability distributions on ${\cal X}\times {\cal Y}$
with ${\cal Y}=\{\y_1,\ldots,\y_N\}$ the set of training data values, $h(x,\y)$ is the joint empirical distribution of targets and data, and $h(\y)$ is the corresponding marginal data distribution. Note that the constraints implicitly approximate the data marginal and the mean log-likelihood values by empirical estimates, suggesting the need for regularization on small training sets.

The marginal constraint $p(\y)=h(\y)$ makes the $I$-projection a special case of maximum entropy classifier \cite{BergerA-96}, ensuring that only the conditional distribution $p(x|\y)$ is optimized. The mean log-likelihood constraints $\E_p[\bell] \succeq \E_h[\bell]$ encapsulate dependences between the target and the data, and consider a model admissible if it guarantees the same average log-likelihood levels as observed separately for each feature. This is weaker a constraint than assuming that the feature generative models $p(z_i|x)$ are true: it only assumes that the ``description length'' \cite{Grunwald-07} achieved by each agent is truthworthy.

%In the $I$-projection framework, the composite weights are the Lagrange multipliers associated with the mean log-likelihood inequality constraints.

A particularly insightful analysis given in \cite{Grunwald-04} shows that the $I$-projection is under broad conditions a minimax strategy for prediction according to the log loss. Hence, any $I$-projection is in some sense a best possible predictive model under particular constraints (in the limit of validity of empirical approximations). Our training of composite weights belongs to the same family as multinomial logistic regression, another maximum entropy classifier, only differing by the choice of constraints, which, in the latter case, are not related to generative models.

As an alternative to the strategy considered here, the mean log-likelihood constraints may be expressed for each class rather than being averaged w.r.t.~the prior, leading to a more general predictive model that we call ``super composite likelihood'', in which the weights become class-dependent (see Appendix~\ref{appendix:super}). This variant has the potential for higher prediction performance, but poses some conceptual difficulties, in particular that of having to define an arbitrary reference class. 

%used to predict whether breast lumps are benign or malignant based on a set of features extracted from fine-needle aspirate images.

%The most discriminative features tend to get large weight, however some quite discriminative features may get low weight if strongly correlated with a more discriminative feature

%Non-linear means that features with average discrimination power may get low weight as they are superseded by the most discriminative features. But some features with low discrimination power may also get large weight, which does not mean that they are very influential in practice (see above). This is illustrated in Figure~\ref{fig:disc_weight_plot} with the UCI breast cancer dataset. 

%{\color{red} Equality constraints (negative weights) are less interpretable.}

%{\color{red} Offsets.}

%{\color{red} Data weights.}

%Need to weight the data to adjust to prior as a general measure when minimizing cross-entropy, which should yield a conditional model that approximates the ``true'' conditional distribution, $q(x|y)\approx p(x|y)$. So the model will absorb the ``true'' marginal of $x$, which should thus be replaced by the prior if we want the model to mimic ideal Bayesian inference. In our case, it means that the composite likelihood approximates the true likelihood.

%We can further enforce exact prior matching by including offsets ($K-1$ free parameters), in which case the training phase learns $\pi(x)$ so that it matches the prior $\pi(x)$ in the sense $\int h(\y)p_\lambda(x|y)dy = \pi(x)$. This ``deforms'' the prior to improve the training score. But why is it important to enforce this constraint? Again, it all makes things just less interpretable. 

%When including offsets, using homoscedastic generative variance estimation and equality constraints, composite likelihood is equivalent to binomial logistic regression. However, equality constraints are not easily interpretable.

%In some sense, composite weights replace the multiple testing correction required in the frequentist inference paradigm for proper control of false positives. Here, we don't control false positives, we control prediction accuracy.

%Why not using the feature-based posteriors or Bayes factors rather than the likelihoods? It's a possibility (perhaps a connection here with Gr\"unwald's luckiness). A justification for not doing it is that the coordinator should discard the priors used by the agents. The constraints are not purely evidence-based since the moments depend on the coordinator's prior but it is arguable a mess of priors wouldn't be good.

%Single feature case ($y=z$): the maxent solution has the form $p(x|y)\propto\pi(x)p(\y|x)^\lambda$. We expect to have $\lambda=1$ but it's not necessarily the case. It is if $h(\y)=\int\pi(x)p(\y|x)dx$ because then $h(\y)p(x|y)=\pi(x)p(\y|x)$ so the constraint is verified (and active). This also tells us that the data marginal should be consistent with the prior on labels for this much expected consistency property to hold. So, either weight datapoints so as to match a desired prior (e.g., uniform) or take the empirical distribution of labels as the ``prior''. 

%Which brings another question: in this case, the (unique) weight is insensitive to the prior -- is it true in general? The answer is no. The weights generally depend on the prior. Hence the maxent composite likelihood is prior-dependent. This is an important conceptual difference with the classical notion of likelihood.



\section{Discussion}
\label{sec:discussion}

{\color{red} It is easy to build a good classifier. You just need to pick a parametric family that is big enough to produce a good fit yet small enough to avoid overfitting. Less easy is to interpret what the classifier does. Deep learning is cool but it is relevant to AI only if the representations it learns are generic enough to be used in other problems, i.e. it should ultimately be validated via transfer learning.}

{\color{red} Why prediction performance is still important? Because it validates (or not) interpretation. Interpretation is the goal but prediction performance is a crucial intermediate step.}

{\color{red} We cannot rank features according to their weights. A feature might incidentally get a large weight because its likelihood is flat. Feature relevance also depends on conditional likelihood distribution. Consider mutually independent features: each one recieves a unit weight, but some may be independent from target.}

{\color{red} Semi-generative method. Good for shallow learning or transfer learning. Still need representation learning methods.}

{\color{red} Alternative to logistic regression in deep learning? Would require a game-theoretic learning approach... future work.} 

{\color{red} Composite Bayes is invariant by affine transformation of the data (unlike multinomial LR, which is problematic when damping is applied).}

{\color{red} Is damping important? (it guarantees training cost is stricly convex).}

{\color{red} Can we use the same dataset to learn both the feature generative distributions and the composite weights? In principle, nothing speaks against this.} 

Some experimental observations. In the iris dataset (historical data used by Fisher to demonstrate linear discriminant analysis), composite Bayes finds that petal length is the most relevant feature followed by petal width, while sepal length and width get zero weight. In the wine dataset, the flavanoids is the most relevant feature for standard composite Bayes, but it's color intensity closely followed by flavanoids for the super composite version. In the breast cancer, both approaches are equivalent since it is a binary problem, weights are very sparse: only 13/30 are non-zero with worst area the most relevant. In the digits prolem, the most relevant features tend to be in the image center; composite Bayes can achieve similar performance to multinomial logistic regression with about 10 times less parameters.

Number of independent discriminative parameters in logistic regression: $(K-1)(F+1)$. In standard composite inference: $F$ (or $K-1+F$ with offsets). Difference $=(K-2)(F+1)+1$ (or $=(K-2)F$). In super-composite inference: $(K-1)(F+1)$ (if reference is a class). Number of generative parameters in composite inference using univariate Gaussian models: $2K$ with heteroscedasticity, $K+1$ with homoscedasticity.

Intuition behind weights: notion of global feature relevance. But does $\lambda_1>\lambda_2$ imply that $z_1$ is more relevant than $z_2$ for a particular instance? No. But we can assess the influence of each scaled likelihood in order to rank features according to their influence on the decision.

CL is a concept from computational statistics that has mainly been developed so far in a frequentist perspective as a surrogate for the maximum likelihood method. We have shown a deep connection between CL and probabilistic inference, thereby establishing CL as a class of discriminative models. Because CL is built from a set of marginal feature-specific generative distributions, it is in essence a two-step semi-generative, semi-discriminative learning approach. In the first, ``generative'' phase, the feature distributions are learned; in the second, ``discriminative'' phase, the feature weights are learned. This strategy can be thought of as a form of non-adaptive boosting.

%The first phase corresponds to the training of ``weak learners''. The second phase amounts to a form of boosting. 

A purely discriminative learning could be used instead but...
Why potentially more efficient than pure discriminative training: because generative training is always more efficient than discriminative training if models are comparable. So the key point is that ``we have less parameters in the discriminant phase". Good in small datasets. But also in asymptotic regime if the features are weakly or highly correlated (???).

Logistic regression / naive Bayes example. Under homoscedasticity (assuming that each feature has class-independent variance), CBI is equivalent to logistic regression -- because the predictive distribution family is the same. This is a case where BCI brings nothing. But consider heteroscedasticity, then BCI yields a quadratic classifier. Compared to a fully discriminative model, the number of parameters to learned in the discriminative phase is reduced by half.

The first training phase is easier if supervised but could be unsupervised too (using mixture models). How do we then deal with label switching issues? Can we safely assume conditional feature independence {\em in the generative training phase}? I believe so provided that the marginal distribution parameters are disjoint. It's obvious in the supervised learning scenario.

If unsupervised, the first learning phase could be compared with contrastive pre-training of RBMs \cite{Hinton-06,Fischer-14}, which also optimize parameters for generation of observable features. BCI is comparable with an RBM with a single output unit (which is just a generative model assuming conditionial feature independence). The key difference is that the RBM is a full generative model while BCI only deals with marginal models, hence relying on weaker assumptions. This won't change anything in the pre-training phase but RBMs have to deal with more parameters in the generative learning phase. In fully supervised context, RBM pre-training is pointless since all parameters learned in the first phase will be overwritten. In BCI, pre-training is crucial even in supervised context because the parameters learned in the discriminative phase (the feature weighs) describe a sub-manifold of the predictive distribution family.

Needs features. Not a representation learning method, but could be coupled why not.


%{\color{red} Moreover, while RBM parameters are typically refined in a supervised discriminative learning step, disjoint set of parameters for SCL. Dunno how to say that.} 

%In such context, SCL competes with classical discriminative models (logistic regression, Gaussian processes \cite{Rasmussen-06}, maximum entropy models \cite{BergerA-96}, etc.), and may compare more or less favorably in practice depending on the amount of training data. For relatively small training datasets, we may hope for more accurate inference using~SCL than using traditional discriminative models, extrapolating from the results of \cite{Ng-01} regarding the comparison between logistic regression and na\"ive Bayes classifiers.
%{\color{red}Ici, il manque la comparaison avec les RBMs qui ne sont pas (forcement) des modeles discriminatifs.}

%%, hence alleviating the need for heavily supervised model training

%In summary, CL has the potential to yield weakly supervised or unsupervised Bayesian-like inference procedures depending on the particular task at hand. This property reflects the encoding of statistical relationships between the data and {\em all} unknown parameters. CL thus appears as a trade-off between generative models, which are optimal for unsupervised learning but possibly intractable, and traditional discriminative models (logistic regression, Gaussian processes \cite{Rasmussen-06}, maximum entropy models \cite{BergerA-96}, etc.), which are inherently supervised. CL models are discriminative models assembled from atomic generative models and, from this point of view, may be considered as {\em semi-generative} models.

%CL may be considered as a {\em semi-generative} model: a discriminative model assembled from partial generative models.

%As a note of caution, we shall stress that the pre-determined weights assigned to the different associations between observed and unobserved values represent prior knowledge regarding the informativeness of clues. A poor choice of weights will inevitably result in a poor approximation to the ``true'' Bayesian posterior -- the posterior that would be obtained from a realistic generative model if it was tractable. In future work, we will investigate feature selection strategies to mitigate this problem.

% Improve the discussion on following aspects:
% ** Why is it compatible with unsupervised learning? Give more insight.
% ** Stress the contribution: class-specific weighting.
% * Pivotality argument.
% * Bayes is a special case of composite Bayes.

{\color{red}Product of fucking experts \cite{Hinton-02}. Log-linear pool to build a generative model (assuming in fact independence between experts). Better seen as a ``log-mixture model''. Sounds like each expert is associated with a class, so it's the same as a RBM. Contrastive learning approximates ML parameter estimation. The experts are learning jointly. It's just a generative model.}

{\color{red}Two-step training: generative phase then discriminative phase. Why is it cool?}

{\color{red}CBI is just a way to reweight Naive Bayes. What's the big deal? Can we really expect massively superior performance? Are we just talking about realistic credibility sets?}

Future work: can we collapse generative and discriminative training into a single step? Maybe using the concept of {\em minimally informative likelihood} \cite{Yuan-99b}.


\appendix


\section{Basic frequentist properties of composite likelihood}
\label{appendix:frequentist}

Assume that the data~$\y$ is distributed according to some target value~$x_\star$. For any extracted feature~$z_i$ and for any hypothetical target value~$x$,
$$
0 \leq
E\left[
\log \frac{p(z_i|x_\star)}{p(z_i|x)}
\right]
\leq
E\left[
\log \frac{p(\y|x_\star)}{p(\y|x)}
\right],
$$
where the expectation is taken with respect to the true distribution $p(\y|x_\star)$. Using a weighted sum of such inequalities, we can bracket the expected variations of the logarithm of any composite likelihood function:
\begin{equation}
\label{eq:variation_bound}
0 \leq
E\left[ \log \frac{L_c(x_\star, \blambda)}{L_c(x,\blambda)} \right]
\leq 
\|\blambda\|_1 E\left[ \log \frac{L(x_\star)}{L(x)} \right]
.
\end{equation}

This implies two asymptotic properties of standard composite likelihood: 
\begin{itemize}
\item {\em Consistency.} The expected composite log-likelihood is maximized by~$x_\star$.
\item {\em Conservativeness.} For any~$x$, if the expected true log-likelihood of~$x$ is finite, then the corresponding expected composite log-likelihood ratio is finite too. 
\end{itemize}



\section{Conditional composite likelihood}
\label{appendix:conditional}

As a straightforward  extension of marginal CL, each feature-based likelihood may be conditioned by an additional ``independent'' feature $z^c_i = f^c_i(\y)$ considered as a predictor of the ``dependent'' feature, $z_i=f_i(\y)$, yielding the more general form:
\begin{equation}
\label{eq:cond_feat_lik}
\ell_i(x) = p(z_i|x,z^c_i).
\end{equation}

Conditioning may be useful if it is believed that $z^c_i$ alone is little or not informative about $x$, but can provide relevant information when considered jointly with $x$, as in the case of regression covariates, for instance. Equation~(\ref{eq:comp_lik}) then amounts to conditional CL \cite{Varin-11}, a more general form of CL also including Besag's historical {\em pseudo-likelihood} \cite{Besag-74} developed for image segmentation.


\section{Composite likelihood training}
\label{appendix:training}

Using the same notation as in Section~\ref{sec:training}, the likelihood function in~(\ref{eq:train_likelihood}) can be expanded as follows:
$$
U(\blambda) 
= \sum_{k=1}^N \left[
\log \pi(x_k) + \blambda^\top \bell(x_k, \y_k) - a(\blambda,\y_k)
\right]
$$

This reads: $U(\blambda) = \sum_k \log \pi(x_k) + N \psi(\blambda)$ with: 
$$
\psi(\blambda) \equiv \blambda^\top \bar{\bell} - \frac{1}{N} \sum_k a(\lambda,\y_k), 
$$
and:
$$
\bar{\bell} \equiv \frac{1}{N} \sum_k \bell(x_k,\y_k),
$$
therefore maximizing~$U(\blambda)$ is equivalent to maximizing~$\psi(\blambda)$. 

Derivatives...
$$
\nabla\psi
= \bar{\bell} - \frac{1}{N}\sum_k \nabla a(\blambda,\y_k)
$$

$$
\nabla\nabla^\top\psi
= - \frac{1}{N}\sum_k
\nabla \nabla^\top a(\blambda, \y_k)
$$

We have:
$$
\nabla a = \E_{\blambda}(\bell|\y),
\qquad
\nabla \nabla^\top a = \text{Var}_{\blambda}(\bell|\y)
$$
proving that $a(\blambda,\y)$ is convex in $\blambda$ for any $\y$, and in turnes proves the concavity of $\psi$.

To see this, let:
$$
z(\blambda,\y) 
\equiv e^{a(\lambda,\y)}  
= \sum_{x\in{\cal X}} \pi(x) e^{\blambda^\top \bell(x,\y)}
$$

We have:
$$
\nabla a = \frac{\nabla z}{z}
$$

$$
\nabla\nabla^\top a 
= \left(
\frac{\nabla \nabla^\top z}{z} 
- \frac{\nabla z \nabla z^\top}{z^2}
\right)
$$

$$
\nabla z 
=
\sum_{x\in{\cal X}} \pi(x) \bell(x,\y)  e^{\blambda^\top \bell(x,\y)}
$$

$$
\nabla \nabla^\top z 
=
\sum_{x\in{\cal X}} \pi(x) \bell(x,\y) \bell(x,\y)^\top e^{\blambda^\top \bell(x,\y)}
$$


\section{Super composite likelihood}
\label{appendix:super}

When chosen for computational simplicity, features may not only convey limited information at individual level, their informativeness may also be very much class-dependent. Consider, for instance, diagnosing a disease from a routine medical checkup. Body temperature may point to a bacterial infection by comparison with normality, but would not help detecting a non-infectious cardiovasc ular disease -- and conversely for, say, blood pressure. 

This motivates a more general pooling than the common log-linear pool~(\ref{eq:log_pool}) where features can be weighted depending on the target variable:
$$
p_\lambda(x|\y) \propto \pi(x) \prod_i \left[\frac{p(z_i|x)}{p_0(z_i)}\right]^{\lambda_i(x)},
$$
where $p_0$ is an arbitrary reference feature distribution (which is implicit if weights are class-independent as it is then effectless). Make sense to have a reference as it gives a customizable baseline for detection. Can be the feature distribution averaged over all classes (not recommended I would say) or the distribution of a particular class if that makes sense (e.g. healthy population). In this case, the super composite likelihood is equivalent to the standard one in the two-class case.

We can train this model by maximum likelihood, yielding a similar optimization problem as in Section~\ref{sec:training}. 

%If some classes are merged or split, then the other classes keep the same weights (assuming $h(\y)$ is kept constant, so we need to make sure the prior is adapted accordingly)! So if a patient was found to have 10 times more probability to be MCI than healthy (and healthy is the reference class), and we later split AD population into, say, pure AD and mixed AD, the MCI probably ratio fact will remain.

Super composite likelihood is essentially about combining binary versions of composite likelihood. It is analogous to the ``one-versus-rest'' strategy in multiple logistic regression. 

Link with PDF projection \cite{Baggenstoss-03}.

Advantages: richer model (so better predictions), probabilty ratios stable to class addition. Drawbacks: require a reference, no more consistency. Still interpretable, that's why it is worthwile mentioning.

%I was thinking about the unit sum weight version... The diagnosis example could be one in which clues are independent, in which case $\lambda_i(x)\equiv 1$ would be optimal, making any super composite trick irrelevant. Also see that with hypothesis-dependent weights, we may lose the estimation consistency. At the end of the day, the interest of the super composite idea is questionable.



\section{Log-likelihood ratio}

$$
\log r = \log \frac{N(x;\mu_1,\sigma_1)}{N(x;\mu_0,\sigma_0)}
$$

$$
\log N(x;\mu,\sigma) = -\frac{1}{2}\left[
\log(2\pi\sigma^2) + \frac{(x-\mu)^2}{\sigma^2}
\right]
$$

$$
\log r = -\frac{1}{2}
\left[
\log \frac{\sigma_1^2}{\sigma_0^2}
+ \frac{(x-\mu_1)^2}{\sigma_1^2} - \frac{(x-\mu_0)^2}{\sigma_0^2}
\right]
$$

It follows:
\begin{eqnarray*}
\log r 
 & = & 
-\frac{1}{2}\left[
\log \frac{\sigma_1^2}{\sigma_0^2}
+ \frac{\sigma_0^2(x-\mu_1)^2 - \sigma_1^2(x-\mu_0)^2}{\sigma_0^2 \sigma_1^2}
\right] \\
 & = & 
\frac{1}{2}\left[
\log \rho^2
+ \frac{(\rho^2-1)x^2 - 2(\rho^2\mu_0-\mu_1)x+ (\rho^2\mu_0 ^2-\mu_1^2)}{\sigma_0^2}
\right]
\end{eqnarray*}
with $\rho=\sigma_1/\sigma_0$. In the case of same variance ($\rho=1$), this simplies to an affine function of $x$ with offset given by the center of $(\mu_0,\mu_1)$:
$$
\log r = \frac{\mu_1-\mu_0}{\sigma^2} \left(
x - \frac{\mu_0 + \mu_1}{2}
\right) 
$$

In any case,
$$
E[\log r | H_1] = 
\frac{1}{2}
\left[
- \log \frac{\sigma_1^2}{\sigma_0^2}
+ \frac{\sigma_1^2 + (\mu_1-\mu_0)^2}{\sigma_0^2}
- 1
\right]
$$

Let $\delta = (\mu_1-\mu_0)/\sigma_0$. We have:
$$
E[\log r | H_1] = 
\frac{1}{2}
\left[
\delta^2
- \log \rho^2
+ \rho^2  - 1
\right]
$$




\bibliographystyle{abbrv}
\bibliography{cvis,stat,alexis}

%%\documentclass[english]{scrartcl}

\usepackage{fullpage}
\usepackage{amstext,amssymb,amsmath,amsthm}
\usepackage{graphicx,subfigure}
\usepackage{multirow}
\usepackage{url}
\usepackage{hyperref}
\usepackage{color}

\graphicspath{{.}{pics/}}

%\newtheorem{theorem}{Theorem}[section]
%\newtheorem{proposition}[theorem]{Proposition}


\title{Composite Bayesian inference}
\date{}
\author{Alexis Roche\thanks{\url{alexis.roche@centraliens.net}}}

\def\x{{\mathbf{x}}}
\def\y{{\mathbf{y}}}
\newcommand{\blambda}{{\boldsymbol{\lambda}}}
\newcommand{\bell}{{\boldsymbol{\ell}}}
\newcommand{\E}{\mathbb{E}}
%\newcommand{\matphi}{\boldsymbol{\Phi}} 
%\def\p{{\bar{\mathbf{p}}}}
%\def\q{{\bar{\mathbf{q}}}}



\begin{document}

\maketitle

\begin{abstract}
We revisit the concept of composite likelihood as a method to make a probabilistic inference by aggregation of multiple Bayesian agents. This view makes it natural to use a machine learning approach to calibrate the weights associated with composite likelihood. For instance, they can be tuned in a typical supervised learning stage so as to minimize cross-entropy on a labeled training dataset, yielding a convex problem. We argue that composite Bayesian inference is a middle way between generative and discriminative modeling approaches that trades off between interpretability and prediction performance, both of which are crucial to many artificial intelligence tasks.
\end{abstract}


\section{Introduction}
\label{sec:intro}

Textbook statistical inference (frequentist or Bayesian) rests upon the existence of a probabilistic data-generating model that is both empirically valid and computationally tractable. Because this double requirement may be very challenging for multi-dimensional data, other inference models have been developed in applied science: deliberately misspecified generative models, as in quasi-likelihood \cite{White-82,Walker-13} or na\"ive Bayes \cite{Ng-01} methods; data compression models as in the minimum description length paradigm \cite{Grunwald-07}; and discriminative models\footnote{A {\em discriminative} model is a parametric family that describes the distribution of a target variable conditional on the data, in contrast with a {\em generative} model, which describes the conditional distribution of the data given the target variable.}, which currently dominate the field of artificial intelligence (AI) and typically require supervised learning on large datasets -- these include ``shallow'' learning \cite{Ho-95,BergerA-96,Vapnik-00,Rasmussen-06} and deep learning \cite{Lecun-15,Goodfellow-16} techniques (with the exception of deep belief networks \cite{Hinton-06}).

Discriminative models, however, lack the ability to establish causality links between data instances and predictions, and therefore cannnot ``support'' their own predictions in a built-in manner. Consider as a straightforward example the problem of deciding whether a pan is a sauce pan or a frying pan based on its depth. A discriminative model can learn an optimal threshold that correctly classifies most pans in practice, but it cannot gain an internal representation that sauce pans tend to be deeper than frying pans, which is {\em why} the thresholding works. To interpret a discriminative model means to ``reverse engineer'' it using external knowledge, which quickly becomes an impossible task as the data dimension increases.

There is growing awareness that AI should be interpretable in life-impacting applications~\cite{Molnar-18}, where it is (and perhaps should remain) confined to a supportive role in human-driven decision making processes. For instance, in medical diagnosis, automated disease predictions should always be corroborated by individualized findings to be communicable and thus be taken into consideration by clinicians -- in other words, there is no interest for ``black boxes'' but for AI systems that can justify their opinions just like human experts. Clinicians have long used reference range analysis for diagnosis, which stems from simple univariate generative models.

%, {\em e.g.}, by pinpointing biomarkers that deviate from healthy reference ranges.

% simple methods based on range analysis of univariate biomarkers can provide meaningful clues, and are thus more useful in practice although less powerful in terms of prediction performance.

%With generative models you can compare data with expected outcomes to get insight. 

%the information they provide to support them is limited to the boundaries between classes, which are difficult to interpret for high-dimensional data representations. 

%Decision making is not about pooling opinions, it is about pooling empirical observations and find which hypothesis they point to.

%Need to emphasize why we need generative models. Below are some valid reasons: unsupervised learning, potentially better performance on small training datasets... But the \#1 reason is that discriminative models are black boxes: they provide an answer but no explanation to support it. Truth is, they perform better in supervised big data context in the sense of accuracy, cross-entropy, and so on, no question about that -- but they are mere inference recipes. Is that what AI should be? Sometimes, yes. But there are application fields where this clearly cannot work, for instance medical diagnosis. 

%Imagine a doctor telling a patient: ``Sorry, sir, you have Alzheimer's disease because this fancy AI software said so''. Unless the said software is 100\% reliable, which is practically impossible, this would obviously be unacceptable. Acceptable would be to corroborate the diagnosis with some key quantitative empirical observations: low memory test scores, hippocampal atrophy, etc. In other words, inference has to come with some insight. An AI system is but one expert relying on hypotheses which may be invalid, as any expert, and is therefore prone to error. A typical error source is to train a classifier on a subset of classes from the real world. We cannot avoid inference errors but we can safeguard against them by being able to interrogate the system: why do you think what you think? To achieve that, the system should be capable of {\em introspection.

%Vapnik: ``one should solve the classification problem directly and never solve a more general problem as an intermediate step''. True if the goal is classification only. Wrong if introspection is needed because the problem is then more general in nature.

%Human inference does not work in a pre-determined universe of ``causes''. We confront theories with reality to evaluate how likely they are, while knowing that none of the theories considered so far may be ``right''. In fact, we are not looking for the truth, but for a conving enough theory. 

%Composite likelihood seen as a pool of experts comes with a potential introspection mechanism: first sort experts by decreasing influence on the predictive distribution, see how likely the different classes are to the most influential experts.

Another limitation of discriminative models is that they are not suitable for unsupervised learning or on-the-fly parameter estimation because they treat the data and the model parameters as {\em marginally independent} variables, meaning that the data conveys no information about the parameters unless the target variable is observed. This is illustrated in Figure~\ref{fig:graph_comparison} by the respective directed graphs representing generative and discriminative models. For the same basic reason, supervised learning in a discriminative model is statistically less efficient than in a generative model spanning the same family of posteriors, hence requires a larger training set to achieve optimal performance \cite{Ng-01}. 

%Overall, pure discriminative models are of little use outside the context of big labeled data.

\begin{figure}[!ht]
\begin{center}
\subfigure[Generative model]{\includegraphics[width=.25\textwidth]{generative.pdf}\label{fig:generative}}
\hspace*{.2\textwidth}
\subfigure[Discriminative model]{\includegraphics[width=.25\textwidth]{discriminative.pdf}\label{fig:discriminative}}
\caption{Directed graphs representing generative and discriminative models, where $X$, $Y$ and $\theta$ respectively denote the target variable, the data, and the model parameters. Note the marginal independence of data and parameters in the discriminative model.}
\label{fig:graph_comparison}
\end{center}
\end{figure}

This note advocates the (old) idea of probabilistic opinion pooling \cite{Genest-86} as a means to incorporate generative information into a discriminative model. Rather than attempting at a full generative model, one may combine multiple low-dimensional generative models for different pieces of information extracted from the input data. Each ``small'' model acts as an isolated ``agent'' that uses a single feature to express an opinion in the form of a likelihood function of the target variable. The agent opinions may then be aggregated into a unique predictive probability distribution analogous to a Bayesian posterior. This idea may be understood as a probabilistic version of boosting, or as a ``divide and conquer'' approximation to the intractable Bayesian posterior. Importantly, predictions made in such a way are interpretable owing to their underlying generative nature.

When choosing the aggregation method as a log-linear pool, the predictive distribution turns out to be proportional to a quantity known as composite likelihood (CL) in computational statistics. As detailed in \cite{Varin-11} and the references therein, CL was developed as a surrogate of traditional likelihood for parameter estimation. While maximum CL does not inherit the general property of maximum likelihood to achieve asymptotical minimum variance, it is asymptotically consistent under mild conditions \cite{Xu-11} and generally offers an excellent trade-off between computational and statistical efficiency.

With the opinion pooling interpretation in mind, the feature weights in CL may be optimized for prediction performance in a typical discriminative learning scenario. As we will see, this strategy amounts to training a maximum entropy classifier using the feature-based log-likelihoods as basis functions. These likelihood functions themselves may need to be pre-trained, therefore the composite inference approach typically invovles a double training scheme: a generative training stage (to learn feature-based likelihood parameters) followed by a discriminative training stage (to learn feature weights in aggregation).

Technical details are given in the remainder.



\section{Composite likelihood as opinion pooling}
\label{sec:log_pool}

Let $\mathbf{Y}$ an observable multivariate random variable with sampling distribution $p(\y|x)$ conditional on some unobserved variable of interest, $X\in{\cal X}$, where ${\cal X}$ is assumed to be a finite set for simplicity. Given an experimental outcome $\y$, the likelihood is the sampling distribution evaluated at $\y$, seen as a function of $x$:
$$
L(x) = p(\y|x)
.
$$

This requires a plausible generative model which, for complex data, may be out of reach or involve too many parameters to be estimated. A natural workaround known as data reduction is to extract some lower-dimensional representation $z=f(\y)$, where $f$ is a many-to-one mapping, and consider the potentially more convenient likelihood function:
$$
\ell(x) = p(z|x)
.
$$

Substituting $L(x)$ with $\ell(x)$ boils down to restricting the feature space, thereby  ``delegating'' statistical inference to an ``agent'' provided with partial information. While it is valid for such an agent observing~$z$ only to consider $\ell(x)$ as the likelihood function of the problem, the drawback is that $\ell(x)$ might yield too vague a prediction of~$X$ due to the information loss incurred by data reduction. To make the trick statistically more efficient, we may extract several features, $z_i=f_i(\y)$ for $i=1,2,\ldots,n$, and try to combine the likelihood functions $\ell_i(x) = p(z_i|x)$ that they elicit.

If we see the likelihoods as posterior distributions corresponding to uniform priors, this is a classical problem of probabilistic opinion aggregation from possibly redundant sources, for which several methods exist in the literature \cite{Tarantola-82,Genest-86,Garg-04,Allard-12}. One that is appealing as it tends to produce single peaked distributions is the {\em generalized logarithmic opinion pool}:
\begin{equation}
\label{eq:log_pool}
p_\blambda(x|\y) \propto \pi(x) \prod_{i=1}^n p(z_i|x)^{\lambda_i},
\end{equation} 
where $\pi(x)$ is a prior distribution, and $\blambda=(\lambda_1,\ldots,\lambda_n)$ is a vector of weights. Negative weights should be ruled out to warrant a natural monotonicity property: if all agents agree that an outcome $x_1$ is less likely than an outcome $x_2$, then the consensus probabilities should always satisfy $p_\blambda(x_1|\y)\leq p_\blambda(x_2|\y)$. Also note that it is possible to have the weights depend on the data~$y$, but this would make the method considerably more complex and less interpretable. 


%This condition holds whenever all weights are positive ($\lambda\succeq 0$), assuming that the agent opinions are upper bounded. Negative weights should be ruled out to warrant a natural monotonicity property: if all agents agree that an outcome $x_1$ is less likely than an outcome $x_2$, then the consensus probabilities should always satisfy $p_\lambda(x_1|\y)\leq p_\lambda(x_2|\y)$. Note that it is possible to have the weights depend on the data~$y$, but this would make the method considerably more complex and less interpretable. 


%Strictly positive weights guarantee the so-called 0/1 forcing property, that is, if an hypothesis~$x$ has zero likelihood according to at least one agent, then its consensus probability vanishes too.

% Not clear yet as to what a negative weight could mean!


\section{Composite Bayes rule}
\label{sec:bayes_rule}

The log-linear pool~(\ref{eq:log_pool}) bears a striking similarity to Bayes rule, yielding  the form: 
$$
p_\blambda(x|\y)\propto \pi(x) L^c_\blambda(x),
$$
where the distribution $\pi(x)$ plays the role of a prior, and the function:
\begin{equation}
\label{eq:comp_lik}
L^c_\blambda(x) \equiv \prod_{i=1}^n \ell_i (x)^{\lambda_i}
\end{equation} 
plays that of a likelihood function. The expression~(\ref{eq:comp_lik}) happens to be known in computational statistics as a {\em marginal composite likelihood} \cite{Varin-11}. The slightly more general {\em conditional composite likelihood} form can be derived in the same way as above by conditioning all probabilities on confounding variables, see Appendix~\ref{appendix:conditional}. The computational advantage of CL over a genuine likelihood function stems from the fact that it is more efficient to evaluate the marginal distributions of each feature than the joint distribution of all features.

CL shares a convenient factorized form with the likelihood derived under the assumption of mutual feature independence, usually referred to as {\em na\"ive Bayes} or {\em simple Bayes} in the machine learning literature, which corresponds to the special case of unitary weights, $\blambda\equiv 1$. While the above derivation does not assume feature independence, it yields an alternative interpretation of na\"ive Bayes as a general feature aggregation method. However, unitary CL weights do not guarantee optimal prediction performance.


\section{Prediction interpretation}

A compelling advantage of the composite approach is to make it easy to interpret a prediction for any instance~$\y$. Given two putative target values $x_0$ and $x_1$, the log-probability ratio of $x_1$ vs~$x_0$ is computed by simply adding up single-feature contributions: 
$$
\log \frac{p_\blambda(x_1|\y)}{p_\blambda(x_0|\y)}
= 
\underbrace{\log \frac{\pi(x_1)}{\pi(x_0)}}_{\text{offset}}
+ \sum_i \underbrace{\lambda_i \log \frac{p(z_i|x_1)}{p(z_i|x_0)}}_{\text{feature contribution}},
$$
therefore it is straightforward to assess which features contribute most to a particular hypothesis comparison (either positively or negatively), see possible disagreements between features, and corroborate predictive probabilities with feature-level testable information. 

Each feature contributes to an hypothesis comparison proportionally to its associated log-likelihood ratio test statistic. Contrary to statistical hypothesis testing, likelihood ratios are not converted into $p$-values but are directly combined into a predictive probability ratio. Non-unitary composite weights reinforce or inhibit features compared to na\"ive Bayes prediction ($\blambda\equiv 1$), and may therefore account for mutual feature dependence if tuned appropriately. Keeping the analogy with hypothesis testing, composite weights may be thought of as the counterpart of multiple testing correction \cite{Benjamini-10}.

To interpret the predictive distribution globally, the user(s) may go through successive pairwise hypothesis comparisons in a dynamic way, starting by comparing the automated prediction, {\em i.e.} the most probable target, to some educated guess or reference class (for instance, normality in medical diagnosis). Next, depending on the feature contributions highlighted by the comparison, they may submit another hypothesis for comparison to the automated prediction. This process may be repeated until the composite prediction is subjectively understood by the user(s), yielding a form of human machine dialog.



\section{Composite weight tuning}
\label{sec:tuning}

%So far, we have assumed that the composite weights~$\blambda=(\lambda_1,\ldots,\lambda_n)$ in~(\ref{eq:log_pool}) are known. 

Let us now turn to the problem of tuning the composite weights~$\blambda=(\lambda_1,\ldots,\lambda_n)$ in~(\ref{eq:log_pool}) as part of the method calibration. It is assumed at this stage that the feature generative models are known, which may require a pre-training stage using standard parameter estimation techniques.


\subsection{Previous work}

In the absence of knowledge, a default rule is to choose constant weights. The CL function is then a scaled version of na\"ive Bayes likelihood, the common weight value being irrelevant to the maximum CL estimator (MCLE). To get meaningful predictive probabilities, it may be tuned empirically so as to best adjust the pseudo posterior variance matrix to the asymptotic MCLE variance matrix \cite{Pauli-11}, or via a close-in-spirit curvature adjustment \cite{Ribatet-12}, in attempts to match frequentist and composite Bayesian notions of uncertainty.

Another justifiable recommendation for log-linear pooling weights is to sum up to one. This may be motivated by the fact that the only pooling operator that preserves {\em external Bayesianity} is the log-linear opinion pool with unit sum weights \cite{Genest-86b}. External Bayesianity essentially means that it should not matter whether the prior is incorporated before or after pooling agent opinions, provided that all agents agree on the same prior. A further justification given in \cite{Garg-04} is that the log-linear pool with unit sum weights minimizes the sum of Kullback-Leibler (KL) divergences to the agent opinions (see also \cite{Wang-14}, Theorem~3, for a related optimality argument).

Unit sum weights, however, implicitly assume strong redundancy between features and thus tend to produce over-conservative predictions when features are weakly correlated, as is generally the case in practice since features are supposed to be chosen for complementarity. An ideal method to tune weights is one that can capture and adjust for the pattern of statistical dependences between features. This calls for a training approach, which, as we will see, is very unlikely to pick constant or unit sum weights.


\subsection{Maximum likelihood training}
\label{sec:training}

Given a labeled training dataset ${\cal D}=\{(x_k,\y_k), k=1,\ldots,N\}$, the most direct approach to tune the composite weights is to maximize their likelihood (or, equivalently, minimize their ``cross-entropy'') under the composite predictive distribution:
\begin{equation}
\label{eq:train_likelihood}
\max_{\blambda\succeq 0} U(\blambda),
\qquad \text{with} \quad
U(\blambda) \equiv\sum_{k=1}^N \log p_\blambda(x_k|\y_k).
\end{equation}

From~(\ref{eq:log_pool}), we see that the set of conditional distributions $p_\blambda(x|\y)$ spanned by the weights~$\blambda$ is the exponential family with natural parameter~$\blambda$ and basis functions given by the feature-based log-likelihoods:
$$
p_\blambda(x|\y) = \pi(x) \exp[\blambda^\top \bell(x,\y) - a(\blambda,y)],
$$
with:
$$
\ell_i(x,\y) \equiv \log p(z_i|x),
\qquad
a(\blambda, y) \equiv \log \sum_{x\in{\cal X}} \pi(x) e^{\blambda^\top \bell(x,\y)}.
$$

It follows that the utility function $U(\blambda)$ is concave as a general property of likelihood in exponential families, hence the training can be implemented using a standard convex optimization scheme such as the limited-memory BFGS algorithm \cite{Byrd-95}, see Appendix~\ref{appendix:training} for details. 

Because some positive weight constraints may turn out inactive, the training has a natural tendency to produce sparse weights. As illustrated in Figure~\ref{fig:disc_weight_plot} on the breast cancer UCI dataset\footnote{\url{https://archive.ics.uci.edu/ml/datasets/Breast+Cancer+Wisconsin+(Diagnostic)}}, there is no clear relation between optimal weights and feature-level discrimination power, clearly showing that maximum likelihood training differs from univariate feature selection as it takes feature redundancy into account.

\begin{figure}[!ht]
  \begin{center}
    \includegraphics[width=.6\textwidth]{disc_weight_plot.pdf}
  \end{center}
\caption{Distribution of optimal composite weights vs individual feature discrimination power in a binary classification task (breast cancer UCI dataset). Discrimination is measured by the maximum KL divergence between class-conditional generative distributions.}
\label{fig:disc_weight_plot}
\end{figure}

As is customary and is generally good practice, the training examples may be weighted in~(\ref{eq:train_likelihood}) so that the empirical distribution of classes matches the prior~$\pi(x)$ (enforcing balanced classes if the prior is uniform). Also, some regularization may be useful for stability: we shall in practice maximize $U(\lambda)-\alpha \|\blambda\|^2$, where $\alpha>0$ is a small damping factor. 

Other training variants that we mention here but do not particularly recommend for interpretability include adding basis functions to the exponential family. For instance, class-dependent offsets may be added, hence learning the ``prior'' together with the composite weights. It is also possible to perform unconstrained optimization to achieve higher training likelihood, meaning that some composite weights are then negative.


\subsection{Maximum entropy interpretation}

Exponential family properties also imply that maximum likelihood training~(\ref{eq:train_likelihood}) is dual to a maximum entropy problem or, more precisely, an $I$-projection \cite{Csiszar-84}:
$$
\min_{p\in{\cal P}} D(p\|\pi \cdot h) = \sum_{x\in{\cal X}}\sum_{y\in{\cal Y}} p(x,\y) \log \frac{p(x,\y)}{\pi(x)h(\y)},
$$
subject to the constraints $p(\y)=h(\y)$ and $\E_p[\bell] \succeq \E_h[\bell]$, where 
${\cal P}$ is the set of joint probability distributions on ${\cal X}\times {\cal Y}$
with ${\cal Y}=\{\y_1,\ldots,\y_N\}$ the set of training data values, $h(x,\y)$ is the joint empirical distribution of targets and data, and $h(\y)$ is the corresponding marginal data distribution. Note that the constraints implicitly approximate the data marginal and the mean log-likelihood values by empirical estimates, suggesting the need for regularization on small training sets.

The marginal constraint $p(\y)=h(\y)$ makes the $I$-projection a special case of maximum entropy classifier \cite{BergerA-96}, ensuring that only the conditional distribution $p(x|\y)$ is optimized. The mean log-likelihood constraints $\E_p[\bell] \succeq \E_h[\bell]$ encapsulate dependences between the target and the data, and consider a model admissible if it guarantees the same average log-likelihood levels as observed separately for each feature. This is weaker a constraint than assuming that the feature generative models $p(z_i|x)$ are true: it only assumes that the ``description length'' \cite{Grunwald-07} achieved by each agent is truthworthy.

%In the $I$-projection framework, the composite weights are the Lagrange multipliers associated with the mean log-likelihood inequality constraints.

A particularly insightful analysis given in \cite{Grunwald-04} shows that the $I$-projection is under broad conditions a minimax strategy for prediction according to the log loss. Hence, any $I$-projection is in some sense a best possible predictive model under particular constraints (in the limit of validity of empirical approximations). Our training of composite weights belongs to the same family as multinomial logistic regression, another maximum entropy classifier, only differing by the choice of constraints, which, in the latter case, are not related to generative models.

As an alternative to the strategy considered here, the mean log-likelihood constraints may be expressed for each class rather than being averaged w.r.t.~the prior, leading to a more general predictive model that we call ``super composite likelihood'', in which the weights become class-dependent (see Appendix~\ref{appendix:super}). This variant has the potential for higher prediction performance, but poses some conceptual difficulties, in particular that of having to define an arbitrary reference class. 

%used to predict whether breast lumps are benign or malignant based on a set of features extracted from fine-needle aspirate images.

%The most discriminative features tend to get large weight, however some quite discriminative features may get low weight if strongly correlated with a more discriminative feature

%Non-linear means that features with average discrimination power may get low weight as they are superseded by the most discriminative features. But some features with low discrimination power may also get large weight, which does not mean that they are very influential in practice (see above). This is illustrated in Figure~\ref{fig:disc_weight_plot} with the UCI breast cancer dataset. 

%{\color{red} Equality constraints (negative weights) are less interpretable.}

%{\color{red} Offsets.}

%{\color{red} Data weights.}

%Need to weight the data to adjust to prior as a general measure when minimizing cross-entropy, which should yield a conditional model that approximates the ``true'' conditional distribution, $q(x|y)\approx p(x|y)$. So the model will absorb the ``true'' marginal of $x$, which should thus be replaced by the prior if we want the model to mimic ideal Bayesian inference. In our case, it means that the composite likelihood approximates the true likelihood.

%We can further enforce exact prior matching by including offsets ($K-1$ free parameters), in which case the training phase learns $\pi(x)$ so that it matches the prior $\pi(x)$ in the sense $\int h(\y)p_\lambda(x|y)dy = \pi(x)$. This ``deforms'' the prior to improve the training score. But why is it important to enforce this constraint? Again, it all makes things just less interpretable. 

%When including offsets, using homoscedastic generative variance estimation and equality constraints, composite likelihood is equivalent to binomial logistic regression. However, equality constraints are not easily interpretable.

%In some sense, composite weights replace the multiple testing correction required in the frequentist inference paradigm for proper control of false positives. Here, we don't control false positives, we control prediction accuracy.

%Why not using the feature-based posteriors or Bayes factors rather than the likelihoods? It's a possibility (perhaps a connection here with Gr\"unwald's luckiness). A justification for not doing it is that the coordinator should discard the priors used by the agents. The constraints are not purely evidence-based since the moments depend on the coordinator's prior but it is arguable a mess of priors wouldn't be good.

%Single feature case ($y=z$): the maxent solution has the form $p(x|y)\propto\pi(x)p(\y|x)^\lambda$. We expect to have $\lambda=1$ but it's not necessarily the case. It is if $h(\y)=\int\pi(x)p(\y|x)dx$ because then $h(\y)p(x|y)=\pi(x)p(\y|x)$ so the constraint is verified (and active). This also tells us that the data marginal should be consistent with the prior on labels for this much expected consistency property to hold. So, either weight datapoints so as to match a desired prior (e.g., uniform) or take the empirical distribution of labels as the ``prior''. 

%Which brings another question: in this case, the (unique) weight is insensitive to the prior -- is it true in general? The answer is no. The weights generally depend on the prior. Hence the maxent composite likelihood is prior-dependent. This is an important conceptual difference with the classical notion of likelihood.



\section{Discussion}
\label{sec:discussion}

{\color{red} It is easy to build a good classifier. You just need to pick a parametric family that is big enough to produce a good fit yet small enough to avoid overfitting. Less easy is to interpret what the classifier does. Deep learning is cool but it is relevant to AI only if the representations it learns are generic enough to be used in other problems, i.e. it should ultimately be validated via transfer learning.}

{\color{red} Why prediction performance is still important? Because it validates (or not) interpretation. Interpretation is the goal but prediction performance is a crucial intermediate step.}

{\color{red} We cannot rank features according to their weights. A feature might incidentally get a large weight because its likelihood is flat. Feature relevance also depends on conditional likelihood distribution. Consider mutually independent features: each one recieves a unit weight, but some may be independent from target.}

{\color{red} Semi-generative method. Good for shallow learning or transfer learning. Still need representation learning methods.}

{\color{red} Alternative to logistic regression in deep learning? Would require a game-theoretic learning approach... future work.} 

{\color{red} Composite Bayes is invariant by affine transformation of the data (unlike multinomial LR, which is problematic when damping is applied).}

{\color{red} Is damping important? (it guarantees training cost is stricly convex).}

{\color{red} Can we use the same dataset to learn both the feature generative distributions and the composite weights? In principle, nothing speaks against this.} 

Some experimental observations. In the iris dataset (historical data used by Fisher to demonstrate linear discriminant analysis), composite Bayes finds that petal length is the most relevant feature followed by petal width, while sepal length and width get zero weight. In the wine dataset, the flavanoids is the most relevant feature for standard composite Bayes, but it's color intensity closely followed by flavanoids for the super composite version. In the breast cancer, both approaches are equivalent since it is a binary problem, weights are very sparse: only 13/30 are non-zero with worst area the most relevant. In the digits prolem, the most relevant features tend to be in the image center; composite Bayes can achieve similar performance to multinomial logistic regression with about 10 times less parameters.

Number of independent discriminative parameters in logistic regression: $(K-1)(F+1)$. In standard composite inference: $F$ (or $K-1+F$ with offsets). Difference $=(K-2)(F+1)+1$ (or $=(K-2)F$). In super-composite inference: $(K-1)(F+1)$ (if reference is a class). Number of generative parameters in composite inference using univariate Gaussian models: $2K$ with heteroscedasticity, $K+1$ with homoscedasticity.

Intuition behind weights: notion of global feature relevance. But does $\lambda_1>\lambda_2$ imply that $z_1$ is more relevant than $z_2$ for a particular instance? No. But we can assess the influence of each scaled likelihood in order to rank features according to their influence on the decision.

CL is a concept from computational statistics that has mainly been developed so far in a frequentist perspective as a surrogate for the maximum likelihood method. We have shown a deep connection between CL and probabilistic inference, thereby establishing CL as a class of discriminative models. Because CL is built from a set of marginal feature-specific generative distributions, it is in essence a two-step semi-generative, semi-discriminative learning approach. In the first, ``generative'' phase, the feature distributions are learned; in the second, ``discriminative'' phase, the feature weights are learned. This strategy can be thought of as a form of non-adaptive boosting.

%The first phase corresponds to the training of ``weak learners''. The second phase amounts to a form of boosting. 

A purely discriminative learning could be used instead but...
Why potentially more efficient than pure discriminative training: because generative training is always more efficient than discriminative training if models are comparable. So the key point is that ``we have less parameters in the discriminant phase". Good in small datasets. But also in asymptotic regime if the features are weakly or highly correlated (???).

Logistic regression / naive Bayes example. Under homoscedasticity (assuming that each feature has class-independent variance), CBI is equivalent to logistic regression -- because the predictive distribution family is the same. This is a case where BCI brings nothing. But consider heteroscedasticity, then BCI yields a quadratic classifier. Compared to a fully discriminative model, the number of parameters to learned in the discriminative phase is reduced by half.

The first training phase is easier if supervised but could be unsupervised too (using mixture models). How do we then deal with label switching issues? Can we safely assume conditional feature independence {\em in the generative training phase}? I believe so provided that the marginal distribution parameters are disjoint. It's obvious in the supervised learning scenario.

If unsupervised, the first learning phase could be compared with contrastive pre-training of RBMs \cite{Hinton-06,Fischer-14}, which also optimize parameters for generation of observable features. BCI is comparable with an RBM with a single output unit (which is just a generative model assuming conditionial feature independence). The key difference is that the RBM is a full generative model while BCI only deals with marginal models, hence relying on weaker assumptions. This won't change anything in the pre-training phase but RBMs have to deal with more parameters in the generative learning phase. In fully supervised context, RBM pre-training is pointless since all parameters learned in the first phase will be overwritten. In BCI, pre-training is crucial even in supervised context because the parameters learned in the discriminative phase (the feature weighs) describe a sub-manifold of the predictive distribution family.

Needs features. Not a representation learning method, but could be coupled why not.


%{\color{red} Moreover, while RBM parameters are typically refined in a supervised discriminative learning step, disjoint set of parameters for SCL. Dunno how to say that.} 

%In such context, SCL competes with classical discriminative models (logistic regression, Gaussian processes \cite{Rasmussen-06}, maximum entropy models \cite{BergerA-96}, etc.), and may compare more or less favorably in practice depending on the amount of training data. For relatively small training datasets, we may hope for more accurate inference using~SCL than using traditional discriminative models, extrapolating from the results of \cite{Ng-01} regarding the comparison between logistic regression and na\"ive Bayes classifiers.
%{\color{red}Ici, il manque la comparaison avec les RBMs qui ne sont pas (forcement) des modeles discriminatifs.}

%%, hence alleviating the need for heavily supervised model training

%In summary, CL has the potential to yield weakly supervised or unsupervised Bayesian-like inference procedures depending on the particular task at hand. This property reflects the encoding of statistical relationships between the data and {\em all} unknown parameters. CL thus appears as a trade-off between generative models, which are optimal for unsupervised learning but possibly intractable, and traditional discriminative models (logistic regression, Gaussian processes \cite{Rasmussen-06}, maximum entropy models \cite{BergerA-96}, etc.), which are inherently supervised. CL models are discriminative models assembled from atomic generative models and, from this point of view, may be considered as {\em semi-generative} models.

%CL may be considered as a {\em semi-generative} model: a discriminative model assembled from partial generative models.

%As a note of caution, we shall stress that the pre-determined weights assigned to the different associations between observed and unobserved values represent prior knowledge regarding the informativeness of clues. A poor choice of weights will inevitably result in a poor approximation to the ``true'' Bayesian posterior -- the posterior that would be obtained from a realistic generative model if it was tractable. In future work, we will investigate feature selection strategies to mitigate this problem.

% Improve the discussion on following aspects:
% ** Why is it compatible with unsupervised learning? Give more insight.
% ** Stress the contribution: class-specific weighting.
% * Pivotality argument.
% * Bayes is a special case of composite Bayes.

{\color{red}Product of fucking experts \cite{Hinton-02}. Log-linear pool to build a generative model (assuming in fact independence between experts). Better seen as a ``log-mixture model''. Sounds like each expert is associated with a class, so it's the same as a RBM. Contrastive learning approximates ML parameter estimation. The experts are learning jointly. It's just a generative model.}

{\color{red}Two-step training: generative phase then discriminative phase. Why is it cool?}

{\color{red}CBI is just a way to reweight Naive Bayes. What's the big deal? Can we really expect massively superior performance? Are we just talking about realistic credibility sets?}

Future work: can we collapse generative and discriminative training into a single step? Maybe using the concept of {\em minimally informative likelihood} \cite{Yuan-99b}.


\appendix


\section{Basic frequentist properties of composite likelihood}
\label{appendix:frequentist}

Assume that the data~$\y$ is distributed according to some target value~$x_\star$. For any extracted feature~$z_i$ and for any hypothetical target value~$x$,
$$
0 \leq
E\left[
\log \frac{p(z_i|x_\star)}{p(z_i|x)}
\right]
\leq
E\left[
\log \frac{p(\y|x_\star)}{p(\y|x)}
\right],
$$
where the expectation is taken with respect to the true distribution $p(\y|x_\star)$. Using a weighted sum of such inequalities, we can bracket the expected variations of the logarithm of any composite likelihood function:
\begin{equation}
\label{eq:variation_bound}
0 \leq
E\left[ \log \frac{L_c(x_\star, \blambda)}{L_c(x,\blambda)} \right]
\leq 
\|\blambda\|_1 E\left[ \log \frac{L(x_\star)}{L(x)} \right]
.
\end{equation}

This implies two asymptotic properties of standard composite likelihood: 
\begin{itemize}
\item {\em Consistency.} The expected composite log-likelihood is maximized by~$x_\star$.
\item {\em Conservativeness.} For any~$x$, if the expected true log-likelihood of~$x$ is finite, then the corresponding expected composite log-likelihood ratio is finite too. 
\end{itemize}



\section{Conditional composite likelihood}
\label{appendix:conditional}

As a straightforward  extension of marginal CL, each feature-based likelihood may be conditioned by an additional ``independent'' feature $z^c_i = f^c_i(\y)$ considered as a predictor of the ``dependent'' feature, $z_i=f_i(\y)$, yielding the more general form:
\begin{equation}
\label{eq:cond_feat_lik}
\ell_i(x) = p(z_i|x,z^c_i).
\end{equation}

Conditioning may be useful if it is believed that $z^c_i$ alone is little or not informative about $x$, but can provide relevant information when considered jointly with $x$, as in the case of regression covariates, for instance. Equation~(\ref{eq:comp_lik}) then amounts to conditional CL \cite{Varin-11}, a more general form of CL also including Besag's historical {\em pseudo-likelihood} \cite{Besag-74} developed for image segmentation.


\section{Composite likelihood training}
\label{appendix:training}

Using the same notation as in Section~\ref{sec:training}, the likelihood function in~(\ref{eq:train_likelihood}) can be expanded as follows:
$$
U(\blambda) 
= \sum_{k=1}^N \left[
\log \pi(x_k) + \blambda^\top \bell(x_k, \y_k) - a(\blambda,\y_k)
\right]
$$

This reads: $U(\blambda) = \sum_k \log \pi(x_k) + N \psi(\blambda)$ with: 
$$
\psi(\blambda) \equiv \blambda^\top \bar{\bell} - \frac{1}{N} \sum_k a(\lambda,\y_k), 
$$
and:
$$
\bar{\bell} \equiv \frac{1}{N} \sum_k \bell(x_k,\y_k),
$$
therefore maximizing~$U(\blambda)$ is equivalent to maximizing~$\psi(\blambda)$. 

Derivatives...
$$
\nabla\psi
= \bar{\bell} - \frac{1}{N}\sum_k \nabla a(\blambda,\y_k)
$$

$$
\nabla\nabla^\top\psi
= - \frac{1}{N}\sum_k
\nabla \nabla^\top a(\blambda, \y_k)
$$

We have:
$$
\nabla a = \E_{\blambda}(\bell|\y),
\qquad
\nabla \nabla^\top a = \text{Var}_{\blambda}(\bell|\y)
$$
proving that $a(\blambda,\y)$ is convex in $\blambda$ for any $\y$, and in turnes proves the concavity of $\psi$.

To see this, let:
$$
z(\blambda,\y) 
\equiv e^{a(\lambda,\y)}  
= \sum_{x\in{\cal X}} \pi(x) e^{\blambda^\top \bell(x,\y)}
$$

We have:
$$
\nabla a = \frac{\nabla z}{z}
$$

$$
\nabla\nabla^\top a 
= \left(
\frac{\nabla \nabla^\top z}{z} 
- \frac{\nabla z \nabla z^\top}{z^2}
\right)
$$

$$
\nabla z 
=
\sum_{x\in{\cal X}} \pi(x) \bell(x,\y)  e^{\blambda^\top \bell(x,\y)}
$$

$$
\nabla \nabla^\top z 
=
\sum_{x\in{\cal X}} \pi(x) \bell(x,\y) \bell(x,\y)^\top e^{\blambda^\top \bell(x,\y)}
$$


\section{Super composite likelihood}
\label{appendix:super}

When chosen for computational simplicity, features may not only convey limited information at individual level, their informativeness may also be very much class-dependent. Consider, for instance, diagnosing a disease from a routine medical checkup. Body temperature may point to a bacterial infection by comparison with normality, but would not help detecting a non-infectious cardiovasc ular disease -- and conversely for, say, blood pressure. 

This motivates a more general pooling than the common log-linear pool~(\ref{eq:log_pool}) where features can be weighted depending on the target variable:
$$
p_\lambda(x|\y) \propto \pi(x) \prod_i \left[\frac{p(z_i|x)}{p_0(z_i)}\right]^{\lambda_i(x)},
$$
where $p_0$ is an arbitrary reference feature distribution (which is implicit if weights are class-independent as it is then effectless). Make sense to have a reference as it gives a customizable baseline for detection. Can be the feature distribution averaged over all classes (not recommended I would say) or the distribution of a particular class if that makes sense (e.g. healthy population). In this case, the super composite likelihood is equivalent to the standard one in the two-class case.

We can train this model by maximum likelihood, yielding a similar optimization problem as in Section~\ref{sec:training}. 

%If some classes are merged or split, then the other classes keep the same weights (assuming $h(\y)$ is kept constant, so we need to make sure the prior is adapted accordingly)! So if a patient was found to have 10 times more probability to be MCI than healthy (and healthy is the reference class), and we later split AD population into, say, pure AD and mixed AD, the MCI probably ratio fact will remain.

Super composite likelihood is essentially about combining binary versions of composite likelihood. It is analogous to the ``one-versus-rest'' strategy in multiple logistic regression. 

Link with PDF projection \cite{Baggenstoss-03}.

Advantages: richer model (so better predictions), probabilty ratios stable to class addition. Drawbacks: require a reference, no more consistency. Still interpretable, that's why it is worthwile mentioning.

%I was thinking about the unit sum weight version... The diagnosis example could be one in which clues are independent, in which case $\lambda_i(x)\equiv 1$ would be optimal, making any super composite trick irrelevant. Also see that with hypothesis-dependent weights, we may lose the estimation consistency. At the end of the day, the interest of the super composite idea is questionable.



\section{Log-likelihood ratio}

$$
\log r = \log \frac{N(x;\mu_1,\sigma_1)}{N(x;\mu_0,\sigma_0)}
$$

$$
\log N(x;\mu,\sigma) = -\frac{1}{2}\left[
\log(2\pi\sigma^2) + \frac{(x-\mu)^2}{\sigma^2}
\right]
$$

$$
\log r = -\frac{1}{2}
\left[
\log \frac{\sigma_1^2}{\sigma_0^2}
+ \frac{(x-\mu_1)^2}{\sigma_1^2} - \frac{(x-\mu_0)^2}{\sigma_0^2}
\right]
$$

It follows:
\begin{eqnarray*}
\log r 
 & = & 
-\frac{1}{2}\left[
\log \frac{\sigma_1^2}{\sigma_0^2}
+ \frac{\sigma_0^2(x-\mu_1)^2 - \sigma_1^2(x-\mu_0)^2}{\sigma_0^2 \sigma_1^2}
\right] \\
 & = & 
\frac{1}{2}\left[
\log \rho^2
+ \frac{(\rho^2-1)x^2 - 2(\rho^2\mu_0-\mu_1)x+ (\rho^2\mu_0 ^2-\mu_1^2)}{\sigma_0^2}
\right]
\end{eqnarray*}
with $\rho=\sigma_1/\sigma_0$. In the case of same variance ($\rho=1$), this simplies to an affine function of $x$ with offset given by the center of $(\mu_0,\mu_1)$:
$$
\log r = \frac{\mu_1-\mu_0}{\sigma^2} \left(
x - \frac{\mu_0 + \mu_1}{2}
\right) 
$$

In any case,
$$
E[\log r | H_1] = 
\frac{1}{2}
\left[
- \log \frac{\sigma_1^2}{\sigma_0^2}
+ \frac{\sigma_1^2 + (\mu_1-\mu_0)^2}{\sigma_0^2}
- 1
\right]
$$

Let $\delta = (\mu_1-\mu_0)/\sigma_0$. We have:
$$
E[\log r | H_1] = 
\frac{1}{2}
\left[
\delta^2
- \log \rho^2
+ \rho^2  - 1
\right]
$$




\bibliographystyle{abbrv}
\bibliography{cvis,stat,alexis}

%%\documentclass[english]{scrartcl}

\usepackage{fullpage}
\usepackage{amstext,amssymb,amsmath,amsthm}
\usepackage{graphicx,subfigure}
\usepackage{multirow}
\usepackage{url}
\usepackage{hyperref}
\usepackage{color}

\graphicspath{{.}{pics/}}

%\newtheorem{theorem}{Theorem}[section]
%\newtheorem{proposition}[theorem]{Proposition}


\title{Composite Bayesian inference}
\date{}
\author{Alexis Roche\thanks{\url{alexis.roche@centraliens.net}}}

\def\x{{\mathbf{x}}}
\def\y{{\mathbf{y}}}
\newcommand{\blambda}{{\boldsymbol{\lambda}}}
\newcommand{\bell}{{\boldsymbol{\ell}}}
\newcommand{\E}{\mathbb{E}}
%\newcommand{\matphi}{\boldsymbol{\Phi}} 
%\def\p{{\bar{\mathbf{p}}}}
%\def\q{{\bar{\mathbf{q}}}}



\begin{document}

\maketitle

\begin{abstract}
We revisit the concept of composite likelihood as a method to make a probabilistic inference by aggregation of multiple Bayesian agents. This view makes it natural to use a machine learning approach to calibrate the weights associated with composite likelihood. For instance, they can be tuned in a typical supervised learning stage so as to minimize cross-entropy on a labeled training dataset, yielding a convex problem. We argue that composite Bayesian inference is a middle way between generative and discriminative modeling approaches that trades off between interpretability and prediction performance, both of which are crucial to many artificial intelligence tasks.
\end{abstract}


\section{Introduction}
\label{sec:intro}

Textbook statistical inference (frequentist or Bayesian) rests upon the existence of a probabilistic data-generating model that is both empirically valid and computationally tractable. Because this double requirement may be very challenging for multi-dimensional data, other inference models have been developed in applied science: deliberately misspecified generative models, as in quasi-likelihood \cite{White-82,Walker-13} or na\"ive Bayes \cite{Ng-01} methods; data compression models as in the minimum description length paradigm \cite{Grunwald-07}; and discriminative models\footnote{A {\em discriminative} model is a parametric family that describes the distribution of a target variable conditional on the data, in contrast with a {\em generative} model, which describes the conditional distribution of the data given the target variable.}, which currently dominate the field of artificial intelligence (AI) and typically require supervised learning on large datasets -- these include ``shallow'' learning \cite{Ho-95,BergerA-96,Vapnik-00,Rasmussen-06} and deep learning \cite{Lecun-15,Goodfellow-16} techniques (with the exception of deep belief networks \cite{Hinton-06}).

Discriminative models, however, lack the ability to establish causality links between data instances and predictions, and therefore cannnot ``support'' their own predictions in a built-in manner. Consider as a straightforward example the problem of deciding whether a pan is a sauce pan or a frying pan based on its depth. A discriminative model can learn an optimal threshold that correctly classifies most pans in practice, but it cannot gain an internal representation that sauce pans tend to be deeper than frying pans, which is {\em why} the thresholding works. To interpret a discriminative model means to ``reverse engineer'' it using external knowledge, which quickly becomes an impossible task as the data dimension increases.

There is growing awareness that AI should be interpretable in life-impacting applications~\cite{Molnar-18}, where it is (and perhaps should remain) confined to a supportive role in human-driven decision making processes. For instance, in medical diagnosis, automated disease predictions should always be corroborated by individualized findings to be communicable and thus be taken into consideration by clinicians -- in other words, there is no interest for ``black boxes'' but for AI systems that can justify their opinions just like human experts. Clinicians have long used reference range analysis for diagnosis, which stems from simple univariate generative models.

%, {\em e.g.}, by pinpointing biomarkers that deviate from healthy reference ranges.

% simple methods based on range analysis of univariate biomarkers can provide meaningful clues, and are thus more useful in practice although less powerful in terms of prediction performance.

%With generative models you can compare data with expected outcomes to get insight. 

%the information they provide to support them is limited to the boundaries between classes, which are difficult to interpret for high-dimensional data representations. 

%Decision making is not about pooling opinions, it is about pooling empirical observations and find which hypothesis they point to.

%Need to emphasize why we need generative models. Below are some valid reasons: unsupervised learning, potentially better performance on small training datasets... But the \#1 reason is that discriminative models are black boxes: they provide an answer but no explanation to support it. Truth is, they perform better in supervised big data context in the sense of accuracy, cross-entropy, and so on, no question about that -- but they are mere inference recipes. Is that what AI should be? Sometimes, yes. But there are application fields where this clearly cannot work, for instance medical diagnosis. 

%Imagine a doctor telling a patient: ``Sorry, sir, you have Alzheimer's disease because this fancy AI software said so''. Unless the said software is 100\% reliable, which is practically impossible, this would obviously be unacceptable. Acceptable would be to corroborate the diagnosis with some key quantitative empirical observations: low memory test scores, hippocampal atrophy, etc. In other words, inference has to come with some insight. An AI system is but one expert relying on hypotheses which may be invalid, as any expert, and is therefore prone to error. A typical error source is to train a classifier on a subset of classes from the real world. We cannot avoid inference errors but we can safeguard against them by being able to interrogate the system: why do you think what you think? To achieve that, the system should be capable of {\em introspection.

%Vapnik: ``one should solve the classification problem directly and never solve a more general problem as an intermediate step''. True if the goal is classification only. Wrong if introspection is needed because the problem is then more general in nature.

%Human inference does not work in a pre-determined universe of ``causes''. We confront theories with reality to evaluate how likely they are, while knowing that none of the theories considered so far may be ``right''. In fact, we are not looking for the truth, but for a conving enough theory. 

%Composite likelihood seen as a pool of experts comes with a potential introspection mechanism: first sort experts by decreasing influence on the predictive distribution, see how likely the different classes are to the most influential experts.

Another limitation of discriminative models is that they are not suitable for unsupervised learning or on-the-fly parameter estimation because they treat the data and the model parameters as {\em marginally independent} variables, meaning that the data conveys no information about the parameters unless the target variable is observed. This is illustrated in Figure~\ref{fig:graph_comparison} by the respective directed graphs representing generative and discriminative models. For the same basic reason, supervised learning in a discriminative model is statistically less efficient than in a generative model spanning the same family of posteriors, hence requires a larger training set to achieve optimal performance \cite{Ng-01}. 

%Overall, pure discriminative models are of little use outside the context of big labeled data.

\begin{figure}[!ht]
\begin{center}
\subfigure[Generative model]{\includegraphics[width=.25\textwidth]{generative.pdf}\label{fig:generative}}
\hspace*{.2\textwidth}
\subfigure[Discriminative model]{\includegraphics[width=.25\textwidth]{discriminative.pdf}\label{fig:discriminative}}
\caption{Directed graphs representing generative and discriminative models, where $X$, $Y$ and $\theta$ respectively denote the target variable, the data, and the model parameters. Note the marginal independence of data and parameters in the discriminative model.}
\label{fig:graph_comparison}
\end{center}
\end{figure}

This note advocates the (old) idea of probabilistic opinion pooling \cite{Genest-86} as a means to incorporate generative information into a discriminative model. Rather than attempting at a full generative model, one may combine multiple low-dimensional generative models for different pieces of information extracted from the input data. Each ``small'' model acts as an isolated ``agent'' that uses a single feature to express an opinion in the form of a likelihood function of the target variable. The agent opinions may then be aggregated into a unique predictive probability distribution analogous to a Bayesian posterior. This idea may be understood as a probabilistic version of boosting, or as a ``divide and conquer'' approximation to the intractable Bayesian posterior. Importantly, predictions made in such a way are interpretable owing to their underlying generative nature.

When choosing the aggregation method as a log-linear pool, the predictive distribution turns out to be proportional to a quantity known as composite likelihood (CL) in computational statistics. As detailed in \cite{Varin-11} and the references therein, CL was developed as a surrogate of traditional likelihood for parameter estimation. While maximum CL does not inherit the general property of maximum likelihood to achieve asymptotical minimum variance, it is asymptotically consistent under mild conditions \cite{Xu-11} and generally offers an excellent trade-off between computational and statistical efficiency.

With the opinion pooling interpretation in mind, the feature weights in CL may be optimized for prediction performance in a typical discriminative learning scenario. As we will see, this strategy amounts to training a maximum entropy classifier using the feature-based log-likelihoods as basis functions. These likelihood functions themselves may need to be pre-trained, therefore the composite inference approach typically invovles a double training scheme: a generative training stage (to learn feature-based likelihood parameters) followed by a discriminative training stage (to learn feature weights in aggregation).

Technical details are given in the remainder.



\section{Composite likelihood as opinion pooling}
\label{sec:log_pool}

Let $\mathbf{Y}$ an observable multivariate random variable with sampling distribution $p(\y|x)$ conditional on some unobserved variable of interest, $X\in{\cal X}$, where ${\cal X}$ is assumed to be a finite set for simplicity. Given an experimental outcome $\y$, the likelihood is the sampling distribution evaluated at $\y$, seen as a function of $x$:
$$
L(x) = p(\y|x)
.
$$

This requires a plausible generative model which, for complex data, may be out of reach or involve too many parameters to be estimated. A natural workaround known as data reduction is to extract some lower-dimensional representation $z=f(\y)$, where $f$ is a many-to-one mapping, and consider the potentially more convenient likelihood function:
$$
\ell(x) = p(z|x)
.
$$

Substituting $L(x)$ with $\ell(x)$ boils down to restricting the feature space, thereby  ``delegating'' statistical inference to an ``agent'' provided with partial information. While it is valid for such an agent observing~$z$ only to consider $\ell(x)$ as the likelihood function of the problem, the drawback is that $\ell(x)$ might yield too vague a prediction of~$X$ due to the information loss incurred by data reduction. To make the trick statistically more efficient, we may extract several features, $z_i=f_i(\y)$ for $i=1,2,\ldots,n$, and try to combine the likelihood functions $\ell_i(x) = p(z_i|x)$ that they elicit.

If we see the likelihoods as posterior distributions corresponding to uniform priors, this is a classical problem of probabilistic opinion aggregation from possibly redundant sources, for which several methods exist in the literature \cite{Tarantola-82,Genest-86,Garg-04,Allard-12}. One that is appealing as it tends to produce single peaked distributions is the {\em generalized logarithmic opinion pool}:
\begin{equation}
\label{eq:log_pool}
p_\blambda(x|\y) \propto \pi(x) \prod_{i=1}^n p(z_i|x)^{\lambda_i},
\end{equation} 
where $\pi(x)$ is a prior distribution, and $\blambda=(\lambda_1,\ldots,\lambda_n)$ is a vector of weights. Negative weights should be ruled out to warrant a natural monotonicity property: if all agents agree that an outcome $x_1$ is less likely than an outcome $x_2$, then the consensus probabilities should always satisfy $p_\blambda(x_1|\y)\leq p_\blambda(x_2|\y)$. Also note that it is possible to have the weights depend on the data~$y$, but this would make the method considerably more complex and less interpretable. 


%This condition holds whenever all weights are positive ($\lambda\succeq 0$), assuming that the agent opinions are upper bounded. Negative weights should be ruled out to warrant a natural monotonicity property: if all agents agree that an outcome $x_1$ is less likely than an outcome $x_2$, then the consensus probabilities should always satisfy $p_\lambda(x_1|\y)\leq p_\lambda(x_2|\y)$. Note that it is possible to have the weights depend on the data~$y$, but this would make the method considerably more complex and less interpretable. 


%Strictly positive weights guarantee the so-called 0/1 forcing property, that is, if an hypothesis~$x$ has zero likelihood according to at least one agent, then its consensus probability vanishes too.

% Not clear yet as to what a negative weight could mean!


\section{Composite Bayes rule}
\label{sec:bayes_rule}

The log-linear pool~(\ref{eq:log_pool}) bears a striking similarity to Bayes rule, yielding  the form: 
$$
p_\blambda(x|\y)\propto \pi(x) L^c_\blambda(x),
$$
where the distribution $\pi(x)$ plays the role of a prior, and the function:
\begin{equation}
\label{eq:comp_lik}
L^c_\blambda(x) \equiv \prod_{i=1}^n \ell_i (x)^{\lambda_i}
\end{equation} 
plays that of a likelihood function. The expression~(\ref{eq:comp_lik}) happens to be known in computational statistics as a {\em marginal composite likelihood} \cite{Varin-11}. The slightly more general {\em conditional composite likelihood} form can be derived in the same way as above by conditioning all probabilities on confounding variables, see Appendix~\ref{appendix:conditional}. The computational advantage of CL over a genuine likelihood function stems from the fact that it is more efficient to evaluate the marginal distributions of each feature than the joint distribution of all features.

CL shares a convenient factorized form with the likelihood derived under the assumption of mutual feature independence, usually referred to as {\em na\"ive Bayes} or {\em simple Bayes} in the machine learning literature, which corresponds to the special case of unitary weights, $\blambda\equiv 1$. While the above derivation does not assume feature independence, it yields an alternative interpretation of na\"ive Bayes as a general feature aggregation method. However, unitary CL weights do not guarantee optimal prediction performance.


\section{Prediction interpretation}

A compelling advantage of the composite approach is to make it easy to interpret a prediction for any instance~$\y$. Given two putative target values $x_0$ and $x_1$, the log-probability ratio of $x_1$ vs~$x_0$ is computed by simply adding up single-feature contributions: 
$$
\log \frac{p_\blambda(x_1|\y)}{p_\blambda(x_0|\y)}
= 
\underbrace{\log \frac{\pi(x_1)}{\pi(x_0)}}_{\text{offset}}
+ \sum_i \underbrace{\lambda_i \log \frac{p(z_i|x_1)}{p(z_i|x_0)}}_{\text{feature contribution}},
$$
therefore it is straightforward to assess which features contribute most to a particular hypothesis comparison (either positively or negatively), see possible disagreements between features, and corroborate predictive probabilities with feature-level testable information. 

Each feature contributes to an hypothesis comparison proportionally to its associated log-likelihood ratio test statistic. Contrary to statistical hypothesis testing, likelihood ratios are not converted into $p$-values but are directly combined into a predictive probability ratio. Non-unitary composite weights reinforce or inhibit features compared to na\"ive Bayes prediction ($\blambda\equiv 1$), and may therefore account for mutual feature dependence if tuned appropriately. Keeping the analogy with hypothesis testing, composite weights may be thought of as the counterpart of multiple testing correction \cite{Benjamini-10}.

To interpret the predictive distribution globally, the user(s) may go through successive pairwise hypothesis comparisons in a dynamic way, starting by comparing the automated prediction, {\em i.e.} the most probable target, to some educated guess or reference class (for instance, normality in medical diagnosis). Next, depending on the feature contributions highlighted by the comparison, they may submit another hypothesis for comparison to the automated prediction. This process may be repeated until the composite prediction is subjectively understood by the user(s), yielding a form of human machine dialog.



\section{Composite weight tuning}
\label{sec:tuning}

%So far, we have assumed that the composite weights~$\blambda=(\lambda_1,\ldots,\lambda_n)$ in~(\ref{eq:log_pool}) are known. 

Let us now turn to the problem of tuning the composite weights~$\blambda=(\lambda_1,\ldots,\lambda_n)$ in~(\ref{eq:log_pool}) as part of the method calibration. It is assumed at this stage that the feature generative models are known, which may require a pre-training stage using standard parameter estimation techniques.


\subsection{Previous work}

In the absence of knowledge, a default rule is to choose constant weights. The CL function is then a scaled version of na\"ive Bayes likelihood, the common weight value being irrelevant to the maximum CL estimator (MCLE). To get meaningful predictive probabilities, it may be tuned empirically so as to best adjust the pseudo posterior variance matrix to the asymptotic MCLE variance matrix \cite{Pauli-11}, or via a close-in-spirit curvature adjustment \cite{Ribatet-12}, in attempts to match frequentist and composite Bayesian notions of uncertainty.

Another justifiable recommendation for log-linear pooling weights is to sum up to one. This may be motivated by the fact that the only pooling operator that preserves {\em external Bayesianity} is the log-linear opinion pool with unit sum weights \cite{Genest-86b}. External Bayesianity essentially means that it should not matter whether the prior is incorporated before or after pooling agent opinions, provided that all agents agree on the same prior. A further justification given in \cite{Garg-04} is that the log-linear pool with unit sum weights minimizes the sum of Kullback-Leibler (KL) divergences to the agent opinions (see also \cite{Wang-14}, Theorem~3, for a related optimality argument).

Unit sum weights, however, implicitly assume strong redundancy between features and thus tend to produce over-conservative predictions when features are weakly correlated, as is generally the case in practice since features are supposed to be chosen for complementarity. An ideal method to tune weights is one that can capture and adjust for the pattern of statistical dependences between features. This calls for a training approach, which, as we will see, is very unlikely to pick constant or unit sum weights.


\subsection{Maximum likelihood training}
\label{sec:training}

Given a labeled training dataset ${\cal D}=\{(x_k,\y_k), k=1,\ldots,N\}$, the most direct approach to tune the composite weights is to maximize their likelihood (or, equivalently, minimize their ``cross-entropy'') under the composite predictive distribution:
\begin{equation}
\label{eq:train_likelihood}
\max_{\blambda\succeq 0} U(\blambda),
\qquad \text{with} \quad
U(\blambda) \equiv\sum_{k=1}^N \log p_\blambda(x_k|\y_k).
\end{equation}

From~(\ref{eq:log_pool}), we see that the set of conditional distributions $p_\blambda(x|\y)$ spanned by the weights~$\blambda$ is the exponential family with natural parameter~$\blambda$ and basis functions given by the feature-based log-likelihoods:
$$
p_\blambda(x|\y) = \pi(x) \exp[\blambda^\top \bell(x,\y) - a(\blambda,y)],
$$
with:
$$
\ell_i(x,\y) \equiv \log p(z_i|x),
\qquad
a(\blambda, y) \equiv \log \sum_{x\in{\cal X}} \pi(x) e^{\blambda^\top \bell(x,\y)}.
$$

It follows that the utility function $U(\blambda)$ is concave as a general property of likelihood in exponential families, hence the training can be implemented using a standard convex optimization scheme such as the limited-memory BFGS algorithm \cite{Byrd-95}, see Appendix~\ref{appendix:training} for details. 

Because some positive weight constraints may turn out inactive, the training has a natural tendency to produce sparse weights. As illustrated in Figure~\ref{fig:disc_weight_plot} on the breast cancer UCI dataset\footnote{\url{https://archive.ics.uci.edu/ml/datasets/Breast+Cancer+Wisconsin+(Diagnostic)}}, there is no clear relation between optimal weights and feature-level discrimination power, clearly showing that maximum likelihood training differs from univariate feature selection as it takes feature redundancy into account.

\begin{figure}[!ht]
  \begin{center}
    \includegraphics[width=.6\textwidth]{disc_weight_plot.pdf}
  \end{center}
\caption{Distribution of optimal composite weights vs individual feature discrimination power in a binary classification task (breast cancer UCI dataset). Discrimination is measured by the maximum KL divergence between class-conditional generative distributions.}
\label{fig:disc_weight_plot}
\end{figure}

As is customary and is generally good practice, the training examples may be weighted in~(\ref{eq:train_likelihood}) so that the empirical distribution of classes matches the prior~$\pi(x)$ (enforcing balanced classes if the prior is uniform). Also, some regularization may be useful for stability: we shall in practice maximize $U(\lambda)-\alpha \|\blambda\|^2$, where $\alpha>0$ is a small damping factor. 

Other training variants that we mention here but do not particularly recommend for interpretability include adding basis functions to the exponential family. For instance, class-dependent offsets may be added, hence learning the ``prior'' together with the composite weights. It is also possible to perform unconstrained optimization to achieve higher training likelihood, meaning that some composite weights are then negative.


\subsection{Maximum entropy interpretation}

Exponential family properties also imply that maximum likelihood training~(\ref{eq:train_likelihood}) is dual to a maximum entropy problem or, more precisely, an $I$-projection \cite{Csiszar-84}:
$$
\min_{p\in{\cal P}} D(p\|\pi \cdot h) = \sum_{x\in{\cal X}}\sum_{y\in{\cal Y}} p(x,\y) \log \frac{p(x,\y)}{\pi(x)h(\y)},
$$
subject to the constraints $p(\y)=h(\y)$ and $\E_p[\bell] \succeq \E_h[\bell]$, where 
${\cal P}$ is the set of joint probability distributions on ${\cal X}\times {\cal Y}$
with ${\cal Y}=\{\y_1,\ldots,\y_N\}$ the set of training data values, $h(x,\y)$ is the joint empirical distribution of targets and data, and $h(\y)$ is the corresponding marginal data distribution. Note that the constraints implicitly approximate the data marginal and the mean log-likelihood values by empirical estimates, suggesting the need for regularization on small training sets.

The marginal constraint $p(\y)=h(\y)$ makes the $I$-projection a special case of maximum entropy classifier \cite{BergerA-96}, ensuring that only the conditional distribution $p(x|\y)$ is optimized. The mean log-likelihood constraints $\E_p[\bell] \succeq \E_h[\bell]$ encapsulate dependences between the target and the data, and consider a model admissible if it guarantees the same average log-likelihood levels as observed separately for each feature. This is weaker a constraint than assuming that the feature generative models $p(z_i|x)$ are true: it only assumes that the ``description length'' \cite{Grunwald-07} achieved by each agent is truthworthy.

%In the $I$-projection framework, the composite weights are the Lagrange multipliers associated with the mean log-likelihood inequality constraints.

A particularly insightful analysis given in \cite{Grunwald-04} shows that the $I$-projection is under broad conditions a minimax strategy for prediction according to the log loss. Hence, any $I$-projection is in some sense a best possible predictive model under particular constraints (in the limit of validity of empirical approximations). Our training of composite weights belongs to the same family as multinomial logistic regression, another maximum entropy classifier, only differing by the choice of constraints, which, in the latter case, are not related to generative models.

As an alternative to the strategy considered here, the mean log-likelihood constraints may be expressed for each class rather than being averaged w.r.t.~the prior, leading to a more general predictive model that we call ``super composite likelihood'', in which the weights become class-dependent (see Appendix~\ref{appendix:super}). This variant has the potential for higher prediction performance, but poses some conceptual difficulties, in particular that of having to define an arbitrary reference class. 

%used to predict whether breast lumps are benign or malignant based on a set of features extracted from fine-needle aspirate images.

%The most discriminative features tend to get large weight, however some quite discriminative features may get low weight if strongly correlated with a more discriminative feature

%Non-linear means that features with average discrimination power may get low weight as they are superseded by the most discriminative features. But some features with low discrimination power may also get large weight, which does not mean that they are very influential in practice (see above). This is illustrated in Figure~\ref{fig:disc_weight_plot} with the UCI breast cancer dataset. 

%{\color{red} Equality constraints (negative weights) are less interpretable.}

%{\color{red} Offsets.}

%{\color{red} Data weights.}

%Need to weight the data to adjust to prior as a general measure when minimizing cross-entropy, which should yield a conditional model that approximates the ``true'' conditional distribution, $q(x|y)\approx p(x|y)$. So the model will absorb the ``true'' marginal of $x$, which should thus be replaced by the prior if we want the model to mimic ideal Bayesian inference. In our case, it means that the composite likelihood approximates the true likelihood.

%We can further enforce exact prior matching by including offsets ($K-1$ free parameters), in which case the training phase learns $\pi(x)$ so that it matches the prior $\pi(x)$ in the sense $\int h(\y)p_\lambda(x|y)dy = \pi(x)$. This ``deforms'' the prior to improve the training score. But why is it important to enforce this constraint? Again, it all makes things just less interpretable. 

%When including offsets, using homoscedastic generative variance estimation and equality constraints, composite likelihood is equivalent to binomial logistic regression. However, equality constraints are not easily interpretable.

%In some sense, composite weights replace the multiple testing correction required in the frequentist inference paradigm for proper control of false positives. Here, we don't control false positives, we control prediction accuracy.

%Why not using the feature-based posteriors or Bayes factors rather than the likelihoods? It's a possibility (perhaps a connection here with Gr\"unwald's luckiness). A justification for not doing it is that the coordinator should discard the priors used by the agents. The constraints are not purely evidence-based since the moments depend on the coordinator's prior but it is arguable a mess of priors wouldn't be good.

%Single feature case ($y=z$): the maxent solution has the form $p(x|y)\propto\pi(x)p(\y|x)^\lambda$. We expect to have $\lambda=1$ but it's not necessarily the case. It is if $h(\y)=\int\pi(x)p(\y|x)dx$ because then $h(\y)p(x|y)=\pi(x)p(\y|x)$ so the constraint is verified (and active). This also tells us that the data marginal should be consistent with the prior on labels for this much expected consistency property to hold. So, either weight datapoints so as to match a desired prior (e.g., uniform) or take the empirical distribution of labels as the ``prior''. 

%Which brings another question: in this case, the (unique) weight is insensitive to the prior -- is it true in general? The answer is no. The weights generally depend on the prior. Hence the maxent composite likelihood is prior-dependent. This is an important conceptual difference with the classical notion of likelihood.



\section{Discussion}
\label{sec:discussion}

{\color{red} It is easy to build a good classifier. You just need to pick a parametric family that is big enough to produce a good fit yet small enough to avoid overfitting. Less easy is to interpret what the classifier does. Deep learning is cool but it is relevant to AI only if the representations it learns are generic enough to be used in other problems, i.e. it should ultimately be validated via transfer learning.}

{\color{red} Why prediction performance is still important? Because it validates (or not) interpretation. Interpretation is the goal but prediction performance is a crucial intermediate step.}

{\color{red} We cannot rank features according to their weights. A feature might incidentally get a large weight because its likelihood is flat. Feature relevance also depends on conditional likelihood distribution. Consider mutually independent features: each one recieves a unit weight, but some may be independent from target.}

{\color{red} Semi-generative method. Good for shallow learning or transfer learning. Still need representation learning methods.}

{\color{red} Alternative to logistic regression in deep learning? Would require a game-theoretic learning approach... future work.} 

{\color{red} Composite Bayes is invariant by affine transformation of the data (unlike multinomial LR, which is problematic when damping is applied).}

{\color{red} Is damping important? (it guarantees training cost is stricly convex).}

{\color{red} Can we use the same dataset to learn both the feature generative distributions and the composite weights? In principle, nothing speaks against this.} 

Some experimental observations. In the iris dataset (historical data used by Fisher to demonstrate linear discriminant analysis), composite Bayes finds that petal length is the most relevant feature followed by petal width, while sepal length and width get zero weight. In the wine dataset, the flavanoids is the most relevant feature for standard composite Bayes, but it's color intensity closely followed by flavanoids for the super composite version. In the breast cancer, both approaches are equivalent since it is a binary problem, weights are very sparse: only 13/30 are non-zero with worst area the most relevant. In the digits prolem, the most relevant features tend to be in the image center; composite Bayes can achieve similar performance to multinomial logistic regression with about 10 times less parameters.

Number of independent discriminative parameters in logistic regression: $(K-1)(F+1)$. In standard composite inference: $F$ (or $K-1+F$ with offsets). Difference $=(K-2)(F+1)+1$ (or $=(K-2)F$). In super-composite inference: $(K-1)(F+1)$ (if reference is a class). Number of generative parameters in composite inference using univariate Gaussian models: $2K$ with heteroscedasticity, $K+1$ with homoscedasticity.

Intuition behind weights: notion of global feature relevance. But does $\lambda_1>\lambda_2$ imply that $z_1$ is more relevant than $z_2$ for a particular instance? No. But we can assess the influence of each scaled likelihood in order to rank features according to their influence on the decision.

CL is a concept from computational statistics that has mainly been developed so far in a frequentist perspective as a surrogate for the maximum likelihood method. We have shown a deep connection between CL and probabilistic inference, thereby establishing CL as a class of discriminative models. Because CL is built from a set of marginal feature-specific generative distributions, it is in essence a two-step semi-generative, semi-discriminative learning approach. In the first, ``generative'' phase, the feature distributions are learned; in the second, ``discriminative'' phase, the feature weights are learned. This strategy can be thought of as a form of non-adaptive boosting.

%The first phase corresponds to the training of ``weak learners''. The second phase amounts to a form of boosting. 

A purely discriminative learning could be used instead but...
Why potentially more efficient than pure discriminative training: because generative training is always more efficient than discriminative training if models are comparable. So the key point is that ``we have less parameters in the discriminant phase". Good in small datasets. But also in asymptotic regime if the features are weakly or highly correlated (???).

Logistic regression / naive Bayes example. Under homoscedasticity (assuming that each feature has class-independent variance), CBI is equivalent to logistic regression -- because the predictive distribution family is the same. This is a case where BCI brings nothing. But consider heteroscedasticity, then BCI yields a quadratic classifier. Compared to a fully discriminative model, the number of parameters to learned in the discriminative phase is reduced by half.

The first training phase is easier if supervised but could be unsupervised too (using mixture models). How do we then deal with label switching issues? Can we safely assume conditional feature independence {\em in the generative training phase}? I believe so provided that the marginal distribution parameters are disjoint. It's obvious in the supervised learning scenario.

If unsupervised, the first learning phase could be compared with contrastive pre-training of RBMs \cite{Hinton-06,Fischer-14}, which also optimize parameters for generation of observable features. BCI is comparable with an RBM with a single output unit (which is just a generative model assuming conditionial feature independence). The key difference is that the RBM is a full generative model while BCI only deals with marginal models, hence relying on weaker assumptions. This won't change anything in the pre-training phase but RBMs have to deal with more parameters in the generative learning phase. In fully supervised context, RBM pre-training is pointless since all parameters learned in the first phase will be overwritten. In BCI, pre-training is crucial even in supervised context because the parameters learned in the discriminative phase (the feature weighs) describe a sub-manifold of the predictive distribution family.

Needs features. Not a representation learning method, but could be coupled why not.


%{\color{red} Moreover, while RBM parameters are typically refined in a supervised discriminative learning step, disjoint set of parameters for SCL. Dunno how to say that.} 

%In such context, SCL competes with classical discriminative models (logistic regression, Gaussian processes \cite{Rasmussen-06}, maximum entropy models \cite{BergerA-96}, etc.), and may compare more or less favorably in practice depending on the amount of training data. For relatively small training datasets, we may hope for more accurate inference using~SCL than using traditional discriminative models, extrapolating from the results of \cite{Ng-01} regarding the comparison between logistic regression and na\"ive Bayes classifiers.
%{\color{red}Ici, il manque la comparaison avec les RBMs qui ne sont pas (forcement) des modeles discriminatifs.}

%%, hence alleviating the need for heavily supervised model training

%In summary, CL has the potential to yield weakly supervised or unsupervised Bayesian-like inference procedures depending on the particular task at hand. This property reflects the encoding of statistical relationships between the data and {\em all} unknown parameters. CL thus appears as a trade-off between generative models, which are optimal for unsupervised learning but possibly intractable, and traditional discriminative models (logistic regression, Gaussian processes \cite{Rasmussen-06}, maximum entropy models \cite{BergerA-96}, etc.), which are inherently supervised. CL models are discriminative models assembled from atomic generative models and, from this point of view, may be considered as {\em semi-generative} models.

%CL may be considered as a {\em semi-generative} model: a discriminative model assembled from partial generative models.

%As a note of caution, we shall stress that the pre-determined weights assigned to the different associations between observed and unobserved values represent prior knowledge regarding the informativeness of clues. A poor choice of weights will inevitably result in a poor approximation to the ``true'' Bayesian posterior -- the posterior that would be obtained from a realistic generative model if it was tractable. In future work, we will investigate feature selection strategies to mitigate this problem.

% Improve the discussion on following aspects:
% ** Why is it compatible with unsupervised learning? Give more insight.
% ** Stress the contribution: class-specific weighting.
% * Pivotality argument.
% * Bayes is a special case of composite Bayes.

{\color{red}Product of fucking experts \cite{Hinton-02}. Log-linear pool to build a generative model (assuming in fact independence between experts). Better seen as a ``log-mixture model''. Sounds like each expert is associated with a class, so it's the same as a RBM. Contrastive learning approximates ML parameter estimation. The experts are learning jointly. It's just a generative model.}

{\color{red}Two-step training: generative phase then discriminative phase. Why is it cool?}

{\color{red}CBI is just a way to reweight Naive Bayes. What's the big deal? Can we really expect massively superior performance? Are we just talking about realistic credibility sets?}

Future work: can we collapse generative and discriminative training into a single step? Maybe using the concept of {\em minimally informative likelihood} \cite{Yuan-99b}.


\appendix


\section{Basic frequentist properties of composite likelihood}
\label{appendix:frequentist}

Assume that the data~$\y$ is distributed according to some target value~$x_\star$. For any extracted feature~$z_i$ and for any hypothetical target value~$x$,
$$
0 \leq
E\left[
\log \frac{p(z_i|x_\star)}{p(z_i|x)}
\right]
\leq
E\left[
\log \frac{p(\y|x_\star)}{p(\y|x)}
\right],
$$
where the expectation is taken with respect to the true distribution $p(\y|x_\star)$. Using a weighted sum of such inequalities, we can bracket the expected variations of the logarithm of any composite likelihood function:
\begin{equation}
\label{eq:variation_bound}
0 \leq
E\left[ \log \frac{L_c(x_\star, \blambda)}{L_c(x,\blambda)} \right]
\leq 
\|\blambda\|_1 E\left[ \log \frac{L(x_\star)}{L(x)} \right]
.
\end{equation}

This implies two asymptotic properties of standard composite likelihood: 
\begin{itemize}
\item {\em Consistency.} The expected composite log-likelihood is maximized by~$x_\star$.
\item {\em Conservativeness.} For any~$x$, if the expected true log-likelihood of~$x$ is finite, then the corresponding expected composite log-likelihood ratio is finite too. 
\end{itemize}



\section{Conditional composite likelihood}
\label{appendix:conditional}

As a straightforward  extension of marginal CL, each feature-based likelihood may be conditioned by an additional ``independent'' feature $z^c_i = f^c_i(\y)$ considered as a predictor of the ``dependent'' feature, $z_i=f_i(\y)$, yielding the more general form:
\begin{equation}
\label{eq:cond_feat_lik}
\ell_i(x) = p(z_i|x,z^c_i).
\end{equation}

Conditioning may be useful if it is believed that $z^c_i$ alone is little or not informative about $x$, but can provide relevant information when considered jointly with $x$, as in the case of regression covariates, for instance. Equation~(\ref{eq:comp_lik}) then amounts to conditional CL \cite{Varin-11}, a more general form of CL also including Besag's historical {\em pseudo-likelihood} \cite{Besag-74} developed for image segmentation.


\section{Composite likelihood training}
\label{appendix:training}

Using the same notation as in Section~\ref{sec:training}, the likelihood function in~(\ref{eq:train_likelihood}) can be expanded as follows:
$$
U(\blambda) 
= \sum_{k=1}^N \left[
\log \pi(x_k) + \blambda^\top \bell(x_k, \y_k) - a(\blambda,\y_k)
\right]
$$

This reads: $U(\blambda) = \sum_k \log \pi(x_k) + N \psi(\blambda)$ with: 
$$
\psi(\blambda) \equiv \blambda^\top \bar{\bell} - \frac{1}{N} \sum_k a(\lambda,\y_k), 
$$
and:
$$
\bar{\bell} \equiv \frac{1}{N} \sum_k \bell(x_k,\y_k),
$$
therefore maximizing~$U(\blambda)$ is equivalent to maximizing~$\psi(\blambda)$. 

Derivatives...
$$
\nabla\psi
= \bar{\bell} - \frac{1}{N}\sum_k \nabla a(\blambda,\y_k)
$$

$$
\nabla\nabla^\top\psi
= - \frac{1}{N}\sum_k
\nabla \nabla^\top a(\blambda, \y_k)
$$

We have:
$$
\nabla a = \E_{\blambda}(\bell|\y),
\qquad
\nabla \nabla^\top a = \text{Var}_{\blambda}(\bell|\y)
$$
proving that $a(\blambda,\y)$ is convex in $\blambda$ for any $\y$, and in turnes proves the concavity of $\psi$.

To see this, let:
$$
z(\blambda,\y) 
\equiv e^{a(\lambda,\y)}  
= \sum_{x\in{\cal X}} \pi(x) e^{\blambda^\top \bell(x,\y)}
$$

We have:
$$
\nabla a = \frac{\nabla z}{z}
$$

$$
\nabla\nabla^\top a 
= \left(
\frac{\nabla \nabla^\top z}{z} 
- \frac{\nabla z \nabla z^\top}{z^2}
\right)
$$

$$
\nabla z 
=
\sum_{x\in{\cal X}} \pi(x) \bell(x,\y)  e^{\blambda^\top \bell(x,\y)}
$$

$$
\nabla \nabla^\top z 
=
\sum_{x\in{\cal X}} \pi(x) \bell(x,\y) \bell(x,\y)^\top e^{\blambda^\top \bell(x,\y)}
$$


\section{Super composite likelihood}
\label{appendix:super}

When chosen for computational simplicity, features may not only convey limited information at individual level, their informativeness may also be very much class-dependent. Consider, for instance, diagnosing a disease from a routine medical checkup. Body temperature may point to a bacterial infection by comparison with normality, but would not help detecting a non-infectious cardiovasc ular disease -- and conversely for, say, blood pressure. 

This motivates a more general pooling than the common log-linear pool~(\ref{eq:log_pool}) where features can be weighted depending on the target variable:
$$
p_\lambda(x|\y) \propto \pi(x) \prod_i \left[\frac{p(z_i|x)}{p_0(z_i)}\right]^{\lambda_i(x)},
$$
where $p_0$ is an arbitrary reference feature distribution (which is implicit if weights are class-independent as it is then effectless). Make sense to have a reference as it gives a customizable baseline for detection. Can be the feature distribution averaged over all classes (not recommended I would say) or the distribution of a particular class if that makes sense (e.g. healthy population). In this case, the super composite likelihood is equivalent to the standard one in the two-class case.

We can train this model by maximum likelihood, yielding a similar optimization problem as in Section~\ref{sec:training}. 

%If some classes are merged or split, then the other classes keep the same weights (assuming $h(\y)$ is kept constant, so we need to make sure the prior is adapted accordingly)! So if a patient was found to have 10 times more probability to be MCI than healthy (and healthy is the reference class), and we later split AD population into, say, pure AD and mixed AD, the MCI probably ratio fact will remain.

Super composite likelihood is essentially about combining binary versions of composite likelihood. It is analogous to the ``one-versus-rest'' strategy in multiple logistic regression. 

Link with PDF projection \cite{Baggenstoss-03}.

Advantages: richer model (so better predictions), probabilty ratios stable to class addition. Drawbacks: require a reference, no more consistency. Still interpretable, that's why it is worthwile mentioning.

%I was thinking about the unit sum weight version... The diagnosis example could be one in which clues are independent, in which case $\lambda_i(x)\equiv 1$ would be optimal, making any super composite trick irrelevant. Also see that with hypothesis-dependent weights, we may lose the estimation consistency. At the end of the day, the interest of the super composite idea is questionable.



\section{Log-likelihood ratio}

$$
\log r = \log \frac{N(x;\mu_1,\sigma_1)}{N(x;\mu_0,\sigma_0)}
$$

$$
\log N(x;\mu,\sigma) = -\frac{1}{2}\left[
\log(2\pi\sigma^2) + \frac{(x-\mu)^2}{\sigma^2}
\right]
$$

$$
\log r = -\frac{1}{2}
\left[
\log \frac{\sigma_1^2}{\sigma_0^2}
+ \frac{(x-\mu_1)^2}{\sigma_1^2} - \frac{(x-\mu_0)^2}{\sigma_0^2}
\right]
$$

It follows:
\begin{eqnarray*}
\log r 
 & = & 
-\frac{1}{2}\left[
\log \frac{\sigma_1^2}{\sigma_0^2}
+ \frac{\sigma_0^2(x-\mu_1)^2 - \sigma_1^2(x-\mu_0)^2}{\sigma_0^2 \sigma_1^2}
\right] \\
 & = & 
\frac{1}{2}\left[
\log \rho^2
+ \frac{(\rho^2-1)x^2 - 2(\rho^2\mu_0-\mu_1)x+ (\rho^2\mu_0 ^2-\mu_1^2)}{\sigma_0^2}
\right]
\end{eqnarray*}
with $\rho=\sigma_1/\sigma_0$. In the case of same variance ($\rho=1$), this simplies to an affine function of $x$ with offset given by the center of $(\mu_0,\mu_1)$:
$$
\log r = \frac{\mu_1-\mu_0}{\sigma^2} \left(
x - \frac{\mu_0 + \mu_1}{2}
\right) 
$$

In any case,
$$
E[\log r | H_1] = 
\frac{1}{2}
\left[
- \log \frac{\sigma_1^2}{\sigma_0^2}
+ \frac{\sigma_1^2 + (\mu_1-\mu_0)^2}{\sigma_0^2}
- 1
\right]
$$

Let $\delta = (\mu_1-\mu_0)/\sigma_0$. We have:
$$
E[\log r | H_1] = 
\frac{1}{2}
\left[
\delta^2
- \log \rho^2
+ \rho^2  - 1
\right]
$$




\bibliographystyle{abbrv}
\bibliography{cvis,stat,alexis}

%%\input{draft.biblio}

\end{document}


\end{document}


\end{document}


\end{document}
